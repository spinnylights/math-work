\documentclass[12pt]{article}
\usepackage{mathtools}
\usepackage{amsthm}
\usepackage{amsfonts}
\usepackage{amssymb}
\usepackage{fontspec}
\usepackage{xfrac}
\usepackage{array}
\usepackage{siunitx}
\usepackage{gensymb}
\usepackage{enumitem}
\usepackage{dirtytalk}
\usepackage{bm}
\title{Sets}
\author{Zoë Sparks}

\begin{document}

\theoremstyle{definition}

\sisetup{quotient-mode=fraction}
\newtheorem{thm}{Theorem}
\newtheorem*{nthm}{Theorem}
\newtheorem{sthm}{}[thm]
\newtheorem{lemma}{Lemma}[thm]
\newtheorem*{nlemma}{Lemma}
\newtheorem{cor}{Corollary}[thm]
\newtheorem*{prop}{Property}
\newtheorem*{defn}{Definition}
\newtheorem*{comm}{Comment}
\newtheorem*{exm}{Example}

\maketitle

\begin{defn}
  A \textbf{set} is an object made up of objects. It can also be referred to as a
  \textbf{class}, \textbf{collection}, or \textbf{family} of objects. If $S$ is a set
  and $x$ is an object in $S$, we may say that $x$ is a \textbf{member of} $S$, an
  \textbf{element of} $S$, that $x$ \textbf{belongs to} $S$, or just that $x$ is
  \textbf{in} $S$. If $x$ is in $S$, this can be represented by $x \in S$, or by $S
  \ni x$. If $x$ is not in $S$, this can be represented by $x \notin S$.

  $S$ may be finite or infinite. If $S$ is finite, it can be described by displaying
  its members inside braces:
  \begin{align*}
    S = \{x_1,\ldots,x_n\}.
  \end{align*}
\end{defn}

\begin{exm}
  The set $S$ of positive integers from $1$ to $5$ would be
  \begin{align*}
    S = \{1,2,3,4,5\}.
  \end{align*}
\end{exm}

\begin{defn}
  If $S$ and $T$ are sets, $S$ is called a \textbf{subset of} $T$, or said to be
  \textbf{contained in} $T$, if each member of $S$ is also a member of $T$. Any set
  is a subset of itself. If $S$ is a subset of $T$ and $T$ differs from $S$, $S$ is
  then called a \textbf{proper subset} of $T$. In other words, $S$ is a proper subset
  of $T$ if $S$ is contained in $T$ but $T$ is not contained in $S$.
\end{defn}

\begin{exm}
  If
  \begin{align*}
    S &= \{1,2,3,4,5\}\\
    \shortintertext{and}
    T &= \{0,1,2,3,4,5\},
  \end{align*}
  then $S$ is a subset of $T$.
\end{exm}

\begin{defn}
  If $S \subset T$ and $T \subset S$, we say that $S$ and $T$ are \textbf{equal}, and
  we write $S = T$. Otherwise, we write $S \neq T$.
\end{defn}

\begin{defn}
  If $S$ and $T$ are sets, the \textbf{union of $S$ and $T$} is the set $S \cup T$,
  which contains all objects $x$ that are members of either $S$ or $T$.

  \begin{exm}
    If
    \begin{align*}
      S &= \{0,1,2,3,4,5\}\\
      \shortintertext{and}
      T &= \{-5,-4,-3,-2,-1,0\},
    \end{align*}
    then
    \begin{align*}
      S \cup T &= \{-5,-4,-3,-2,-1,0,1,2,3,4,5\}.
    \end{align*}
  \end{exm}

  If $S_1,\ldots,S_n$ are sets, their \textbf{union} is the set $\bigcup_{j =
  1}^{n}S_j$, which consists of all $x$ which are members of at least one of
  $S_1,\ldots,S_n$.
\end{defn}

\begin{comm}
  If $x$ is said to be either in $S$ or in $T$, this does not exclude the possibility
  that $x$ is in both.
\end{comm}

\begin{defn}
  If $S$ and $T$ are sets, the \textbf{intersection of $S$ and $T$} is the set $S
  \cap T$, which contains all objects $x$ that are members of both $S$ and $T$.

  \begin{exm}
    If
    \begin{align*}
      S &= \{0,1,2,3,4,5\}\\
      \shortintertext{and}
      T &= \{-5,-4,-3,-2,-1,0\},
    \end{align*}
    then
    \begin{align*}
      S \cap T &= \{0\}.
    \end{align*}
  \end{exm}

  If $S_1,\ldots,S_n$ are sets, their \textbf{intersection} is the set $\bigcap_{j =
  1}^{n}S_j$, which consists of all $x$ which are members of each of
  $S_1,\ldots,S_n$.
\end{defn}

\begin{defn}
  The \textbf{empty set} is the set which contains no members. If and only if $S$ and
  $T$ have no members in common, $S \cap T$ is the empty set.
\end{defn}

\begin{exm}
  If $t$ is in $\mathbb{R}$, we define a subset $S_t$ of $\mathbb{R}$ which consists
  of all $x \nless t,\ x \in \mathbb{R}$. Then
  \begin{enumerate}
    \item
      $S_{t_1} \cup S_{t_2} = S_t$, where $t$ is the smaller of $t_1$ and $t_2$,
    \item
      $S_{t_1} \cap S_{t_2} = S_t$, where $t$ is the larger of $t_1$ and $t_2$, and
    \item
      if $I$ is the unit interval, i.e. the set of all $t$ in $\mathbb{R}$ such that
      $0 \leq t \leq 1$, then
      \begin{align*}
        \bigcup_{t \in I}S_t &= S_0\\
        \shortintertext{and}
        \bigcap_{t \in I}S_t &= S_1.
      \end{align*}
  \end{enumerate}
\end{exm}

\begin{defn}
  Let $S$ be a set. An \textbf{order} on $S$ is a relation, represented by $<$, for
  which both of these properties hold.
  \begin{enumerate}
    \item
      If $x,y \in S$, then one and only one of
      \begin{align*}
        x < y,\ \ \ x = y,\ \ \ y < x
      \end{align*}
      is true.
    \item
      If $x,y,z \in S$, $x < y$, and $y < z$, then $x < z$.
  \end{enumerate}

  If $x < y$, we say that $x$ is \textbf{less than} $y$, \textbf{smaller than} $y$,
  or that $x$ \textbf{precedes} $y$.

  If $x < y$, we may also write $y > x$ to represent the same relation.

  If we write $x \leq y$, this indicates that either $x < y$ or $x = y$, but without
  specifying which of the two is true. $x \leq y$ is the negation of $x > y$.

  If $S$ has an order defined in it, then we say that $S$ is an \textbf{ordered set}.
\end{defn}

\begin{exm}
  $\mathbb{Q}$ is an ordered set if the relation $r < s$ is defined to mean $s - r
  \in \mathbb{Q}$ is positive, i.e. that $s - r > 0$.
\end{exm}

\begin{defn}
  Let $S$ be an ordered set, and let $E \subset S$. If there is some $\beta \in S$
  such that $x \leq \beta$ for every $x \in E$, $E$ is then said to be
  \textbf{bounded above}, and $\beta$ is then called an \textbf{upper bound} of $E$.
  If $x \geq \beta$ for every $x \in E$, $E$ is then said to be \textbf{bounded
  below}, and $\beta$ is then called a \textbf{lower bound} of $E$.
\end{defn}

\end{document}
