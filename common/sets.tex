\documentclass[12pt]{article}
\usepackage{mathtools}
\usepackage{amsthm}
\usepackage{amsfonts}
\usepackage{amssymb}
\usepackage{fontspec}
\usepackage{xfrac}
\usepackage{array}
\usepackage{siunitx}
\usepackage{gensymb}
\usepackage{enumitem}
\usepackage{dirtytalk}
\usepackage{bm}
\title{Sets}
\author{Zoë Sparks}

\begin{document}

\theoremstyle{definition}

\sisetup{quotient-mode=fraction}
\newtheorem{thm}{Theorem}
\newtheorem*{nthm}{Theorem}
\newtheorem{sthm}{}[thm]
\newtheorem{lemma}{Lemma}[thm]
\newtheorem*{nlemma}{Lemma}
\newtheorem{cor}{Corollary}[thm]
\newtheorem*{prop}{Property}
\newtheorem*{defn}{Definition}
\newtheorem*{comm}{Comment}
\newtheorem*{exm}{Example}

\maketitle

\begin{defn}
  A \textbf{set} is an object made up of objects. It can also be referred to as a
  \textbf{class}, \textbf{collection}, or \textbf{family} of objects. If $S$ is a set
  and $x$ is an object in $S$, we may say that $x$ is a \textbf{member of} $S$, an
  \textbf{element of} $S$, that $x$ \textbf{belongs to} $S$, or just that $x$ is
  \textbf{in} $S$. If $x$ is in $S$, this can be represented by $x \in S$, or by $S
  \ni x$. If $x$ is not in $S$, this can be represented by $x \notin S$.

  $S$ may be finite or infinite. If $S$ is finite, it can be described by displaying
  its members inside braces:
  \begin{align*}
    S = \{x_1,\ldots,x_n\}.
  \end{align*}
\end{defn}

\begin{exm}
  The set $S$ of positive integers from $1$ to $5$ would be
  \begin{align*}
    S = \{1,2,3,4,5\}.
  \end{align*}
\end{exm}

\begin{defn}
  If $S$ and $T$ are sets, $S$ is called a \textbf{subset of} $T$, or said to be
  \textbf{contained in} $T$, if each member of $S$ is also a member of $T$. Any set
  is a subset of itself. If $S$ is a subset of $T$ and $T$ differs from $S$, $S$ is
  then called a \textbf{proper subset} of $T$. In other words, $S$ is a proper subset
  of $T$ if $S$ is contained in $T$ but $T$ is not contained in $S$.
\end{defn}

\begin{exm}
  If
  \begin{align*}
    S &= \{1,2,3,4,5\}\\
    \shortintertext{and}
    T &= \{0,1,2,3,4,5\},
  \end{align*}
  then $S$ is a subset of $T$.
\end{exm}

\begin{defn}
  If $S \subset T$ and $T \subset S$, we say that $S$ and $T$ are \textbf{equal}, and
  we write $S = T$. Otherwise, we write $S \neq T$.
\end{defn}

\begin{defn}
  If $S$ and $T$ are sets, the \textbf{union of $S$ and $T$} is the set $S \cup T$,
  which contains all objects $x$ that are members of either $S$ or $T$.

  \begin{exm}
    If
    \begin{align*}
      S &= \{0,1,2,3,4,5\}\\
      \shortintertext{and}
      T &= \{-5,-4,-3,-2,-1,0\},
    \end{align*}
    then
    \begin{align*}
      S \cup T &= \{-5,-4,-3,-2,-1,0,1,2,3,4,5\}.
    \end{align*}
  \end{exm}

  If $S_1,\ldots,S_n$ are sets, their \textbf{union} is the set $\bigcup_{j =
  1}^{n}S_j$, which consists of all $x$ which are members of at least one of
  $S_1,\ldots,S_n$.
\end{defn}

\begin{comm}
  If $x$ is said to be either in $S$ or in $T$, this does not exclude the possibility
  that $x$ is in both.
\end{comm}

\begin{defn}
  If $S$ and $T$ are sets, the \textbf{intersection of $S$ and $T$} is the set $S
  \cap T$, which contains all objects $x$ that are members of both $S$ and $T$.

  \begin{exm}
    If
    \begin{align*}
      S &= \{0,1,2,3,4,5\}\\
      \shortintertext{and}
      T &= \{-5,-4,-3,-2,-1,0\},
    \end{align*}
    then
    \begin{align*}
      S \cap T &= \{0\}.
    \end{align*}
  \end{exm}

  If $S_1,\ldots,S_n$ are sets, their \textbf{intersection} is the set $\bigcap_{j =
  1}^{n}S_j$, which consists of all $x$ which are members of each of
  $S_1,\ldots,S_n$.
\end{defn}

\begin{defn}
  The \textbf{empty set} is the set which contains no members. If and only if $S$ and
  $T$ have no members in common, $S \cap T$ is the empty set.
\end{defn}

\begin{exm}
  If $t$ is in $\mathbb{R}$, we define a subset $S_t$ of $\mathbb{R}$ which consists
  of all $x \nless t,\ x \in \mathbb{R}$. Then
  \begin{enumerate}
    \item
      $S_{t_1} \cup S_{t_2} = S_t$, where $t$ is the smaller of $t_1$ and $t_2$,
    \item
      $S_{t_1} \cap S_{t_2} = S_t$, where $t$ is the larger of $t_1$ and $t_2$, and
    \item
      if $I$ is the unit interval, i.e. the set of all $t$ in $\mathbb{R}$ such that
      $0 \leq t \leq 1$, then
      \begin{align*}
        \bigcup_{t \in I}S_t &= S_0\\
        \shortintertext{and}
        \bigcap_{t \in I}S_t &= S_1.
      \end{align*}
  \end{enumerate}
\end{exm}

\begin{defn}
  If $S$ and $T$ are sets, and $S \cap T$ is the empty set, $S$ and $T$ are said to
  be \textbf{disjoint}.
\end{defn}

\begin{defn}
  Let $S$ be a set. An \textbf{order} on $S$ is a relation, represented by $<$, for
  which both of these properties hold.
  \begin{enumerate}
    \item
      If $x,y \in S$, then one and only one of
      \begin{align*}
        x < y,\ \ \ x = y,\ \ \ y < x
      \end{align*}
      is true.
    \item
      If $x,y,z \in S$, $x < y$, and $y < z$, then $x < z$.
  \end{enumerate}

  If $x < y$, we say that $x$ is \textbf{less than} $y$, \textbf{smaller than} $y$,
  or that $x$ \textbf{precedes} $y$.

  If $x < y$, we may also write $y > x$ to represent the same relation.

  If we write $x \leq y$, this indicates that either $x < y$ or $x = y$, but without
  specifying which of the two is true. $x \leq y$ is the negation of $x > y$.

  If $S$ has an order defined in it, then we say that $S$ is an \textbf{ordered set}.
\end{defn}

\begin{exm}
  $\mathbb{Q}$ is an ordered set if the relation $r < s$ is defined to mean $s - r
  \in \mathbb{Q}$ is positive, i.e. that $s - r > 0$.
\end{exm}

\begin{defn}
  Let $S$ be an ordered set, and let $E \subset S$. If there is some $\beta \in S$
  such that $x \leq \beta$ for every $x \in E$, $E$ is then said to be
  \textbf{bounded above}, and $\beta$ is then called an \textbf{upper bound} of $E$.
  If $x \geq \beta$ for every $x \in E$, $E$ is then said to be \textbf{bounded
  below}, and $\beta$ is then called a \textbf{lower bound} of $E$.
\end{defn}

\begin{defn}
  Let $S$ be an ordered set, $E \subset S$, and $E$ be bounded above. Let $\alpha \in
  S$ have the following properties:
  \begin{enumerate}
    \item
      $\alpha$ is an upper bound of $E$, and
    \item
      if $\gamma < \alpha$ then $\gamma$ is not an upper bound of $E$.
  \end{enumerate}
  Then we call $\alpha$ the \textbf{least upper bound} of $E$, or the
  \textbf{supremum} of $E$, and we write
  \begin{align*}
    \alpha = \sup E.
  \end{align*}

  If
  \begin{enumerate}
    \item
      $\alpha$ is a lower bound of $E$, and
    \item
      if $\beta > \alpha$ then $\beta$ is not a lower bound of $E$,
  \end{enumerate}
  then we call $\alpha$ the \textbf{greatest lower bound} of $E$, or the
  \textbf{infimum} of $E$, and we write
  \begin{align*}
    \alpha = \inf E.
  \end{align*}
\end{defn}

\begin{exm}
  Let $P = \{p \in \mathbb{Q}: p > 0\}$. Let $S = \{p \in P: p^2 < 2\}$, and let $T =
  \{p \in P: p^2 > 2\}$. Then $S$ is bounded above; as $S$ and $T$ are both subsets
  of $P$, every member of $T$ is an upper bound of $S$ by their definitions.
  Likewise, $T$ is bounded below, with every member of $S$ being a lower bound of
  $T$.

  Let $f$ be a function over $P$ defined as $f(x) = x - \frac{x^2 - 2}{x + 2}$; then
  $f: P \to P$, i.e. $f(x)$ is always
  positive:
  \begin{align*}
    x - \frac{x^2 - 2}{x + 2} &= \frac{x(x +2)}{x + 2} - \frac{x^2 - 2}{x + 2}\\
    &= \frac{x^2 + 2x}{x + 2} - \frac{x^2 - 2}{x + 2}\\
    &= \frac{x^2 + 2x - (x^2 - 2)}{x + 2}\\
    &= \frac{x^2 + 2x - x^2 + 2}{x + 2}\\
    &= \frac{2x + 2}{x + 2}.
  \end{align*}
  $x$ is always positive by definition, so $\frac{2x + 2}{x + 2}$ is as well.

  Furthermore, note that
  \begin{align*}
    f(x)^2 - 2 &= \left(\frac{2x + 2}{x + 2}\right)^2 - 2\\
    &= \frac{(2x + 2)^2}{(x + 2)^2} - 2\\
    &= \frac{(2x + 2)^2}{(x + 2)^2} - \frac{2(x + 2)^2}{(x + 2)^2}\\
    &= \frac{(4x^2 + 8x + 4) - 2(x^2 + 4x + 4)}{(x + 2)^2}\\
    &= \frac{4x^2 + 8x + 4 - 2x^2 - 8x - 8}{(x + 2)^2}\\
    &= \frac{(4x^2 - 2x^2) + (8x - 8x) + (4 - 8)}{(x + 2)^2}\\
    &= \frac{2x^2 - 4}{(x + 2)^2}\\
    &= \frac{2(x^2 - 2)}{(x + 2)^2}.
  \end{align*}

  If $p \in S$, we know that $p^2 - 2 < 0$. That implies that $\frac{2(p^2 - 2)}{(p +
  2)^2} < 0$, so $f(p)^2 - 2 < 0$, i.e. $f(p)^2 < 2$, and thus $f(p) \in S$ (remember
  that $f(p)$ is always positive). Also, $f(p) > p$; if $p^2 - 2$ is negative,
  $\frac{p^2 - 2}{p + 2}$ is negative, so $-\frac{p^2 - 2}{p + 2}$ is positive, and
  so is $p - \frac{p^2 - 2}{p + 2} = f(p)$.

  Therefore, $S$ has no largest member; if $p \in S$, there is a number $f(p) \in S$
  such that $f(p) > p$. Another way of putting this is to say that $S$ has no least
  upper bound, or no supremum, despite being bounded above.

  If $p \in T$, then $p^2 - 2 > 0$, which implies that $\frac{2(p^2 - 2)}{(p + 2)^2}$
  is positive. That implies that $\frac{2(p^2 - 2)}{(p + 2)^2} = f(p)^2 - 2 > 0$, so
  $f(p)^2 > 2$, and so $f(p) \in T$. Also, $f(p) < p$; if $p^2 - 2$ is positive,
  $\frac{p^2 - 2}{p + 2}$ is positive, so $p - \frac{p^2 - 2}{p + 2} = f(p) < p$.

  Therefore, $T$ has no smallest member; if $p \in T$, there is a number $f(p) \in T$
  such that $f(p) < p$. Another way of putting this is to say that $T$ has no
  greatest lower bound, or no infimum, despite being bounded below.

  As a side note, this shows that even beyond the fact that $\sqrt{2} \notin
  \mathbb{Q}$, there is also no $p \in \mathbb{Q}$ such that $p$ is "as close as
  possible" to $\sqrt{2}$, either from above or from below.
\end{exm}

\begin{exm}
  If $\alpha = \sup E$ exists, $\alpha$ is not necessarily a member of $E$, although
  it may be. As an example, let $E_1 = \{r \in \mathbb{Q}: r < 0\}$, and let $E_2 =
  \{r \in \mathbb{Q}: r \leq 0\}$. Then
  \begin{align*}
    \sup E_1 = \sup E_2 = 0,
  \end{align*}
  and although $0 \in E_2$, $0 \notin E_1$.
\end{exm}

\begin{exm}
  Let $E = \{n \in \mathbb{N}^+: \frac{1}{n}\}$. Then $\sup E = 1$, which is in $E$
  ($1 = \frac{1}{1}$).  However, $\inf E = 0$, which is not in $E$. This is because,
  if $p \in E$, there is a number $q \in E$ such that if $p = \frac{1}{n}$ for some
  $n \in \mathbb{N}^+$, $q = \frac{1}{n + 1}$, and therefore $q < p$. However, there
  is no $n \in \mathbb{N}^+$ such that $\frac{1}{n} = 0$, so $q$ will always be
  greater than $0$, no matter how close it is.
\end{exm}

\begin{defn}
  If $S$ is an ordered set, $S$ is said to have the \textbf{least-upper-bound
  property} if, if $E \subset S$, $E$ is not empty, and $E$ is bounded above, $\sup E
  \in S$.
\end{defn}

\begin{exm}
  As shown above, $\mathbb{Q}$ does not have the least-upper-bound property.
\end{exm}

\begin{thm}
  Let $S$ be an ordered set with the least-upper-bound property, $B \subset S$, $B
  \neq \emptyset$, and $B$ be bounded below. Let $L$ be the set of all lower bounds
  of $B$. Then
  \begin{align*}
    \alpha = \sup L
  \end{align*}
  exists in $S$, and $\alpha = \inf B$, i.e. $\inf B$ exists is $S$.

  \begin{proof}
    Because $B$ is bounded below, $L$ is not empty. Furthermore, if $y \in L$, $y
    \leq x$ for every $x \in B$, and therefore $L$ is bounded above by the members of
    $B$. Also, $y \in S$ by the definition of an upper bound, so $L \subset S$. Since
    $S$ has the least-upper-bound property, $\alpha = \sup L$ is then in $S$.

    If $y < \alpha$ for some $y \in S$, then $y \notin B$, because $y$ is then not an
    upper bound of $L$, and the members of $B$ are all upper bounds of $L$. Therefore
    $\alpha \leq x$ for every $x \in B$, so $\alpha$ is a lower bound of $B$, i.e.
    $\alpha \in L$.

    Since $\alpha \leq x$ for every $x \in B$, it follows that if some $x \in B$ is
    such that $x > \alpha$, then $x \notin L$, because $y \leq \alpha$ for every $y
    \in L$. Therefore $x$ is not a lower bound of $B$, and thus $\alpha = \inf B$.
  \end{proof}
\end{thm}

\begin{defn}
  If $S$ and $T$ are sets, and there exists a 1:1 mapping of $S$ onto $T$, we say
  that $S$ and $T$ can be put into \textbf{1:1 correspondence}, that $S$ and $T$ have
  the same \textbf{cardinal number}, or that $S$ and $T$ are \textbf{equivalent}. If
  this is the case, we write $S \sim T$. This definition of equivalence between sets
  is an equivalence relation.
\end{defn}

\begin{defn}
  If $S$ is a set, we say that
  \begin{enumerate}
    \item
      $S$ is \textbf{finite} if $S = \emptyset$ or $S \sim J_n$, where $J_n =
      \{1,2,\ldots,n\}$ for some $n \in \mathbb{N}+$;
    \item
      $S$ is \textbf{infinite} if $S$ is not finite;
    \item
      $S$ is \textbf{countable}, \textbf{enumerable}, or \textbf{denumerable} if $S
      \sim \mathbb{N}^+$;
    \item
      $S$ is \textbf{uncountable} if $S$ is neither finite nor countable;
    \item
      $S$ is \textbf{at most countable} if $S$ is finite or countable.
  \end{enumerate}
\end{defn}

\begin{comm}
  If $S$ and $T$ are finite sets, then clearly $S \sim T$ if and only if $S$ and $T$
  contain the same number of elements. This strategy is difficult to apply if $S$ and
  $T$ are infinite, but the notion of 1:1 correspondence can still be used in that
  case.
\end{comm}

\begin{exm}
  $\mathbb{Z}$ is countable. Casually,
  \begin{align*}
    \mathbb{Z}:&\ \ \ 0,1,-1,2,-2,3,-3,\ldots\\
    \mathbb{N}^+:&\ \ \ 1,2,3,4,5,6,7,\ldots
  \end{align*}
  More rigorously, a function $f: \mathbb{N}^+ \to \mathbb{Z}$ which sets up a 1:1
  correspondence between $\mathbb{Z}$ and $\mathbb{N}^+$ is given by
  \begin{align*}
    f(n) =
    \begin{cases}
      \frac{n}{2},\ \ \ \ \ \ n\ \text{even}\\
      -\frac{n-1}{2},\ n\ \text{odd}.
    \end{cases}
  \end{align*}
\end{exm}

\begin{comm}
  A finite set cannot be equivalent to one of its proper subsets, clearly. However,
  an infinite set can; after all, $\mathbb{N}^+$ is a proper subset of $\mathbb{Z}$.
  In fact, an infinite set could be defined as a set that is equivalent to one of its
  proper subsets.
\end{comm}

\begin{thm}
  Every infinite subset of a countable set $S$ is countable.

  \begin{proof}
    Let $E \subset S$ be infinite. Consider a sequence $\{x_n\}$ of elements $x \in
    S$. Because $E \subset S$, there is some $i \in \mathbb{N}^+$ such that $x_i$ is
    the first element in $\{x_n\}$ that is contained in $E$. Let $p_1 = i$. We can define a sequence
    $\{x_{p_k}\},\ x \in \{x_n\}$ of elements in $E$ by selecting $x_i,\ldots,x_{k-1},\
    (k = 2,3,4,\ldots)$ such that $p_k$ is the smallest index of an element in $E$
    greater than $p_{k-1}$. Then a function $f: \mathbb{N}^+ \to E$ defined by $f(k)
    = x_{p_k}$ for $x \in \{x_n\}$ creates a 1:1 correspondence from $\mathbb{N}^+$
    to $E$, so $E$ is countable.
  \end{proof}
\end{thm}

\begin{comm}
  This theorem shows that countability of sets is the \say{smallest} form of infinity
  in a sense, as an uncountable set cannot be a subset of a countable set.
\end{comm}

\end{document}
