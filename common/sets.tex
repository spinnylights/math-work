\documentclass[12pt]{article}
\usepackage{mathtools}
\usepackage{amsthm}
\usepackage{amsfonts}
\usepackage{amssymb}
\usepackage{fontspec}
\usepackage{xfrac}
\usepackage{array}
\usepackage{siunitx}
\usepackage{gensymb}
\usepackage{enumitem}
\usepackage{dirtytalk}
\usepackage{bm}
\title{Sets}
\author{Zoë Sparks}

\begin{document}

\theoremstyle{definition}

\sisetup{quotient-mode=fraction}
\newtheorem{thm}{Theorem}
\newtheorem*{nthm}{Theorem}
\newtheorem{sthm}{}[thm]
\newtheorem{lemma}{Lemma}[thm]
\newtheorem*{nlemma}{Lemma}
\newtheorem{cor}{Corollary}[thm]
\newtheorem*{prop}{Property}
\newtheorem*{defn}{Definition}
\newtheorem*{comm}{Comment}
\newtheorem*{exm}{Example}

\maketitle

\begin{defn}
  A \textbf{set} is an object made up of objects. It can also be referred to as a
  \textbf{class}, \textbf{collection}, or \textbf{family} of objects. If $S$ is a set
  and $x$ is an object in $S$, we may say that $x$ is a \textbf{member of} $S$, an
  \textbf{element of} $S$, that $x$ \textbf{belongs to} $S$, or just that $x$ is
  \textbf{in} $S$. If $x$ is in $S$, this can be represented by $x \in S$, or by $S
  \ni x$.

  $S$ may be finite or infinite. If $S$ is finite, it can be described by displaying
  its members inside braces:
  \begin{align*}
    S = \{x_1,\ldots,x_n\}.
  \end{align*}
\end{defn}

\begin{exm}
  The set $S$ of positive integers from $1$ to $5$ would be
  \begin{align*}
    S = \{1,2,3,4,5\}.
  \end{align*}
\end{exm}

\begin{defn}
  If $S$ and $T$ are sets, $S$ is called a \textbf{subset of} $T$, or said to be
  \textbf{contained in} $T$, if each member of $S$ is also a member of $T$. Any set
  is a subset of itself. If $S$ is a subset of $T$ and $T$ differs from $S$, $S$ is
  then called a \textbf{proper subset} of $T$. In other words, $S$ is a proper subset
  of $T$ if $S$ is contained in $T$ but $T$ is not contained in $S$.
\end{defn}

\begin{exm}
  If
  \begin{align*}
    S &= \{1,2,3,4,5\}\\
    \shortintertext{and}
    T &= \{0,1,2,3,4,5\},
  \end{align*}
  then $S$ is a subset of $T$.
\end{exm}

\end{document}
