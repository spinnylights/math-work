\documentclass[12pt]{article}
\usepackage{mathtools}
\usepackage{amsthm}
\usepackage{amsfonts}
\usepackage{amssymb}
\usepackage{fontspec}
\usepackage{xfrac}
\usepackage{array}
\usepackage{siunitx}
\usepackage{gensymb}
\usepackage{enumitem}
\usepackage{dirtytalk}
\usepackage{bm}
\title{Functions}
\author{Zoë Sparks}

\begin{document}

\theoremstyle{definition}

\sisetup{quotient-mode=fraction}
\newtheorem{thm}{Theorem}
\newtheorem*{nthm}{Theorem}
\newtheorem{sthm}{}[thm]
\newtheorem{lemma}{Lemma}[thm]
\newtheorem*{nlemma}{Lemma}
\newtheorem{cor}{Corollary}[thm]
\newtheorem*{prop}{Property}
\newtheorem*{defn}{Definition}
\newtheorem*{comm}{Comment}
\newtheorem*{exm}{Example}

\maketitle

\begin{defn}
  A \textbf{function} is a composite object consisting of:
  \begin{enumerate}
    \item
      a set $X$, called the \textbf{domain} of the function,
    \item
      a set $Y$, called the \textbf{co-domain} of the function, and
    \item
      a rule (or correspondence) $f$, which associates each element $x$ in $X$ with a
      single element $f(x)$ in $Y$.
  \end{enumerate}

  If $(X,Y,f)$ is a function, we may say that \textbf{$f$ is a function from $X$ into
  $Y$}. This is imprecise, because $f$ is actually the rule of the function, not the
  function itself. However, it makes speaking about functions much easier if we use
  the same symbol for the rule and its function. So, if we say that $f$ is a function
  from $X$ into $Y$, that $X$ is the domain of $f$, and that $Y$ is the co-domain of
  $f$, we mean all together that $(X,Y,f)$ is a function by the above definition.

  A function $f$ can also be defined as
  \begin{enumerate}
    \item
      a set of ordered pairs of objects $(a,b)$ such that if $(a,b)$ and $(a,c)$ are
      in $f$, $b = c$, and
    \item
      a set $Y$ such that if $(a,b) \in f$, $b \in Y$.
  \end{enumerate}
  Then if $(a,b) \in f$ we say that $f(a) = b$. This is equivalent to the above
  formulation; $X$ (the domain of $f$) is then the set of all $a$ such that $(a,b)
  \in f$.

  A function may also be called a \textbf{transformation}, \textbf{operator},
  \textbf{mapping}, or another similar term when this is more expressive than simply
  saying \say{function.}
\end{defn}

\end{document}
