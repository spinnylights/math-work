\documentclass[12pt]{article}
\usepackage{mathtools}
\usepackage{amsthm}
\usepackage{amsfonts}
\usepackage{amssymb}
\usepackage{fontspec}
\usepackage{xfrac}
\usepackage{array}
\usepackage{siunitx}
\usepackage{gensymb}
\usepackage{enumitem}
\usepackage{dirtytalk}
\usepackage{bm}
\title{Functions}
\author{Zoë Sparks}

\begin{document}

\theoremstyle{definition}

\sisetup{quotient-mode=fraction}
\newtheorem{thm}{Theorem}
\newtheorem*{nthm}{Theorem}
\newtheorem{sthm}{}[thm]
\newtheorem{lemma}{Lemma}[thm]
\newtheorem*{nlemma}{Lemma}
\newtheorem{cor}{Corollary}[thm]
\newtheorem*{prop}{Property}
\newtheorem*{defn}{Definition}
\newtheorem*{comm}{Comment}
\newtheorem*{exm}{Example}

\maketitle

\begin{defn}
  A \textbf{function} is a composite object consisting of:
  \begin{enumerate}
    \item
      a set $X$, called the \textbf{domain} of the function,
    \item
      a set $Y$, called the \textbf{co-domain} of the function, and
    \item
      a rule (or correspondence) $f$, which associates each element $x$ in $X$ with a
      single element $f(x)$ in $Y$.
  \end{enumerate}

  If $(X,Y,f)$ is a function, we may say that \textbf{$f$ is a function from $X$ into
  $Y$}. This is imprecise, because $f$ is actually the rule of the function, not the
  function itself. However, it makes speaking about functions much easier if we use
  the same symbol for the rule and its function. So, if we say that $f$ is a function
  from $X$ into $Y$, that $X$ is the domain of $f$, and that $Y$ is the co-domain of
  $f$, we mean all together that $(X,Y,f)$ is a function by the above definition.

  A function $f$ can also be defined as
  \begin{enumerate}
    \item
      a set of ordered pairs of objects $(a,b)$ such that if $(a,b)$ and $(a,c)$ are
      in $f$, $b = c$, and
    \item
      a set $Y$ such that if $(a,b) \in f$, $b \in Y$.
  \end{enumerate}
  Then if $(a,b) \in f$ we say that $f(a) = b$. This is equivalent to the above
  formulation; $X$ (the domain of $f$) is then the set of all $a$ such that $(a,b)
  \in f$.

  A function may also be called a \textbf{transformation}, \textbf{operator},
  \textbf{mapping}, or another similar term when this is more expressive than simply
  saying \say{function.}
\end{defn}

\begin{defn}
  If $f$ is a function from $X$ into $Y$, the \textbf{range} (or \textbf{image}) of
  $f$ is the set of all $f(x),\ x \in X$. In other words, if $f$ is a set of ordered
  pairs $(a,b)$ as defined above, the range of $f$ is the set of all $b$ in $f$.
\end{defn}

\begin{defn}
  If the range of a function $f$ from $X$ into $Y$ is all of $Y$, $f$ is said to be
  \textbf{a function from $X$ onto $Y$}, or simply \textbf{onto}. Then the range of
  $f$ is often denoted $f(X)$. If $f$ is described as a set of ordered pairs $(a,b)$,
  then if $f$ is onto, $Y = \{(a,b) \in f:\ b\}$.
\end{defn}

\begin{exm}
  Let $X = Y = \mathbb{R}$. Let $f$ be the function from $X$ into $Y$ such that $f(x)
  = x^2$. Then the range of $f$ is the set of all non-negative real numbers, so $f$
  is not onto.
\end{exm}

\begin{exm}
  Let $X$ be the Euclidean plane, and let $Y = X$. Let $f$ be defined such that, if
  $P$ is a point in the plane, then $f(P)$ is the point obtained by rotating $P$
  $90\degree$ counterclockwise about the origin. Then the range of $f$ is all of $Y$,
  so $f$ is onto.
\end{exm}

\begin{exm}
  Let $X$ again be the Euclidean plane, and let a coordinate system be defined for
  $X$ as in analytic geometry, such that two parallel lines are used to identity
  points of $X$ as pairs of numbers $(x_1,x_2)$. Let $Y$ be the $x_1$-axis (i.e. the
  set of all points $(x_1,x_2)$ where $x_2 = 0$). Then if $P$ is a point in $X$, let
  $f(P)$ be the point given by projecting $P$ onto the $x_1$-axis via a line parallel
  to the $x_2$-axis, i.e. $f((x_1,x_2)) = (x_1,0)$. Then the range of $f$ is all of
  $Y$, so $f$ is onto.
\end{exm}

\begin{exm}
  Let $X = \mathbb{R}$, and let $Y = \{b \in \mathbb{R}:\ b > 0\}$. Let $f$ be a
  function from $X$ into $Y$ such that $f(x) = e^x$. Then $f$ is a function from $X$
  onto $Y$.
\end{exm}

\begin{exm}
  Let $X = \{a \in \mathbb{R}:\ a > 0\}$, and let $Y = \mathbb{R}$. Let $f$ be the
  natural logarithm function, i.e. $f(x) = \ln x$. Then $f$ is onto, i.e. every real
  number is the natural logarithm of some positive real number.
\end{exm}

\begin{defn}
  Let $X$, $Y$, and $Z$ be sets. Let $f$ be a function from $X$ into $Y$, and let $g$
  be a function from $Y$ into $Z$. Then there is a function associated with $f$ and
  $g$ denoted $g \circ f$ such that
  \begin{align*}
    (g \circ f)(x) = g(f(x)).
  \end{align*}
  This function is known as the \textbf{composition} of $g$ and $f$. Sometimes it's
  denoted simply by $gf$, although this notation can be confusing.
\end{defn}

\begin{exm}
  Let $X = Y = Z = \mathbb{R}$. Let $f,g,h$ be functions from $X$ into $X$ such that
  \begin{align*}
    f(x) = x^2,\ g(x) = e^x,\ h(x) = e^{x^2};
  \end{align*}
  then $h = g \circ f$.
\end{exm}

\begin{defn}
  A function $f$ is said to be \textbf{1:1} if, given that $x_1 \neq x_2$, $f(x_1)
  \neq f(x_2)$.
\end{defn}

\begin{exm}
  If $f$ is a function from $\mathbb{R}$ into $\mathbb{R}$ such that $f(x) = -x$,
  then $f$ is 1:1.
\end{exm}

\end{document}
