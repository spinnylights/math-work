\documentclass[12pt]{article}
\usepackage{mathtools}
\usepackage{amsthm}
\usepackage{amsfonts}
\usepackage{amssymb}
\usepackage{fontspec}
\usepackage{xfrac}
\usepackage{array}
\usepackage{siunitx}
\usepackage{gensymb}
\usepackage{enumitem}
\usepackage{dirtytalk}
\usepackage{bm}
\title{Functions}
\author{Zoë Sparks}

\begin{document}

\theoremstyle{definition}

\sisetup{quotient-mode=fraction}
\newtheorem{thm}{Theorem}
\newtheorem*{nthm}{Theorem}
\newtheorem{sthm}{}[thm]
\newtheorem{lemma}{Lemma}[thm]
\newtheorem*{nlemma}{Lemma}
\newtheorem{cor}{Corollary}[thm]
\newtheorem*{prop}{Property}
\newtheorem*{defn}{Definition}
\newtheorem*{comm}{Comment}
\newtheorem*{exm}{Example}

\maketitle

\begin{defn}
  A \textbf{function} is a composite object consisting of:
  \begin{enumerate}
    \item
      a set $X$, called the \textbf{domain} of the function,
    \item
      a set $Y$, called the \textbf{co-domain} of the function, and
    \item
      a rule (or correspondence) $f$, which associates each element $x$ in $X$ with a
      single element $f(x)$ in $Y$.
  \end{enumerate}

  If $(X,Y,f)$ is a function, we may say that \textbf{$f$ is a function from $X$ into
  $Y$}. This is imprecise, because $f$ is actually the rule of the function, not the
  function itself. However, it makes speaking about functions much easier if we use
  the same symbol for the rule and its function. So, if we say that $f$ is a function
  from $X$ into $Y$, that $X$ is the domain of $f$, and that $Y$ is the co-domain of
  $f$, we mean all together that $(X,Y,f)$ is a function by the above definition.

  A function $f$ can also be defined as
  \begin{enumerate}
    \item
      a set of ordered pairs of objects $(a,b)$ such that if $(a,b)$ and $(a,c)$ are
      in $f$, $b = c$, and
    \item
      a set $Y$ such that if $(a,b) \in f$, $b \in Y$.
  \end{enumerate}
  Then if $(a,b) \in f$ we say that $f(a) = b$. This is equivalent to the above
  formulation; $X$ (the domain of $f$) is then the set of all $a$ such that $(a,b)
  \in f$.

  A function may also be called a \textbf{transformation}, \textbf{operator},
  \textbf{mapping}, or another similar term when this is more expressive than simply
  saying \say{function.}
\end{defn}

\begin{defn}
  If $f$ is a function from $X$ into $Y$, the \textbf{range} of $f$ is the set of all
  $f(x),\ x \in X$. In other words, if $f$ is a set of ordered pairs $(a,b)$ as
  defined above, the range of $f$ is the set of all $b$ in $f$.

  If $E \subset X$, we call $f(E)$ the \textbf{image} of $E$ under $f$. With this
  notation, $f(X)$ represents the range of $f$. Naturally, $f(E) \subset Y$.
\end{defn}

\begin{defn}
  If the range of a function $f$ from $X$ into $Y$ is all of $Y$, i.e. if $f(X) = Y$,
  $f$ is said to be \textbf{a function from $X$ onto $Y$}, or simply \textbf{onto}.
  We can also say that $f$ \textbf{maps $X$ onto $Y$}. If $f$ is described as a set
  of ordered pairs $(a,b)$, then if $f$ is onto, $Y = \{(a,b) \in f:\ b\}$.
\end{defn}

\begin{exm}
  Let $X = Y = \mathbb{R}$. Let $f$ be the function from $X$ into $Y$ such that $f(x)
  = x^2$. Then the range of $f$ is the set of all non-negative real numbers, so $f$
  is not onto.
\end{exm}

\begin{exm}
  Let $X$ be the Euclidean plane, and let $Y = X$. Let $f$ be defined such that, if
  $P$ is a point in the plane, then $f(P)$ is the point obtained by rotating $P$
  $90\degree$ counterclockwise about the origin. Then the range of $f$ is all of $Y$,
  so $f$ is onto.
\end{exm}

\begin{exm}
  Let $X$ again be the Euclidean plane, and let a coordinate system be defined for
  $X$ as in analytic geometry, such that two parallel lines are used to identity
  points of $X$ as pairs of numbers $(x_1,x_2)$. Let $Y$ be the $x_1$-axis (i.e. the
  set of all points $(x_1,x_2)$ where $x_2 = 0$). Then if $P$ is a point in $X$, let
  $f(P)$ be the point given by projecting $P$ onto the $x_1$-axis via a line parallel
  to the $x_2$-axis, i.e. $f((x_1,x_2)) = (x_1,0)$. Then the range of $f$ is all of
  $Y$, so $f$ is onto.
\end{exm}

\begin{exm}
  Let $X = \mathbb{R}$, and let $Y = \{b \in \mathbb{R}:\ b > 0\}$. Let $f$ be a
  function from $X$ into $Y$ such that $f(x) = e^x$. Then $f$ is a function from $X$
  onto $Y$.
\end{exm}

\begin{exm}
  Let $X = \{a \in \mathbb{R}:\ a > 0\}$, and let $Y = \mathbb{R}$. Let $f$ be the
  natural logarithm function, i.e. $f(x) = \ln x$. Then $f$ is onto, i.e. every real
  number is the natural logarithm of some positive real number.
\end{exm}

\begin{defn}
  Let $X$, $Y$, and $Z$ be sets. Let $f$ be a function from $X$ into $Y$, and let $g$
  be a function from $Y$ into $Z$. Then there is a function associated with $f$ and
  $g$ denoted $g \circ f$ such that
  \begin{align*}
    (g \circ f)(x) = g(f(x)).
  \end{align*}
  This function is known as the \textbf{composition} of $g$ and $f$. Sometimes it's
  denoted simply by $gf$, although this notation can be confusing.
\end{defn}

\begin{exm}
  Let $X = Y = Z = \mathbb{R}$. Let $f,g,h$ be functions from $X$ into $X$ such that
  \begin{align*}
    f(x) = x^2,\ g(x) = e^x,\ h(x) = e^{x^2};
  \end{align*}
  then $h = g \circ f$.
\end{exm}

\begin{defn}
  A function $f$ is said to be \textbf{1:1} if, given that $x_1 \neq x_2$, $f(x_1)
  \neq f(x_2)$. If $f: X \to Y$, we then say that $f$ is a \textbf{1:1 mapping of $X$
  into $Y$}.
\end{defn}

\begin{exm}
  If $f$ is a function from $\mathbb{R}$ into $\mathbb{R}$ such that $f(x) = -x$,
  then $f$ is 1:1.
\end{exm}

\begin{defn}
  If $X$ is a set, we call the \textbf{identity function on $X$} the function $I$
  from $X$ into $X$ such that $I(x) = x$. It is common to denote this function as $I$
  regardless of the value of $X$.
\end{defn}

\begin{defn}
  Let $X$ and $Y$ be sets, and let $f$ be a function from $X$ into $Y$. We call $f$
  \textbf{invertible} if there is a function $g$ from $Y$ into $X$ such that
  \begin{enumerate}
    \item
      $g \circ f$ is the identity function on $X$, and
    \item
      $f \circ g$ is the identity function on $Y$.
  \end{enumerate}

  Naturally, such a function $g$ only exists if $f$ is 1:1. In particular, $g$ can be
  defined as follows: if an object $y \in Y$ is in the range of $f$, i.e. such that
  $f(x) = y$ for some $x$, then $g(y) = x$. (Because $f$ is 1:1, $x$ will always be
  unique by definition). If $y$ is not in the range of $f$, then $g(y)$ can be any
  element of $X$.

  If $f$ is both 1:1 and onto, then $y$ will always be in the range of $f$. In this
  case, we call $g$ the \textbf{inverse} of $f$, and conventionally denote it as
  $f^{-1}$. Then
  \begin{enumerate}
    \item
      $f^{-1}(f(x)) = x$ for each $x$ in $X$, and
    \item
      $f(f^{-1}(y)) = y$ for each $y$ in $Y$.
  \end{enumerate}

  If $E \subset Y$, $f^{-1}(E)$ is then the set of all $x \in X$ such that $f(x) \in
  E$. In this case, we say that $f^{-1}(E)$ is the \textit{inverse image} of $E$
  under $f$.
\end{defn}

\begin{exm}
  If $X = Y = \mathbb{R}$ and $f(x) = x^2$, then $f$ is not invertible, because it's
  neither 1:1 nor onto.
\end{exm}

\begin{exm}
  If $X = Y$ is the Euclidean plane, and $f$ is \say{rotation through $90\degree$,}
  then $f$ is both 1:1 and onto; $f^{-1}$ is then \say{rotation through
  $-90\degree$,} or \say{rotation through $270\degree$.}
\end{exm}

\begin{exm}
  If $X$ is the plane, $Y$ is the $x_1$-axis, and $f((x_1,x_2)) = (x_1,0)$, then $f$
  is not invertible. Although it is onto, it is not 1:1.
\end{exm}

\begin{exm}
  If $X = \mathbb{R}$, $Y = \{b \in \mathbb{R}: b > 0\}$, and $f(x) = e^x$, then $f$
  is invertible. $f^{-1}$ is then the natural logarithm function; $f^{-1}(f(x)) = \ln
  e^x = x$, and $f(f^{-1}(y)) = e^{\ln y} = y$.
\end{exm}

\begin{defn}
  Let $X,Y,X_0,Y_0$ be sets, let $f$ be a function from $X$ into $Y$, and let $f_0$
  be a function from $X_0$ into $Y_0$. We say that $f_0$ is a \textbf{restriction} of
  $f$ (or a restriction of $f$ to $X_0$) if
  \begin{enumerate}
    \item
      $X_0 \subset X$, and
    \item
      $f_0(x) = f(x)$, $x \in X_0$.
  \end{enumerate}
  If this is true, then clearly $Y_0 \subset Y$.

  Given a set $X_0 \subset X$, there is an obvious way to define $f_0$—that is, to
  say $f_0(x) = f(x)$ for each $x \in X_0$. The reason we don't call this function
  \textit{the} restriction of $f$ to $X_0$ is to give ourselves freedom in defining
  the co-domain of $f$ as well as the domain.
\end{defn}

\begin{exm}
  Let $X = \mathbb{R}$ and let $f:X \to X$ be the function such that $f(x) = x^2$;
  then $f$ is not invertible. However, let $X_0 = \{x \in \mathbb{R}: x \geq 0\}$,
  and let $f_0:X_0 \to X_0$ be such that $f_0(x) = x^2$. Then $f_0$ is a restriction
  of $f$ to $X_0$, and while $f$ is neither 1:1 nor onto, $f_0$ is both. In other
  words, each non-negative real number is the square of exactly one non-negative real
  number. $f_0^{-1}: X_0 \to X_0$ is defined by $f_0^{-1}(x) = \sqrt{x}$.
\end{exm}

\begin{exm}
  Let $f: \mathbb{R} \to \mathbb{R}$ be defined by $f(x) = x^3 + x^2 + 1$. $f$ is
  onto as its range is all of $\mathbb{R}$. However, it is not 1:1; for instance,
  $f(-1) = f(0)$. But if $X_0 = \{x \in \mathbb{R}: x \geq 0\}$ again, then $f$ is
  1:1 on $X_0$;
  \begin{align*}
    f'(x) &= \lim_{h \to 0}\frac{f(x + h) - f(x)}{h}\\
    &= \lim_{h \to 0}\frac{(x + h)^3 + (x + h)^2 + 1 - x^3 - x^2 - 1}{h}\\
    &= \lim_{h \to 0}\frac{x^3 + 3hx^2 + 3xh^2 + h^3 + x^2 + 2xh + h^2 + 1 - x^3 - x^2 - 1}{h}\\
    &= \lim_{h \to 0}\frac{(x^3 - x^3) + 3hx^2 + 3xh^2 + h^3 + (x^2 - x^2) + 2xh + h^2 + (1 - 1)}{h}\\
    &= \lim_{h \to 0}\frac{3hx^2 + 3xh^2 + h^3 + 2xh + h^2}{h}\\
    &= \lim_{h \to 0}\frac{h^3 + h^2 + 3hx^2 + 3xh^2 + 2xh}{h}\\
    &= \lim_{h \to 0}h^2 + h + 3hx^2 + 3xh + 2x\\
    &= 2x.
  \end{align*}
  Since $f'(x) = 2x > 0$ for $x > 0$, $f$ is increasing for $x > 0$, and therefore
  $f$ is 1:1 for $x \geq 0$, i.e. $x \in X_0$. As such, $f: X_0 \to X_0$ such that
  $f(x) = x^3 + x^2 + 1$ ranges over $\{y \in \mathbb{R}: y \geq 1\}$, as $f(0) = 0^3
  + 0^2 + 1 = 1$. Therefore if we let $Y_0 = \{y \in \mathbb{R}: y \geq 1\}$ and say
  that $f_0: X_0 \to Y_0$ has the same rule as $f$, $f_0$ is then both 1:1 and onto,
  and thus it has an inverse function $f_0^{-1}: Y_0 \to X_0$. The precise statement
  of $f_0^{-1}$ is hairy, however.
\end{exm}

\begin{exm}
  Let $f: \mathbb{R} \to \mathbb{R}$ be defined by $f(x) = \sin x$, i.e. let $f$ be
  the sine function. $f$ is not onto, because it ranges over $\{y: -1 \leq y \leq
  1\}$. Furthermore, since $f(x + 2\pi) = f(x)$, $f$ is not 1:1. However, if we let
  $X_0 = \{x: -\pi/2 \leq x \leq \pi/2\}$, then $f$ is 1:1 on $X_0$, and if we let
  $Y_0 = \{y: -1 \leq y \leq 1\}$ and let $f_0: X_0 \to Y_0$ be defined by $f_0(x) =
  \sin x$, then $f_0$ is both 1:1 and onto. In other words, over the interval from
  $-\pi/2$ to $\pi/2$, the sine function takes on each value from $-1$ to $1$ exactly
  once. $f_0^{-1}$ is naturally the inverse sine function:
  \begin{align*}
    f_0^{-1}(x) = \sin^{-1} x = \arcsin x.
  \end{align*}
\end{exm}

\begin{defn}
  Let $X$ be a set and let $f: X \to X$ be a function. Let $X_0$ be a subset of $X$.
  We say that $X_0$ is \textbf{invariant under $f$} if, for each $x \in X_0$, we have
  that $f(x) \in X_0$ as well. If $X_0$ is invariant under $f$, then $f$ induces a
  function $f_0: X_0 \to X_0$ which is a restriction of $f$ to $X_0$. We can thus
  make use of this to obtain a function from $X_0$ into itself, rather than simply a
  function from $X_0$ into $X$.
\end{defn}

\begin{exm}
  If $X = \{x \in \mathbb{R}: x \geq 0\}$ and $f: \mathbb{R} \to \mathbb{R}$ is a
  function defined by $f(x) = x^2$, then $f$ is invariant under $X$.
\end{exm}

\end{document}
