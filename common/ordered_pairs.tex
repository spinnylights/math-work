\documentclass[12pt]{article}
\usepackage{mathtools}
\usepackage{amsthm}
\usepackage{amsfonts}
\usepackage{amssymb}
\usepackage{fontspec}
\usepackage{xfrac}
\usepackage{array}
\usepackage{siunitx}
\usepackage{gensymb}
\usepackage{enumitem}
\usepackage{dirtytalk}
\usepackage{bm}
\title{Ordered pairs}
\author{Zoë Sparks}

\begin{document}

\theoremstyle{definition}

\sisetup{quotient-mode=fraction}
\newtheorem{thm}{Theorem}
\newtheorem*{nthm}{Theorem}
\newtheorem{sthm}{}[thm]
\newtheorem{lemma}{Lemma}[thm]
\newtheorem*{nlemma}{Lemma}
\newtheorem{cor}{Corollary}[thm]
\newtheorem*{prop}{Property}
\newtheorem*{defn}{Definition}
\newtheorem*{comm}{Comment}
\newtheorem*{exm}{Example}

\maketitle

\begin{defn}
  Consider a set $\{a,b\}$ consisting of two objects $a$ and $b$. We say that the
  \textit{ordered pair} $(a,b)$ is the set
  \begin{align*}
    \{\{a\},\{a,b\}\}.
  \end{align*}
  $a$ is then called the \textit{first element} of $(a,b)$, and $b$ is called the
  \textit{second element}.
\end{defn}

\begin{thm}
  If $(a,b) = (c,d)$, then $a = c$ and $b = d$.

  \begin{proof}
    First, consider the case in which $a = b$. Then $(a,b) = \{\{a\},\{a,b\}\} =
    \{\{a\},\{a,a\}\} = \{\{a\},\{a\}\} = \{\{a\}\}$. If $\{\{a\}\} =
    \{\{c\},\{c,d\}\}$, this implies that $\{a\} = \{c\} = \{c,d\}$. Therefore $a = c
    = d = b$.

    Now consider the case in which $a \neq b$. Then $\{a\} \neq \{a,b\}$, so
    $\{\{a\},\{a,b\}\}$ has two distinct elements. This implies that
    $\{\{c\},\{c,d\}\}$ has two distinct elements, and thus that $c \neq d$. Then
    $\{a\} \neq \{c,d\}$ because $\{a\}$ has only one element whereas
    $\{c,d\}$ has two; therefore $\{a\} = \{c\}$ is the only possibility, and thus $a
    = c$.

    Likewise, $\{a,b\} \neq \{c\}$ by the same logic, so $\{a,b\} = \{c,d\}$. Since
    we know that $a = c$, it must then be the case that $a \neq d$. Furthermore,
    since $a \neq b$, then $b \neq c$. Thus if $\{a,b\} = \{c,d\}$ and $a = c$, it
    must be the case that $b = d$.
  \end{proof}
\end{thm}

\end{document}
