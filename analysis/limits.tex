\documentclass[12pt]{article}
\usepackage{mathtools}
\usepackage{amsthm}
\usepackage{amsfonts}
\usepackage{amssymb}
\usepackage{fontspec}
\usepackage{xfrac}
\usepackage{array}
\usepackage{siunitx}
\usepackage{gensymb}
\usepackage{enumitem}
\usepackage{dirtytalk}
\usepackage{bm}
\title{Limits}
\author{Zoë Sparks}

\begin{document}

\theoremstyle{definition}

\sisetup{quotient-mode=fraction}
\newtheorem{thm}{Theorem}
\newtheorem*{nthm}{Theorem}
\newtheorem{sthm}{}[thm]
\newtheorem{lemma}{Lemma}[thm]
\newtheorem*{nlemma}{Lemma}
\newtheorem{cor}{Corollary}[thm]
\newtheorem*{prop}{Property}
\newtheorem*{defn}{Definition}
\newtheorem*{comm}{Comment}
\newtheorem*{exm}{Example}

\maketitle

\begin{defn}
  Let $X$ be a metric space with a metric $d$. A sequence $\{p_n\}$ in $X$ is said to
  \textbf{converge} if there is a point $q \in X$ such that, for every $\epsilon > 0
  \in \mathbb{R}$, there is a corresponding integer $N$ such that if $n \geq N$, then
  $d(p_n,q) < \epsilon$. If this is the case, we also say that $\{p_n\}$
  \textbf{converges to} $q$, or that $q$ is the \textbf{limit} of $\{p_n\}$.

  If $q$ is the limit of $\{p_n\}$, we write $p_n \to p$, or
  \begin{align*}
    \lim_{n \to \infty} p_n = p.
  \end{align*}

  If $\{p_n\}$ does not converge, it is said to \textbf{diverge}.
\end{defn}

\begin{defn}
  Let $X$ and $Y$ be metric spaces, and let $E \subset X$. Let $f$ be a function that
  maps $E$ into $Y$, and let $p$ be a limit point of $E$. Now suppose there is a
  point $q \in Y$ such that, for every $\epsilon > 0$, there exists a $\delta > 0$
  such that
  \begin{align*}
    d_Y(f(x),q) < \epsilon
  \end{align*}
  for all points $x \in E$ for which
  \begin{align*}
    0 < d_X(x,p) < \delta.
  \end{align*}
  If this is the case, we write $f(x) \to q$ as $x \to p$, or
  \begin{align*}
    \lim_{x \to p} f(x) = q.
  \end{align*}
  We can also say that \textbf{$f$ approaches the limit $q$ near $p$}.
\end{defn}

\begin{comm}
  In the above definition, note that $p \in X$, but $p$ is not necessarily in $E$.
  Also, even if $p \in E$, it may be the case that $f(p) \neq \lim_{x \to p} f(x)$.
\end{comm}

\begin{defn}
  A \textbf{Cauchy sequence} is a sequence $\{a_n\}$ in a metric space $X$ such that,
  for all $\epsilon > 0$, there exists $N \in \mathbb{N}$ such that $d(a_n,a_m) <
  \epsilon$ if $n,m \geq N$.
\end{defn}

\begin{comm}
  Cauchy sequences are named for Augustin-Louis Cauchy, a 19th-century French
  mathematician, physicist, and engineer well-known for his contributions to
  analysis. He sought a more rigorous set of foundations for analysis than had been
  used by his forebears, being one of the first to give stringent proofs for theorems
  of calculus. He also did the most significant early work in complex analysis, and
  was the first to define complex numbers as pairs of real numbers.
\end{comm}

\begin{defn}
  A metric space $X$ is said to be \textbf{complete} if every Cauchy sequence in $X$
  converges.
\end{defn}

\end{document}
