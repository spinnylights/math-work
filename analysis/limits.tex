\documentclass[12pt]{article}
\usepackage{mathtools}
\usepackage{amsthm}
\usepackage{amsfonts}
\usepackage{amssymb}
\usepackage{fontspec}
\usepackage{xfrac}
\usepackage{array}
\usepackage{siunitx}
\usepackage{gensymb}
\usepackage{enumitem}
\usepackage{dirtytalk}
\usepackage{bm}
\title{Limits}
\author{Zoë Sparks}

\begin{document}

\theoremstyle{definition}

\sisetup{quotient-mode=fraction}
\newtheorem{thm}{Theorem}
\newtheorem*{nthm}{Theorem}
\newtheorem{sthm}{}[thm]
\newtheorem{lemma}{Lemma}[thm]
\newtheorem*{nlemma}{Lemma}
\newtheorem{cor}{Corollary}[thm]
\newtheorem*{prop}{Property}
\newtheorem*{defn}{Definition}
\newtheorem*{comm}{Comment}
\newtheorem*{exm}{Example}

\maketitle

\begin{defn}
  Let $X$ be a metric space with a metric $d$. A sequence $\{p_n\}$ in $X$ is said to
  \textbf{converge} if there is a point $q \in X$ such that, for every $\epsilon > 0
  \in \mathbb{R}$, there is a corresponding integer $N$ such that if $n \geq N$, then
  $d(p_n,q) < \epsilon$. If this is the case, we also say that $\{p_n\}$
  \textbf{converges to} $q$, or that $q$ is the \textbf{limit} of $\{p_n\}$.

  If $q$ is the limit of $\{p_n\}$, we write $p_n \to p$, or
  \begin{align*}
    \lim_{n \to \infty} p_n = p.
  \end{align*}

  If $\{p_n\}$ does not converge, it is said to \textbf{diverge}.
\end{defn}

\begin{defn}
  Let $X$ and $Y$ be metric spaces, and let $E \subset X$. Let $f$ be a function that
  maps $E$ into $Y$, and let $p$ be a limit point of $E$. Now suppose there is a
  point $q \in Y$ such that, for every $\epsilon > 0$, there exists a $\delta > 0$
  such that
  \begin{align*}
    d_Y(f(x),q) < \epsilon
  \end{align*}
  for all points $x \in E$ for which
  \begin{align*}
    0 < d_X(x,p) < \delta.
  \end{align*}
  If this is the case, we write $f(x) \to q$ as $x \to p$, or
  \begin{align*}
    \lim_{x \to p} f(x) = q.
  \end{align*}
  We can also say that \textbf{$f$ approaches the limit $q$ near $p$}.
\end{defn}

\begin{comm}
  In the above definition, note that $p \in X$, but $p$ is not necessarily in $E$.
  Also, even if $p \in E$, it may be the case that $f(p) \neq \lim_{x \to p} f(x)$.
\end{comm}

\begin{defn}
  A \textbf{Cauchy sequence} is a sequence $\{a_n\}$ in a metric space $X$ such that,
  for all $\epsilon > 0$, there exists $N \in \mathbb{N}$ such that $d(a_n,a_m) <
  \epsilon$ if $n,m \geq N$.
\end{defn}

\begin{comm}
  Cauchy sequences are named for Augustin-Louis Cauchy, a 19th-century French
  mathematician, physicist, and engineer well-known for his contributions to
  analysis. He sought a more rigorous set of foundations for analysis than had been
  used by his forebears, being one of the first to give stringent proofs for theorems
  of calculus. He also did the most significant early work in complex analysis, and
  was the first to define complex numbers as pairs of real numbers.
\end{comm}

\begin{defn}
  A metric space $X$ is said to be \textbf{complete} if every Cauchy sequence in $X$
  converges.
\end{defn}

\begin{thm}
  $\mathbb{R}^1$ is complete.

  \begin{proof}
    Let $A = \{a_n\}$ be a Cauchy sequence in $\mathbb{R}^1$. Suppose we take
    $\epsilon = 1$; then there is some $N_1 \in \mathbb{N}$ such that $|a_n - a_m| <
    1$ when $n,m \geq N_1$. As such, $|a_n - a_{N_1}| < 1$ for $n \geq N_1$.

    Consider the sequence $a_1,a_2,\ldots,a_{N_1 - 1},a_{N_1},a_{N_1} - 1, a_{N_1} +
    1$. This sequence is finite, and thus bounded, as the real numbers are infinite;
    say all of its elements fall within the interval $[-M,M]$. Because any element
    $a_n$ of $A$ where $n \geq N_1$ is such that $|a_n - a_{N_1}| < |a_{N_1} -
    (a_{N_1} + 1)| = |a_{N_1} - (a_{N_1} - 1)| = 1$, we know that $[-M,M]$ contains
    all of $A$, so $A$ is bounded as well.

    Now consider a set $S$ of elements in $[-M,M]$. We say that a real number $s$ is
    in $S$ if there are infinitely many $n \in \mathbb{N}$ such that $a_n \geq s$.
    $-M$ is clearly in $S$ because every element of $A$ is greater than or equal to
    $-M$. Also, $S$ is bounded above by $M$. By the least-upper-bound property of
    $\mathbb{R}$, we know then that $S$ has a least upper bound $b \in \mathbb{R}$.
    Since any real number greater than $b$ is such that there are not an infinite
    number of elements of $A$ greater than it, it seems likely that $A$ converges to
    $b$.

    To prove this, we have to show that, given any $\epsilon > 0$, there is some $N
    \in \mathbb{N}$ such that for all $n \geq N$, $|a_n - b| < \epsilon$. Say we have
    some such $\epsilon > 0$. Since $A$ is a Cauchy sequence, there is some $N_2 \in
    \mathbb{N}$ such that, for $n,m \geq N_2$, $|a_n - a_m| < \frac{\epsilon}{2}$.
    Since $b = \sup S$ and $\frac{\epsilon}{2} > 0$, we know that $b +
    \frac{\epsilon}{2} \notin S$.

    This implies that there is not an infinite quantity of $a_n$ such that $a_n > b +
    \frac{\epsilon}{2}$. In other words, there is some $N_3 \geq N_2$ such that, for
    all $n \geq N_3$, $a_n \leq b + \frac{\epsilon}{2}$. Also, since $b = \sup S$,
    the smaller number $b - \frac{\epsilon}{2}$ is not an upper bound of $S$. Rather,
    there is some $s \in S$ such that $s > b - \frac{\epsilon}{2}$, which implies
    that there are infinitely many $a_n \in A$ such that $a_n \geq s > b -
    \frac{\epsilon}{2}$. For instance, there is some $N \geq N_3$ such that $a_N > b
    - \frac{\epsilon}{2}$. Since $N \geq N_3$, we have $a_N \leq b +
    \frac{\epsilon}{2}$. All together,

    \begin{align*}
      a_N \in \left( b - \frac{\epsilon}{2}, b + \frac{\epsilon}{2} \right].
    \end{align*}

    $N \geq N_2$, since $N \geq N_3$ and $N_3 \geq N_2$. Also, $b \geq a_{N_2}$ as $b
    = \sup A$. If we take $a_n \geq b - \frac{\epsilon}{2}$ (which we showed there
    are infinitely many of), we then have

    \begin{align*}
      |a_n - b| \leq |a_n - a_N| + |a_N - b| < \frac{\epsilon}{2} + \frac{\epsilon}{2} = \epsilon.
    \end{align*}

    Therefore $\lim_{n \to \infty} a_n = b$, so $\mathbb{R}^1$ converges.
  \end{proof}
\end{thm}

\begin{comm}
  This theorem can be restated to give the \textbf{Cauchy Convergence Criterion for
  sequences}: a sequence $\{a_n\}$ in $\mathbb{R}^1$ converges iff, for all $\epsilon
  > 0$, there exists $N \in \mathbb{N}$ such that, for $n,m \geq N$, we have $|a_n -
  a_m| < \epsilon$.
\end{comm}

\end{document}
