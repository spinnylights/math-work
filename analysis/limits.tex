\documentclass[12pt]{article}
\usepackage{mathtools}
\usepackage{amsthm}
\usepackage{amsfonts}
\usepackage{amssymb}
\usepackage{fontspec}
\usepackage{xfrac}
\usepackage{array}
\usepackage{siunitx}
\usepackage{gensymb}
\usepackage{enumitem}
\usepackage{dirtytalk}
\usepackage{bm}
\title{Limits}
\author{Zoë Sparks}

\begin{document}

\theoremstyle{definition}

\sisetup{quotient-mode=fraction}
\newtheorem{thm}{Theorem}
\newtheorem*{nthm}{Theorem}
\newtheorem{sthm}{}[thm]
\newtheorem{lemma}{Lemma}[thm]
\newtheorem*{nlemma}{Lemma}
\newtheorem{cor}{Corollary}[thm]
\newtheorem*{prop}{Property}
\newtheorem*{defn}{Definition}
\newtheorem*{comm}{Comment}
\newtheorem*{exm}{Example}

\maketitle

\begin{defn}
  Let $X$ be a metric space with a metric $d$. A sequence $\{p_n\}$ in $X$ is said to
  \textbf{converge} if there is a point $q \in X$ such that, for every $\epsilon > 0
  \in \mathbb{R}$, there is a corresponding integer $N$ such that if $n \geq N$, then
  $d(p_n,q) < \epsilon$. If this is the case, we also say that $\{p_n\}$
  \textbf{converges to} $q$, or that $q$ is the \textbf{limit} of $\{p_n\}$.

  If $q$ is the limit of $\{p_n\}$, we write $p_n \to p$, or
  \begin{align*}
    \lim_{n \to \infty} p_n = p.
  \end{align*}

  If $\{p_n\}$ does not converge, it is said to \textbf{diverge}.
\end{defn}

\end{document}
