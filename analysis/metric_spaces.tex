\documentclass[12pt]{article}
\usepackage{mathtools}
\usepackage{amsthm}
\usepackage{amsfonts}
\usepackage{amssymb}
\usepackage{fontspec}
\usepackage{xfrac}
\usepackage{array}
\usepackage{siunitx}
\usepackage{gensymb}
\usepackage{enumitem}
\usepackage{dirtytalk}
\usepackage{bm}
\title{Metric spaces}
\author{Zoë Sparks}

\begin{document}

\theoremstyle{definition}

\sisetup{quotient-mode=fraction}
\newtheorem{thm}{Theorem}
\newtheorem*{nthm}{Theorem}
\newtheorem{sthm}{}[thm]
\newtheorem{lemma}{Lemma}[thm]
\newtheorem*{nlemma}{Lemma}
\newtheorem{cor}{Corollary}[thm]
\newtheorem*{prop}{Property}
\newtheorem*{defn}{Definition}
\newtheorem*{comm}{Comment}
\newtheorem*{exm}{Example}

\maketitle

\begin{defn}
  Let $X$ be a set. A \textit{metric}, or \textit{distance function}, on $X$ is a
  function
  \begin{align*}
    d: X \times X \to \mathbb{R}
  \end{align*}
  with the following properties:
  \begin{enumerate}
    \item
      $d(x,y) \geq 0$ for all $x,y \in X$, and $d(x,y) = 0$ if and only if $x = y$,
    \item
      $d(x,y) = d(y,x)$ for all $x,y \in X$, and
    \item
      $d(x,y) + d(y,z) \geq d(x,z)$ for all $x,y,z \in X$.
  \end{enumerate}
  If a metric $d$ exists for $X$, then $X$ together with that metric is called a
  \textit{metric space}. The elements of $X$ are then often referred to as
  \textit{points}.
\end{defn}

\begin{exm}
  The euclidean spaces $R^k$, such as the real line $R^1$ or the complex plane $R^2$,
  are metric spaces in concert with the metric defined by
  \begin{align*}
    d(x,y) = |x - y|\ \ \ \ \ \ (x,y \in R^k).
  \end{align*}
\end{exm}

\begin{comm}
  Every subset $Y$ of a metric space $X$ is itself a metric space with the metric of
  $X$; if $p,q,r \in Y$, they are in $X$ as well, so the conditions for the metric of
  $X$ will still hold for $p,q,r$.
\end{comm}

\begin{defn}
  In the following, let $X$ be a metric space. Let $d$ be the metric of $X$, $p,q,r
  \in X$, and $N,E,M \subset X$.
  \begin{enumerate}
    \item
      A \textbf{neighborhood} of $p$ is a set $N_r(p)$ consisting of all $q$ for
      which $d(p,q) < r$ for some $r > 0 \in \mathbb{R}$. $r$ is then called the
      \textbf{radius} of $N_r(p)$.
    \item
      $p$ is a \textbf{limit point} of $E$ if \textit{every} neighborhood of $p$
      contains a point $q \neq p \in E$.
    \item
      If $p \in E$ and $p$ is not a limit point of $E$, $p$ is called an
      \textbf{isolated point} of $E$.
    \item
      $E$ is \textbf{closed} if every limit point of $E$ is contained by $E$.
    \item
      $p$ is an \textbf{interior} point of $E$ if there is a neighborhood $N$ of $p$
      such that $N \subset E$.
    \item
      $E$ is \textbf{open} if every point of $E$ is an interior point of $E$.
    \item
      The \textbf{complement} of $E$ (denoted by $E^c$) is the set of all points $p
      \in X$ such that $p \notin E$.
    \item
      $E$ is \textbf{perfect} if $E$ is closed and if every point of $E$ is a limit
      point of $E$.
    \item
      $E$ is \textbf{bounded} if there is a real number $M$ and a point $q \in X$
      such that $d(p,q) < M$ for all $p \in E$.
    \item
      $E$ is \textbf{dense in $X$} if every point of $X$ is a limit point of $E$, a
      point of $E$, or both.
  \end{enumerate}
\end{defn}

\end{document}
