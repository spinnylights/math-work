\documentclass[12pt]{article}
\usepackage{mathtools}
\usepackage{amsthm}
\usepackage{amsfonts}
\usepackage{amssymb}
\usepackage{fontspec}
\usepackage{xfrac}
\usepackage{array}
\usepackage{siunitx}
\usepackage{gensymb}
\usepackage{enumitem}
\usepackage{dirtytalk}
\usepackage{bm}
\title{Subspaces (exercises)}
\author{Zoë Sparks}

\begin{document}

\theoremstyle{definition}

\sisetup{quotient-mode=fraction}
\newtheorem{thm}{Theorem}
\newtheorem*{nthm}{Theorem}
\newtheorem{sthm}{}[thm]
\newtheorem{lemma}{Lemma}[thm]
\newtheorem*{nlemma}{Lemma}
\newtheorem{cor}{Corollary}[thm]
\newtheorem*{prop}{Property}
\newtheorem*{defn}{Definition}
\newtheorem*{comm}{Comment}
\newtheorem*{exm}{Example}

\maketitle

\begin{enumerate}
  \item
    We would like to determine which of the following sets of vectors $\alpha =
    (a_1,\ldots,a_n)$ in $R^n$ are subspaces of $R^n (n \geq 3)$.
    \begin{enumerate}
      \item
        The set of all $\alpha$ such that $a_1 \geq 0$ is not, because if $a_1 > 0$
        and $c < 0$, then $ca_1 < 0$, so it's possible for $c\alpha =
        (ca_1,\ldots,ca_n)$ to be outside the set.

      \item
        The set of all $\alpha$ such that $a_1 + 3a_2 = a_3$ is:
        \begin{align*}
          c(a_1,a_2,a_1 + 3a_2,\ldots,a_n) + (b_1,b_2,b_1 + 3b_2,\ldots,b_n) &=\\
          (ca_1,ca_2,ca_1 + 3ca_2,\ldots,ca_n) + (b_1,b_2,b_1 + 3b_2,\ldots,b_n) &=\\
          (ca_1+b_1,\ ca_2+b_2,\ ca_1+b_1 + 3ca_2+3b_2,\ \ldots,\ ca_n + b_n) &=\\
          (ca_1+b_1,\ ca_2+b_2,\ (ca_1+b_1) + 3(ca_2+b_2),\ \ldots,\ ca_n + b_n).
        \end{align*}

      \item
        The set of all $\alpha$ such that $a_2 = a_1^2$ is not:
        \begin{align*}
          c(a_1,a_1^2,\ldots,a_n) + (b_1,b_1^2,\ldots,b_n) &=\\
          (ca_1,ca_1^2,\ldots,ca_n) + (b_1,b_1^2,\ldots,b_n) &=\\
          (ca_1+b_1,ca_1^2+b_1^2,\ldots,ca_n+b_n).
        \end{align*}
        $(ca_1+b_1)^2 = ca_1^2 + 2ca_1b_1 + b_1^2 \neq ca_1^2+b_1^2$ (well, unless
        $a,b = 0$).

      \item
        For the set of all $\alpha$ such that $a_1a_2 = 0$, we know by the definition
        of a field that for any given vector in the set, either $a_1 = 0$, $a_2 = 0$,
        or both. If one has $a_1 = 0,\ a_2 \neq 0$ and the other has $a_2 = 0,\ a_1
        \neq 0$, then
        \begin{align*}
          c(a_1,0,\ldots,a_n) + (0,b_2,\ldots,b_n) &=\\
          (ca_1,c0,\ldots,ca_n) + (0,b_2,\ldots,b_n) &=\\
          (ca_1,0,\ldots,ca_n) + (0,b_2,\ldots,b_n) &=\\
          (ca_1,b_2,\ldots,ca_n+b_n).
        \end{align*}
        So, this set is not.

      \item
        The set of all $\alpha$ such that $a_2$ is rational is not, because $R^n$ is
        over the real numbers, and some real numbers are irrational. If $c$ is an
        irrational real number, $ca_2$ will also be irrational unless $a_2 = 0$, so
        $c\alpha$ will be outside the set for some values of $c$.
    \end{enumerate}

    \item
      If $V$ is the real vector space of all functions $f$ from $R$ into $R$, we
      would like to determine which of the following sets are subspaces of $V$. In
      the following, let $f,g$ be members of the set in question and let $c$ be a
      real number.
      \begin{enumerate}
        \item
          The set of all $f$ such that $f(x^2) = f(x)^2$ is not. If
          \begin{align*}
            h(x) &= cf(x) + g(x),\\
            \shortintertext{then}
            h(x^2) &= cf(x^2) + g(x^2)\\
            &= cf(x)^2 + g(x)^2,\\
            \shortintertext{but}
            h(x)^2 = (cf(x) + g(x))^2 &= c^2f(x)^2 + 2cf(x)g(x) + g(x)^2.
          \end{align*}

        \item
          The set of all $f$ such that $f(0) = f(1)$ is. If
          \begin{align*}
            h(x) &= cf(x) + g(x),\\
            \shortintertext{then}
            h(0) &= cf(0) + g(0)\\
            &= cf(1) + g(1)\\
            \shortintertext{and}
            h(1) &= cf(1) + g(1).
          \end{align*}

        \item
          The set of all $f$ such that $f(3) = 1 + f(-5)$ is not. If
          \begin{align*}
            h(x) &= cf(x) + g(x),\\
            \shortintertext{then}
            h(3) &= cf(3) + g(3)\\
            &= 1 + cf(-5) + 1 + g(-5)\\
            &= 2 + cf(-5) + g(-5),
            \shortintertext{but}
            h(-5) &= cf(-5) + g(-5).
          \end{align*}

        \item
          The set of all $f$ such that $f(-1) = 0$ is. If
          \begin{align*}
            h(x) &= cf(x) + g(x),\\
            \shortintertext{then}
            h(-1) &= cf(-1) + g(-1)\\
            &= c0 + 0\\
            &= 0.
          \end{align*}

        \item
          The set of all continuous $f$ is, because if $f,g$ are continuous, $cf$ is
          continuous and $f + g$ is continuous, so $cf + g$ is continuous.
      \end{enumerate}

    \item
      We would like to determine if the vector $(3,-1,0,-1)$ is in the subspace of
      $R^5$ spanned by the vectors $(2,-1,3,2)$, $(-1,1,1,-3)$, and $(1,1,9,-5)$.

      The subspace of $R^5$ spanned by the vectors
      \begin{alignat*}{7}
        \alpha_1 &= (&2&,\ &-1&,\ &3&,\ &2&)\\
        \alpha_2 &= (&-1&,\ &1&,\ &1&,\ &-3&)\\
        \alpha_3 &= (&1&,\ &1&,\ &9&,\ &-5&)
      \end{alignat*}
      is the row space of the matrix
      \begin{align*}
        \begin{bmatrix}
          2  & -1 & 3 & 2  & 0\\
          -1 & 1  & 1 & -3 & 0\\
          1  & 1  & 9 & -5 & 0\\
        \end{bmatrix}.
      \end{align*}
      \begin{align*}
        \begin{bmatrix}
          2 & -1 & 3 & 2 & 0\\
          -1 & 1 & 1 & -3 & 0\\
          1 & 1 & 9 & -5 & 0
        \end{bmatrix}
        \xrightarrow{}
        \begin{bmatrix}
          1 & -\frac{1}{2} & \frac{3}{2} & 1 & 0\\
          -1 & 1 & 1 & -3 & 0\\
          1 & 1 & 9 & -5 & 0
        \end{bmatrix}
        \xrightarrow{}
      \end{align*}
      \begin{align*}
        \begin{bmatrix}
          1 & -\frac{1}{2} & \frac{3}{2} & 1 & 0\\
          0 & \frac{1}{2} & \frac{5}{2} & -2 & 0\\
          1 & 1 & 9 & -5 & 0
        \end{bmatrix}
        \xrightarrow{}
        \begin{bmatrix}
          1 & -\frac{1}{2} & \frac{3}{2} & 1 & 0\\
          0 & \frac{1}{2} & \frac{5}{2} & -2 & 0\\
          0 & \frac{3}{2} & \frac{15}{2} & -6 & 0
        \end{bmatrix}
        \xrightarrow{}
      \end{align*}
      \begin{align*}
        \begin{bmatrix}
          1 & -\frac{1}{2} & \frac{3}{2} & 1 & 0\\
          0 & 1 & 5 & -4 & 0\\
          0 & \frac{3}{2} & \frac{15}{2} & -6 & 0
        \end{bmatrix}
        \xrightarrow{}
        \begin{bmatrix}
          1 & 0 & 4 & -1 & 0\\
          0 & 1 & 5 & -4 & 0\\
          0 & \frac{3}{2} & \frac{15}{2} & -6 & 0
        \end{bmatrix}
        \xrightarrow{}
      \end{align*}
      \begin{align*}
        \begin{bmatrix}
          1 & 0 & 4 & -1 & 0\\
          0 & 1 & 5 & -4 & 0\\
          0 & 0 & 0 & 0 & 0
        \end{bmatrix}.
      \end{align*}

      \begin{align*}
        \alpha = (c_1,c_2,4c_1+5c_2,-c_1-4c_2,0).
      \end{align*}

      \begin{align*}
        [2,\ -1,\ 4(2) + 5(-1), -2 -4(-1)] &=\\
        (2,\ -1,\ 8 - 5, -2 +4) &=\\
        (2,\ -1,\ 3, 2).\\\\
        [-1,\ 1,\ 4(-1) + 5(1),-(-1) - 4(1)] &=\\
        (-1,\ 1,\ -4 + 5,\ 1 - 4) &=\\
        (-1,\ 1,\ 1,\ -3).\\\\
        [1,\ 1,\ 4(1) + 5(1),\ -(1) - 4(1)] &=\\
        (1,\ 1,\ 4 + 5,\ -1 - 4) &=\\
        (1,\ 1,\ 9,\ -5).\\\\
        [3,\ -1,\ 4(3) + 5(-1),\ -(3) -4(-1)] &=\\
        (3,\ -1,\ 12 - 5,\ -3 +4) &=\\
        (3,\ -1,\ 7,\ 1).
      \end{align*}

      So, $(3,-1,0,-1)$ is not in the subspace, although $(3,-1,7,1)$ is.

    \item
      If $W$ is the set of all $(x_1,x_2,x_3,x_4,x_5)$ in $R^5$ which satisfy
      \begin{alignat*}{12}
        2x_1&\ -\ &  x_2\ & +\ & \frac{4}{3}x_3\ & -\ &  x_4\ &  \ &     \ & =\ & 0&\\
         x_1&   \ &     \ & +\ & \frac{2}{3}x_3\ &  \ &     \ & -\ &  x_5\ & =\ & 0&\\
        9x_1&\ -\ & 3x_2\ & +\ &           6x_3\ & -\ & 3x_4\ & -\ & 3x_5\ & =\ & 0&,
      \end{alignat*}
      we would like to find a finite set of vectors $S$ which spans $W$.

      In other words, we would like to find a finite set of vectors $S$ such that $W$ is
      the intersection of all the subspaces of $R^5$ that contain $S$. Another way of
      putting this is that $S$ should be such that $W$ is the smallest subspace of
      $R^5$ which contains $S$.

      Further, we know by theorem 3 that $S$ should be such that $W$ is the set of
      all linear combinations of $S$. Let $A$ be the matrix of coefficients of the
      system which defines $W$, i.e.
      \begin{align*}
        A =
        \begin{bmatrix}
          2 & -1 & \frac{4}{3} &  -1 &  0 \\
          1 &  0 & \frac{2}{3} &   0 & -1 \\
          9 & -3 &           6 &  -3 & -3.
        \end{bmatrix}
      \end{align*}
      By theorem 2 of \say{Matrices,} we know that if a matrix $B$ is row-equivalent to
      $A$, then $AX = 0$ and $BX = 0$ have the same solutions, i.e. they define the
      same set of vectors in $R^5$. Since $A$ is row-equivalent to a row-reduced
      echelon matrix by theorem 5 of \say{Matrices,} we can find $S$ more easily by
      row-reducing $A$.

      \begin{align*}
        \begin{bmatrix}
          2 & -1 & \frac{4}{3} & -1 & 0\\
          1 & 0 & \frac{2}{3} & 0 & -1\\
          9 & -3 & 6 & -3 & -3
        \end{bmatrix}
        \xrightarrow{}
        \begin{bmatrix}
          1 & 0 & \frac{2}{3} & 0 & -1\\
          2 & -1 & \frac{4}{3} & -1 & 0\\
          9 & -3 & 6 & -3 & -3
        \end{bmatrix}
      \end{align*}
      \begin{align*}
        \begin{bmatrix}
          1 & 0 & \frac{2}{3} & 0 & -1\\
          0 & -1 & 0 & -1 & 2\\
          9 & -3 & 6 & -3 & -3
        \end{bmatrix}
        \xrightarrow{}
        \begin{bmatrix}
          1 & 0 & \frac{2}{3} & 0 & -1\\
          0 & -1 & 0 & -1 & 2\\
          0 & -3 & 0 & -3 & 6
        \end{bmatrix}
      \end{align*}
      \begin{align*}
        \begin{bmatrix}
          1 & 0 & \frac{2}{3} & 0 & -1\\
          0 & 1 & 0 & 1 & -2\\
          0 & -3 & 0 & -3 & 6
        \end{bmatrix}
        \xrightarrow{}
        \begin{bmatrix}
          1 & 0 & \frac{2}{3} & 0 & -1\\
          0 & 1 & 0 & 1 & -2\\
          0 & 0 & 0 & 0 & 0
        \end{bmatrix}.
      \end{align*}

      So, $W$ is also the set of all $(x_1,x_2,x_3,x_4,x_5)$ in $R^5$ which satisfy
      \begin{alignat*}{12}
         x_1&   \ &     \ & +\ & \frac{2}{3}x_3\ &  \ &     \ & -\ &  x_5\ & =\ & 0&\\
            &   \ &  x_2\ &  \ &               \ & +\ &  x_4\ & -\ & 2x_5\ & =\ & 0&,
      \end{alignat*}
      or in other words,
      \begin{align*}
        x_1 =& -\frac{2}{3}x_3 + x_5\\
        x_2 =& -x_4 + 2x_5.
      \end{align*}
      Even more concisely, $W$ is the set of all vectors in $R^5$ with the structure
      $(-\frac{2}{3}x_3 + x_5,\ -x_4 + 2x_5,\ x_3,\ x_4,\ x_5)$. Then a finite set of
      vectors which spans $W$ is
      \begin{alignat*}{9}
        \alpha_1 =&\ (&\ -\frac{2}{3},&\ \ \ \ \ 0,&\ 1,&\ 0,&\ 0&\ )&,\\
        \alpha_2 =&\ (&\            0,&\ -1,&\ 0,&\ 1,&\ 0&\ )&,\\
        \alpha_3 =&\ (&\            1,&\ \ \ \ \ 2,&\ 0,&\ 0,&\ 1&\ )&.
      \end{alignat*}

      For proof, let $S$ be this set, i.e. $S = \{ \alpha_1, \alpha_2, \alpha_3 \}$.
      Let $C$ be a linear combination of $S$; then $C = c_1\alpha_1 + c_2\alpha_2 +
      c_3\alpha_3 = (-\frac{2}{3}c_1 + c_3,\ -c_2 + 2c_3,\ c_1,\ c_2,\ c_3)$. Thus
      the set of all $C$ is $W$, so $S$ spans $W$.

      As an addendum,
      \begin{align*}
        2x_1 - x_2 + \frac{4}{3}x_3 - x_4 =&\\
        2(-\frac{2}{3}c_1 + c_3) - (-c_2 + 2c_3) + \frac{4}{3}c_1 - c_2 =&\\
        -\frac{4}{3}c_1 + 2c_3 + c_2 - 2c_3 + \frac{4}{3}c_1 - c_2 =&\\
        (\frac{4}{3}c_1 - \frac{4}{3}c_1) + (2c_3 - 2c_3) + (c_2 - c_2) =&\\
        0.\ \ \ &\\\\
        x_1 + \frac{2}{3}x_3 - x_5 =&\\
        (-\frac{2}{3}c_1 + c_3) + \frac{2}{3}c_1 - c_3 =&\\
        (\frac{2}{3}c_1 - \frac{2}{3}c_1) + (c_3 - c_3) =&\\
        0.\ \ \ &\\\\
        9x_1 - 3x_2 + 6x_3 - 3x_4 - 3x_5 =&\\
        9(-\frac{2}{3}c_1 + c_3) - 3(-c_2 + 2c_3) + 6c_1 - 3c_2 - 3c_3 =&\\
        -6c_1 + 9c_3 + 3c_2 - 6c_3 + 6c_1 - 3c_2 - 3c_3 =&\\
        (6c_1 - 6c_1) + (3c_2 - 3c_2) + (9c_3 - 6c_3 - 3c_3) =&\\
        (6c_1 - 6c_1) + (3c_2 - 3c_2) + (9c_3 - 9c_3) =&\\
        0.\ \ \ &
      \end{align*}

    \item
      If $F$ is a field, $n$ is a positive integer $(n \geq 2)$, and $V$ is then the
      vector space of all $n \times n$ matrices over $F$, we would like to determine
      which of the following sets of matrices $A$ in $V$ are subspaces of $V$.
      \begin{enumerate}
        \item
          \textbf{All invertible $A$.} The set of all invertible $A$ is not a
          subspace of $V$, because the sum of two invertible matrices is not
          necessarily invertible. We proceed by contradiction. Assume that, for any
          invertible $n \times n$ matrices $A$ and $B$, $A + B$ is invertible. By
          theorem 12, an invertible $n \times n$ matrix is row-equivalent to the $n
          \times n$ identity matrix. Consider $I$ and $-I$; $I$ is plainly
          row-equivalent to itself, and $-I$ is row-equivalent to $I$ by repeated
          application of elementary row operation 1 to each of its rows using the
          scalar $-1$. Therefore, both $I$ and $-I$ are invertible. Another way of
          putting this is that $II = I$, and $-I(-I) = I$, so both $I$ and $-I$ are
          their own inverses respectively.

          However, $I - I = 0$. There is no matrix $N$ in $V$ such that $N0 = I$,
          because $N0 = 0$ for all $N$. Therefore $I - I = 0$ is not invertible,
          which contradicts our initial assumption. So, even if both $cA$ and $B$ are
          invertible, the sum $cA + B$ may not be, i.e. in the case where $c = -1$
          and $A,B = I$. Thus the set of all invertible $A$ in $V$ is not a subspace
          of $V$.

        \item
          \textbf{All non-invertible $A$.} The set of all non-invertible $A$ is not a
          subspace of $V$ either, for similar reasons to the set of invertible $A$.
          Again, we proceed by contradiction. Assume that for any non-invertible
          $A,B$ in $V$, $A + B$ is non-invertible. Take the case where $n = 2$, and
          let
          \begin{align*}
            A =
            \begin{bmatrix}
              1 & 0\\
              0 & 0
            \end{bmatrix},\
            B =
            \begin{bmatrix}
              0 & 0\\
              0 & 1
            \end{bmatrix}.
          \end{align*}
          Neither $A$ nor $B$ is invertible, as there is no series of elementary row
          operations that will map $A_{22}$ or $B_{11}$ to $1$. Multiplication of row
          $A_{2}$ or $B_{1}$ by a non-zero scalar will have no effect as those rows
          are all-zero, and adding row $A_{1}$ or $B_{2}$ multiplied by a non-zero
          scalar to their respective other rows will not change the value of $A_{22}$
          or $B_{11}$, nor will swapping rows in either.

          However, $A + B = I$, and $I$ is invertible as we touched on in the first
          problem. This condradicts our initial assumption, and thus even if $cA$ and
          $B$ are non-invertible, $cA + B$ may be, i.e. if $A,B$ are as defined above
          and $c = 1$. Therefore the set of all non-invertible matrices in $V$ is not
          a subspace of $V$.

        \item
          \textbf{All $A$ such that $AB = BA$, where $B$ is some fixed matrix in
          $V$.} Let $A,C$ be matrices in $V$ such that $AB = BA$ and $CB = BC$ for
          some fixed matrix $B$ in $V$. What we would like to determine is whether
          $(cA + C)B = B(cA + C)$ for any scalar $c$ in $F$. By our lemma in
          \say{Vector spaces,} $(cA + C)B = cAB + CB$ and $B(cA + C) = cBA + BC$. If
          $AB = BA$, then $cAB = cBA$, and since $CB = BC$, $cAB + CB = cBA + CB =
          cAB + BC = cBA + BC$. Therefore $(cA + C)B = B(cA + C)$, so the set of all
          $A$ in $V$ such that $AB = BA$ for some fixed matrix $B$ in $V$ is a subset
          of $V$.

        \item
          \textbf{All $A$ such that $A^2 = A$.} If $A,B$ are matrices in $V$ such
          that $A^2 = A$ and $B^2 = B$, what we would like to determine is whether
          $cA + B = (cA + B)^2$ for any scalar $c$ in $F$.
          \begin{align*}
            [(cA + B)^2]_{ij}
            &= \sum_{r = 1}^{n}(cA + B)_{ir}(cA + B)_{rj}\\
            &= \sum_{r = 1}^{n}(cA_{ir} + B_{ir})(cA_{rj} + B_{rj})\\
            &= \sum_{r = 1}^{n}(c^2A_{ir}A_{rj}
              + cB_{ir}A_{rj} + cA_{ir}B_{rj} + B_{ir}B_{rj})\\
            &= \sum_{r = 1}^{n}c^2A_{ir}A_{rj} +
              \sum_{r = 1}^{n} cB_{ir}A_{rj} +
              \sum_{r = 1}^{n} cA_{ir}B_{rj} +
              \sum_{r = 1}^{n} B_{ir}B_{rj}\\
            &= c^2\sum_{r = 1}^{n}A_{ir}A_{rj} +
              c\sum_{r = 1}^{n} B_{ir}A_{rj} +
              c\sum_{r = 1}^{n} A_{ir}B_{rj} +
              \sum_{r = 1}^{n} B_{ir}B_{rj}\\
            &= (c^2A^2 + cBA + cAB + B^2)_{ij}.
          \end{align*}
          Since $A^2 = A$ and $B^2 = B$ in this case, $(cA + B)^2 = c^2A + cBA + cAB
          + B$. This is not necessarily equal to $cA + B$. For instance, $I^2 = I$,
          so if $A,B = I$ and $c = 2$, $(cA + B)^2 = 4I + 2I + 2I + I = 9I$. However
          $2I + I = 3I$, so $cA + B \neq (cA + B)^2$ in this case. Thus the set of
          all $A$ in $V$ such that $A^2 = A$ is not a subspace of $V$.
        \end{enumerate}

        \item
          \begin{enumerate}
            \item
              We would like to prove that the only subspaces of $R^1$ are $R^1$ and
              the zero subspace. We know from the definition of a subspace that $R^1$
              and the zero subspace are subspaces of $R^1$. Now, since $R^1$ is over
              $\mathbb{R}$, if $c$ is any real number and $A \neq 0$ is a vector in
              $R^1$, $cA$ can be made to take on any value in $R^1$, because if $A =
              (\alpha_1),\ \alpha_1 \in \mathbb{R}$, $cA = c\alpha_1$. Therefore
              aside from the zero subspace there is no way to define a subset of
              $R^1$ in which, for $A,B$ in the subset, $cA + B$ cannot be made to
              take on any value in $R^1$ if $A \neq 0$.
            \item
              We would like to prove that a subspace of $R^2$ is either $R^2$, the
              zero subspace, or a space of all scalar muiltiples of some fixed vector
              in $R^2$ (a straight line through the origin in geometric terms). We
              have that $R^2$ and the zero subspace are subspaces of $R^2$ by
              definition. Otherwise, let $A = (\alpha_1,\alpha_2)$ be a vector in
              $R^2$. A scalar multiple of $A$ would be $cA = (c\alpha_1,c\alpha_2),\
              c \in \mathbb{R}$.  If $B = (d\alpha_1,d\alpha_2),\ B \in R^2,\ d \in
              \mathbb{R}$, then $cA + B = (c\alpha_1+d\alpha_1,\ c\alpha_2+d\alpha_2)
              = [(c+d)\alpha_1,\ (c+d)\alpha_2]$. Therefore the space of all scalar
              multiples of some fixed vector $A$ in $R^2$ is indeed a subspace of
              $R^2$. Otherwise, let $C = (\gamma_1,\gamma_2),\ C \in R^2,\ C \neq cA$
              for any $c \neq 0,\ c \in \mathbb{R}$. Then $cA + C$ can be any vector
              in $R^2$ by the choice of $C$, including multiplies of $A$; $C = (0,0)$
              must be permitted or $cA + C$ could not define a subspace by
              definition. Therefore the only subspaces in $R^2$ are those described.
          \end{enumerate}

  \end{enumerate}
\end{document}
