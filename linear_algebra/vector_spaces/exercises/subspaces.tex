\documentclass[12pt]{article}
\usepackage{mathtools}
\usepackage{amsthm}
\usepackage{amsfonts}
\usepackage{amssymb}
\usepackage{fontspec}
\usepackage{xfrac}
\usepackage{array}
\usepackage{siunitx}
\usepackage{gensymb}
\usepackage{enumitem}
\usepackage{dirtytalk}
\usepackage{bm}
\title{Subspaces (exercises)}
\author{Zoë Sparks}

\begin{document}

\theoremstyle{definition}

\sisetup{quotient-mode=fraction}
\newtheorem{thm}{Theorem}
\newtheorem*{nthm}{Theorem}
\newtheorem{sthm}{}[thm]
\newtheorem{lemma}{Lemma}[thm]
\newtheorem*{nlemma}{Lemma}
\newtheorem{cor}{Corollary}[thm]
\newtheorem*{prop}{Property}
\newtheorem*{defn}{Definition}
\newtheorem*{comm}{Comment}
\newtheorem*{exm}{Example}

\maketitle

\begin{enumerate}
  \item
    We would like to determine which of the following sets of vectors $\alpha =
    (a_1,\ldots,a_n)$ in $R^n$ are subspaces of $R^n (n \geq 3)$.
    \begin{enumerate}
      \item
        The set of all $\alpha$ such that $a_1 \geq 0$ is not, because if $a_1 > 0$
        and $c < 0$, then $ca_1 < 0$, so it's possible for $c\alpha =
        (ca_1,\ldots,ca_n)$ to be outside the set.

      \item
        The set of all $\alpha$ such that $a_1 + 3a_2 = a_3$ is:
        \begin{align*}
          c(a_1,a_2,a_1 + 3a_2,\ldots,a_n) + (b_1,b_2,b_1 + 3b_2,\ldots,b_n) &=\\
          (ca_1,ca_2,ca_1 + 3ca_2,\ldots,ca_n) + (b_1,b_2,b_1 + 3b_2,\ldots,b_n) &=\\
          (ca_1+b_1,\ ca_2+b_2,\ ca_1+b_1 + 3ca_2+3b_2,\ \ldots,\ ca_n + b_n) &=\\
          (ca_1+b_1,\ ca_2+b_2,\ (ca_1+b_1) + 3(ca_2+b_2),\ \ldots,\ ca_n + b_n).
        \end{align*}

      \item
        The set of all $\alpha$ such that $a_2 = a_1^2$ is not:
        \begin{align*}
          c(a_1,a_1^2,\ldots,a_n) + (b_1,b_1^2,\ldots,b_n) &=\\
          (ca_1,ca_1^2,\ldots,ca_n) + (b_1,b_1^2,\ldots,b_n) &=\\
          (ca_1+b_1,ca_1^2+b_1^2,\ldots,ca_n+b_n).
        \end{align*}
        $(ca_1+b_1)^2 = ca_1^2 + 2ca_1b_1 + b_1^2 \neq ca_1^2+b_1^2$ (well, unless
        $a,b = 0$).

      \item
        For the set of all $\alpha$ such that $a_1a_2 = 0$, we know by the definition
        of a field that for any given vector in the set, either $a_1 = 0$, $a_2 = 0$,
        or both. If one has $a_1 = 0,\ a_2 \neq 0$ and the other has $a_2 = 0,\ a_1
        \neq 0$, then
        \begin{align*}
          c(a_1,0,\ldots,a_n) + (0,b_2,\ldots,b_n) &=\\
          (ca_1,c0,\ldots,ca_n) + (0,b_2,\ldots,b_n) &=\\
          (ca_1,0,\ldots,ca_n) + (0,b_2,\ldots,b_n) &=\\
          (ca_1,b_2,\ldots,ca_n+b_n).
        \end{align*}
        So, this set is not.

      \item
        The set of all $\alpha$ such that $a_2$ is rational is not, because $R^n$ is
        over the real numbers, and some real numbers are irrational. If $c$ is an
        irrational real number, $ca_2$ will also be irrational unless $a_2 = 0$, so
        $c\alpha$ will be outside the set for some values of $c$.
    \end{enumerate}

    \item
      If $V$ is the real vector space of all functions $f$ from $R$ into $R$, we
      would like to determine which of the following sets are subspaces of $V$. In
      the following, let $f,g$ be members of the set in question and let $c$ be a
      real number.
      \begin{enumerate}
        \item
          The set of all $f$ such that $f(x^2) = f(x)^2$ is not. If
          \begin{align*}
            h(x) &= cf(x) + g(x),\\
            \shortintertext{then}
            h(x^2) &= cf(x^2) + g(x^2)\\
            &= cf(x)^2 + g(x)^2,\\
            \shortintertext{but}
            h(x)^2 = (cf(x) + g(x))^2 &= c^2f(x)^2 + 2cf(x)g(x) + g(x)^2.
          \end{align*}

        \item
          The set of all $f$ such that $f(0) = f(1)$ is. If
          \begin{align*}
            h(x) &= cf(x) + g(x),\\
            \shortintertext{then}
            h(0) &= cf(0) + g(0)\\
            &= cf(1) + g(1)\\
            \shortintertext{and}
            h(1) &= cf(1) + g(1).
          \end{align*}

        \item
          The set of all $f$ such that $f(3) = 1 + f(-5)$ is not. If
          \begin{align*}
            h(x) &= cf(x) + g(x),\\
            \shortintertext{then}
            h(3) &= cf(3) + g(3)\\
            &= 1 + cf(-5) + 1 + g(-5)\\
            &= 2 + cf(-5) + g(-5),
            \shortintertext{but}
            h(-5) &= cf(-5) + g(-5).
          \end{align*}

        \item
          The set of all $f$ such that $f(-1) = 0$ is. If
          \begin{align*}
            h(x) &= cf(x) + g(x),\\
            \shortintertext{then}
            h(-1) &= cf(-1) + g(-1)\\
            &= c0 + 0\\
            &= 0.
          \end{align*}

        \item
          The set of all continuous $f$ is, because if $f,g$ are continuous, $cf$ is
          continuous and $f + g$ is continuous, so $cf + g$ is continuous.
      \end{enumerate}

    \item
      We would like to determine if the vector $(3,-1,0,-1)$ is in the subspace of
      $R^5$ spanned by the vectors $(2,-1,3,2)$, $(-1,1,1,-3)$, and $(1,1,9,-5)$.

      The subspace of $R^5$ spanned by the vectors
      \begin{alignat*}{7}
        \alpha_1 &= (&2&,\ &-1&,\ &3&,\ &2&)\\
        \alpha_2 &= (&-1&,\ &1&,\ &1&,\ &-3&)\\
        \alpha_3 &= (&1&,\ &1&,\ &9&,\ &-5&)
      \end{alignat*}
      is the row space of the matrix
      \begin{align*}
        \begin{bmatrix}
          2  & -1 & 3 & 2  & 0\\
          -1 & 1  & 1 & -3 & 0\\
          1  & 1  & 9 & -5 & 0\\
        \end{bmatrix}.
      \end{align*}
      \begin{align*}
        \begin{bmatrix}
          2 & -1 & 3 & 2 & 0\\
          -1 & 1 & 1 & -3 & 0\\
          1 & 1 & 9 & -5 & 0
        \end{bmatrix}
        \xrightarrow{}
        \begin{bmatrix}
          1 & -\frac{1}{2} & \frac{3}{2} & 1 & 0\\
          -1 & 1 & 1 & -3 & 0\\
          1 & 1 & 9 & -5 & 0
        \end{bmatrix}
        \xrightarrow{}
      \end{align*}
      \begin{align*}
        \begin{bmatrix}
          1 & -\frac{1}{2} & \frac{3}{2} & 1 & 0\\
          0 & \frac{1}{2} & \frac{5}{2} & -2 & 0\\
          1 & 1 & 9 & -5 & 0
        \end{bmatrix}
        \xrightarrow{}
        \begin{bmatrix}
          1 & -\frac{1}{2} & \frac{3}{2} & 1 & 0\\
          0 & \frac{1}{2} & \frac{5}{2} & -2 & 0\\
          0 & \frac{3}{2} & \frac{15}{2} & -6 & 0
        \end{bmatrix}
        \xrightarrow{}
      \end{align*}
      \begin{align*}
        \begin{bmatrix}
          1 & -\frac{1}{2} & \frac{3}{2} & 1 & 0\\
          0 & 1 & 5 & -4 & 0\\
          0 & \frac{3}{2} & \frac{15}{2} & -6 & 0
        \end{bmatrix}
        \xrightarrow{}
        \begin{bmatrix}
          1 & 0 & 4 & -1 & 0\\
          0 & 1 & 5 & -4 & 0\\
          0 & \frac{3}{2} & \frac{15}{2} & -6 & 0
        \end{bmatrix}
        \xrightarrow{}
      \end{align*}
      \begin{align*}
        \begin{bmatrix}
          1 & 0 & 4 & -1 & 0\\
          0 & 1 & 5 & -4 & 0\\
          0 & 0 & 0 & 0 & 0
        \end{bmatrix}.
      \end{align*}

      \begin{align*}
        \alpha = (c_1,c_2,4c_1+5c_2,-c_1-4c_2,0).
      \end{align*}

      \begin{align*}
        [2,\ -1,\ 4(2) + 5(-1), -2 -4(-1)] &=\\
        (2,\ -1,\ 8 - 5, -2 +4) &=\\
        (2,\ -1,\ 3, 2).\\\\
        [-1,\ 1,\ 4(-1) + 5(1),-(-1) - 4(1)] &=\\
        (-1,\ 1,\ -4 + 5,\ 1 - 4) &=\\
        (-1,\ 1,\ 1,\ -3).\\\\
        [1,\ 1,\ 4(1) + 5(1),\ -(1) - 4(1)] &=\\
        (1,\ 1,\ 4 + 5,\ -1 - 4) &=\\
        (1,\ 1,\ 9,\ -5).\\\\
        [3,\ -1,\ 4(3) + 5(-1),\ -(3) -4(-1)] &=\\
        (3,\ -1,\ 12 - 5,\ -3 +4) &=\\
        (3,\ -1,\ 7,\ 1).
      \end{align*}

      So, $(3,-1,0,-1)$ is not in the subspace, although $(3,-1,7,1)$ is.

  \end{enumerate}
\end{document}
