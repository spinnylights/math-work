\documentclass[12pt]{article}
\usepackage{mathtools}
\usepackage{amsthm}
\usepackage{amsfonts}
\usepackage{amssymb}
\usepackage{fontspec}
\usepackage{xfrac}
\usepackage{array}
\usepackage{siunitx}
\usepackage{gensymb}
\usepackage{enumitem}
\usepackage{dirtytalk}
\usepackage{bm}
\title{Vector spaces (exercises)}
\author{Zoë Sparks}

\begin{document}

\theoremstyle{definition}

\sisetup{quotient-mode=fraction}
\newtheorem{thm}{Theorem}
\newtheorem*{nthm}{Theorem}
\newtheorem{sthm}{}[thm]
\newtheorem{lemma}{Lemma}[thm]
\newtheorem*{nlemma}{Lemma}
\newtheorem{cor}{Corollary}[thm]
\newtheorem*{prop}{Property}
\newtheorem*{defn}{Definition}
\newtheorem*{comm}{Comment}
\newtheorem*{exm}{Example}

\maketitle

\begin{enumerate}
  \item
    Given a field $F$, we would like to show that $F^n$ as we
    defined in the first example of \say{Vector spaces} is indeed
    a vector space over $F$.

    If $\alpha = (x_1,x_2,\ldots,x_n)$ of scalars $x_i$ in $F$
    and $\beta = (y_1,y_2,\ldots,y_n)$ with $y_i$ in $F$, we have
    that
    \begin{align*}
      \alpha + \beta = (x_1+y_1,\ x_2+y_2,\ \ldots,\ x_n+y_n),
    \end{align*}
    and thus that
    \begin{align*}
      \beta + \alpha = (y_1+x_1,\ y_2+x_2,\ \ldots,\ y_n+x_n).
    \end{align*}
    Since addition in $F$ is associative,
    \begin{align*}
      (x_1+y_1,\ x_2+y_2,\ \ldots,\ x_n+y_n) = (y_1+x_1,\
      y_2+x_2,\ \ldots,\ y_n+x_n),
    \end{align*}
    and thus
    \begin{align*}
      \alpha + \beta = \beta + \alpha.
    \end{align*}

    If $\gamma = (z_1,z_2,\ldots,z_n)$ of scalars $z_i$ in $F$,
    then
    \begin{align*}
      (\alpha + \beta) + \gamma = [(x_1+y_1)+z_1,\
      (x_2+y_2)+z_2,\ \ldots,\ (x_n+y_n)+z_n].
    \end{align*}
    Also,
    \begin{align*}
      \beta + \gamma = (y_1+z_1,\ y_2+z_2,\ \ldots,\ y_n+z_n),
    \end{align*}
    so
    \begin{align*}
      \alpha + (\beta + \gamma) = [x_1+(y_1+z_1),\
      x_2+(y_2+z_2),\ \ldots,\ x_n+(y_n+z_n)].
    \end{align*}
    Since addition in $F$ is commutative,
    \begin{align*}
      [x_1+(y_1+z_1),\ x_2+(y_2+z_2),\ \ldots,\ x_n+(y_n+z_n)]
      &=\\
      [(x_1+y_1)+z_1,\ (x_2+y_2)+z_2,\ \ldots,\ (x_n+y_n)+z_n],
    \end{align*}
    and thus
    \begin{align*}
      \alpha + (\beta + \gamma) = (\alpha + \beta) + \gamma.
    \end{align*}

    The zero vector in $F^{n}$ is the vector $0 =
    (p_1,p_2,\ldots,p_n)$ for which all $p_i = 0$. This is
    because
    \begin{align*}
      \alpha + 0 &= (x_1+0,\ x_2+0,\ \ldots,\ x_n+0)\\
                 &= (x_1,\ x_2,\ \ldots,\ x_n)\\
                 &= \alpha.
    \end{align*}

    The additive inverse of $\alpha = (x_1,x_2,\ldots,x_n)$ is
    the vector $-\alpha =\\ (-x_1,-x_2,\ldots,-x_n)$. This is
    because
    \begin{align*}
      \alpha + (-\alpha) &= (x_1-x_1,\ x_2-x_2,\ \ldots,\ x_n-x_n)\\
                         &= (0,0,\ldots,0)\\
                         &= 0.
    \end{align*}

    For a scalar $c$, we have that
    \begin{align*}
      c\alpha = (cx_1,cx_2,\ldots,cx_3).
    \end{align*}
    Then
    \begin{align*}
      c(\alpha + \beta) &= [c(x_1+y_1),\ c(x_2+y_2),\ \ldots,\ c(x_n+y_n)]\\
                        &= (cx_1+cy_1,\ cx_2+cy_2,\ \ldots,\ cx_n+cy_n)\\
                        &= c\alpha + c\beta,\\\\
      (c_1+c_2)\alpha &= [(c_1+c_2)x_1,\ (c_1+c_2)x_2,\ \ldots,\ (c_1+c_2)x_n]\\
                      &= (c_1x_1+c_2x_1,\ c_1x_2+c_2x_2,\ \ldots,\ c_1x_n+c_2x_n)\\
                      &= c_1\alpha + c_2\alpha,\\
      \shortintertext{and}
      (c_1c_2)\alpha &= [(c_1c_2)x_1,\ (c_1c_2)x_2,\ \ldots,\ (c_1c_2)x_n]\\
                     &= [c_1(c_2x_1),\ c_1(c_2x_2),\ \ldots,\ c_1(c_2x_n)]\\
                     &= c_1(c_2\alpha).
    \end{align*}
    Also, for every $\alpha$ in $V$, we have that
    \begin{align*}
      1\alpha &= (1x_1,\ 1x_2,\ \ldots,\ 1x_n)\\
              &= (x_1,\ x_2,\ \ldots,\ x_n)\\
              &= \alpha.
    \end{align*}

    Thus, our definitions of vector addition and scalar multiplication for $F^{n}$
    are proper by way of a vector space.

  \item
    Given a field $F$ and a vector space $V$ over it, we would like to verify that
    \begin{align*}
      (\alpha_1 + \alpha_2) + (\alpha_3 + \alpha_4) =
      [\alpha_2 + (\alpha_3 + \alpha_1)] + \alpha_4
    \end{align*}
    for all vectors $\alpha_1,\alpha_2,\alpha_3,\alpha_4$ in $V$.

    By 3(a) and 3(b),
    \begin{align*}
      (\alpha_1 + \alpha_2) + (\alpha_3 + \alpha_4) &=
      (\alpha_2 + \alpha_1) + (\alpha_3 + \alpha_4)\\
      &= \alpha_2 + (\alpha_1 + \alpha_3) + \alpha_4\\
      &= \alpha_2 + (\alpha_3 + \alpha_1) + \alpha_4\\
      &= [\alpha_2 + (\alpha_3 + \alpha_1)] + \alpha_4.
    \end{align*}
\end{enumerate}

\end{document}
