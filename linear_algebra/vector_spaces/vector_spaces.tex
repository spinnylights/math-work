\documentclass[12pt]{article}
\usepackage{mathtools}
\usepackage{amsthm}
\usepackage{amsfonts}
\usepackage{amssymb}
\usepackage{fontspec}
\usepackage{xfrac}
\usepackage{array}
\usepackage{siunitx}
\usepackage{gensymb}
\usepackage{enumitem}
\usepackage{dirtytalk}
\usepackage{bm}
\title{Vector spaces}
\author{Zoë Sparks}

\begin{document}

\theoremstyle{definition}

\sisetup{quotient-mode=fraction}
\newtheorem{thm}{Theorem}
\newtheorem*{nthm}{Theorem}
\newtheorem{sthm}{}[thm]
\newtheorem{lemma}{Lemma}[thm]
\newtheorem*{nlemma}{Lemma}
\newtheorem{cor}{Corollary}[thm]
\newtheorem*{prop}{Property}
\newtheorem*{defn}{Definition}
\newtheorem*{comm}{Comment}
\newtheorem*{exm}{Example}

\maketitle

\begin{defn}
  A \textbf{vector space}, also known as a linear space, consists
  of:
  \begin{enumerate}
    \item
      a field $F$ of scalars;
    \item
      a set $V$ of objects called \textbf{vectors};
    \item
      a rule or operation called \textbf{vector addition},
      which associates each pair of vectors $\alpha,\ \beta$ in
      $V$ with a vector $\alpha + \beta$ in $V$, called the sum
      of $\alpha$ and $\beta$, and which fulfills the criteria
      that
      \begin{enumerate}
        \item
          vector addition is commutative: $\alpha + \beta = \beta
          + \alpha$;
        \item
          vector addition is associative: $\alpha + (\beta +
          \gamma) = (\alpha + \beta) + \gamma$;
        \item
          there is a unique vector $0$ in $V$ called the
          \textbf{zero vector} and which is the additive identity
          for vector addition, i.e. $\alpha + 0 = \alpha$ for all
          $\alpha$ in $V$;
        \item
          for each vector $\alpha$ in $V$ there is a unique
          vector $-\alpha$ in $V$ which is the additive inverse
          of $\alpha$, i.e. $\alpha + (-\alpha) = 0$;
      \end{enumerate}
    \item
      a rule or operation called \textbf{scalar multiplication},
      which associates each scalar $c$ in $F$ and vector $\alpha$
      in $V$ with a vector $c\alpha$ in $V$, called the product
      of $c$ and $\alpha$, and which fulfills the criteria that
      \begin{enumerate}
        \item
          $c(\alpha + \beta) = c\alpha + c\beta$;
        \item
           $(c_1 +c_2)\alpha = c_1\alpha + c_2\alpha$;
        \item
          $(c_1c_2)\alpha = c_1(c_2\alpha)$;
        \item
          $1\alpha = \alpha$ for every $\alpha$ in $V$.
      \end{enumerate}
  \end{enumerate}
  Sometimes we will refer to a vector space simply as $V$ if
  there is no chance of confusion about the specific vector space
  under discussion. In some cases, we will need to specify the
  field, in which case we will say that $V$ is a \textbf{vector
  space over the field} $F$.
\end{defn}

\begin{comm}
  It's important to note that a vector space is a
  \textit{composite} object, consisting of a field $F$, a set $V$
  of some kind of objects called \say{vectors,} and two
  operations with certain properties. You may have certain
  preconceived ideas about what constitutes a vector, but it's
  worth putting those aside to an extent if so, as the vectors of
  a given vector space may be quite different from what you
  currently imagine. In particular, the elements of $V$ need have
  no intrinsic relationship to $F$ aside from what's needed to
  define vector addition and scalar multiplication between them.
\end{comm}

\end{document}
