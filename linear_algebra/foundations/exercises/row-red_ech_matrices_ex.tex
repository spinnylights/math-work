\documentclass[12pt]{article}
\usepackage{mathtools}
\usepackage{amsthm}
\usepackage{amsfonts}
\usepackage{amssymb}
\usepackage{fontspec}
\usepackage{xfrac}
\usepackage{array}
\usepackage{siunitx}
\usepackage{gensymb}
\usepackage{enumitem}
\usepackage{dirtytalk}
\title{Row-reduced echelon matrices (exercises)}
\author{Zoë Sparks}

\begin{document}

\theoremstyle{definition}

\sisetup{quotient-mode=fraction}
\newtheorem{thm}{Theorem}
\newtheorem*{nthm}{Theorem}
\newtheorem{sthm}{}[thm]
\newtheorem{lemma}{Lemma}[thm]
\newtheorem{cor}{Corollary}[thm]
\newtheorem*{prop}{Property}
\newtheorem*{defn}{Definition}
\newtheorem*{comm}{Comment}
\newtheorem*{exm}{Example}

\maketitle

\begin{enumerate}
  \item
    We would like to find all solutions to
    \begin{alignat*}{8}
      \frac{1}{3}x_1\  &+&\ 2x_2\ &-&\  6x_3\          &=&\ &0\\
      -4x_1\           & &        &+&\  5x_3\          &=&\ &0\\
      -3x_1\           &+&\ 6x_2\ &-&\ 13x_3\          &=&\ &0\\
      -\frac{7}{3}x_1\ &+&\ 2x_2\ &-&\ \frac{8}{3}x_3\ &=&\ &0.
    \end{alignat*}
    We can proceed as follows:
    \begin{align*}
      \begin{bmatrix}
        \frac{1}{3}  & 2 & -6          \\
        -4           & 0 & 5           \\
        -3           & 6 & -13         \\
        -\frac{7}{3} & 2 & -\frac{8}{3}
      \end{bmatrix}
      \xrightarrow{(1)}
      \begin{bmatrix}
        \frac{1}{3}  & 2 & -6          \\
        1            & 0 & -\frac{5}{4}\\
        -3           & 6 & -13         \\
        -\frac{7}{3} & 2 & -\frac{8}{3}
      \end{bmatrix}
      \xrightarrow{(2)}
    \end{align*}
    \begin{align*}
      \begin{bmatrix}
        0            & 2 & -\frac{67}{12}\\
        1            & 0 & -\frac{5}{4}  \\
        -3           & 6 & -13           \\
        -\frac{7}{3} & 2 & -\frac{8}{3}
      \end{bmatrix}
      \xrightarrow{(2)}
      \begin{bmatrix}
        0            & 2 & -\frac{67}{12}\\
        1            & 0 & -\frac{5}{4}  \\
        0            & 6 & -\frac{67}{4} \\
        -\frac{7}{3} & 2 & -\frac{8}{3}
      \end{bmatrix}
      \xrightarrow{(2)}
    \end{align*}
    \begin{align*}
      \begin{bmatrix}
        0 & 2 & -\frac{67}{12}\\
        1 & 0 & -\frac{5}{4}  \\
        0 & 6 & -\frac{67}{4} \\
        0 & 2 & -\frac{67}{12}
      \end{bmatrix}
      \xrightarrow{(2)}
      \begin{bmatrix}
        0 & 2 & -\frac{67}{12}\\
        1 & 0 & -\frac{5}{4}  \\
        0 & 6 & -\frac{67}{4} \\
        0 & 0 & 0
      \end{bmatrix}
      \xrightarrow{(2)}
    \end{align*}
    \begin{align*}
      \begin{bmatrix}
        0 & 2 & -\frac{67}{12}\\
        1 & 0 & -\frac{5}{4}  \\
        0 & 0 & 0             \\
        0 & 0 & 0
      \end{bmatrix}
      \xrightarrow{(1)}
      \begin{bmatrix}
        0 & 1 & -\frac{67}{24}\\
        1 & 0 & -\frac{5}{4}  \\
        0 & 0 & 0             \\
        0 & 0 & 0
      \end{bmatrix}
      \xrightarrow{(3)}
    \end{align*}
    \begin{align*}
      \begin{bmatrix}
        1 & 0 & -\frac{5}{4}  \\
        0 & 1 & -\frac{67}{24}\\
        0 & 0 & 0             \\
        0 & 0 & 0
      \end{bmatrix}.
    \end{align*}
    Therefore, if we say $x_3 = c$, the solutions are given by
    \begin{align*}
      x_1 =&\ \frac{5}{4}c\\
      x_2 =&\ \frac{67}{24}c.
    \end{align*}

  \item
    We would like to find a row-reduced echelon matrix which is
    row-equivalent to
    \begin{align*}
      A =
      \begin{bmatrix}
        1 & -i\\
        2 & 2\\
        i & (1 + i)
      \end{bmatrix}.
    \end{align*}
    We can proceed as follows:
    \begin{align*}
      \begin{bmatrix}
        1 & -i\\
        2 & 2\\
        i & (1 + i)
      \end{bmatrix}
      \xrightarrow{(2)}
      \begin{bmatrix}
        1 & -i\\
        0 & (2 + 2i)\\
        i & (1 + i)
      \end{bmatrix}
      \xrightarrow{(2)}
    \end{align*}
    \begin{align*}
      \begin{bmatrix}
        1 & -i\\
        0 & (2 + 2i)\\
        0 & i
      \end{bmatrix}
      \xrightarrow{(1)}
      \begin{bmatrix}
        1 & -i\\
        0 & 1\\
        0 & i
      \end{bmatrix}
      \xrightarrow{(2)}
    \end{align*}
    \begin{align*}
      \begin{bmatrix}
        1 & -i\\
        0 & 1\\
        0 & 0
      \end{bmatrix}
      \xrightarrow{(2)}
      \begin{bmatrix}
        1 & 0\\
        0 & 1\\
        0 & 0
      \end{bmatrix}.
    \end{align*}
    We would also like to know the solutions of $AX = 0$; this
    shows they are $(x_1,\ x_2) = (0,0)$.

  \item
    We would like to explicitly describe all $2 \times 2$
    row-reduced echelon matrices. They are:
    \begin{align*}
      \begin{bmatrix}
        1 & 0\\
        0 & 1
      \end{bmatrix},\ 
      \begin{bmatrix}
        1 & a\\
        0 & 0
      \end{bmatrix},\ 
      \begin{bmatrix}
        0 & 1\\
        0 & 0
      \end{bmatrix},\ \text{and}\ 
      \begin{bmatrix}
        0 & 0\\
        0 & 0
      \end{bmatrix}.
    \end{align*}

  \item
    We consider the system of equations
    \begin{alignat*}{7}
      x_1\ &-&\ x_2\ &+&\ 2x_3\ &=&\ &1\\
      2x_1\ &&\ \ &+&\ 2x_3\ &=&\ &1\\
      x_1\ &-&\ 3x_2\ &+&\ 4x_3\ &=&\ &2.
    \end{alignat*}
    We would like to know if it has a solution, and if so, what
    its solutions are.

    We proceed as follows:
    \begin{align*}
      \begin{bmatrix}
        1 & -1 & 2 & 1\\
        2 & 0  & 2 & 1\\
        1 & -3 & 4 & 2
      \end{bmatrix}
      \xrightarrow{(2)}
      \begin{bmatrix}
        1 & -1 & 2  & 1\\
        0 & 2  & -2 & -1\\
        1 & -3 & 4  & 2
      \end{bmatrix}
      \xrightarrow{(2)}
    \end{align*}
    \begin{align*}
      \begin{bmatrix}
        1 & -1 & 2  & 1\\
        0 & 2  & -2 & -1\\
        0 & -2 & 2  & 0
      \end{bmatrix}
      \xrightarrow{(2)}
      \begin{bmatrix}
        1 & -1 & 2  & 1\\
        0 & 2  & -2 & -1\\
        0 & 0  & 0  & -1
      \end{bmatrix}
      \xrightarrow{(1)}
    \end{align*}
    \begin{align*}
      \begin{bmatrix}
        1 & -1 & 2  & 1\\
        0 & 1  & -1 & -\frac{1}{2}\\
        0 & 0  & 0  & -1
      \end{bmatrix}
      \xrightarrow{(2)}
      \begin{bmatrix}
        1 & 0 & 1  & \frac{1}{2}\\
        0 & 1 & -1 & -\frac{1}{2}\\
        0 & 0 & 0  & -1
      \end{bmatrix}.
    \end{align*}
    We can see here that this system has no solutions, as there
    is no way for $0x_1 + 0x_2 + 0x_3$ to equal $-1$.

  \item
    We would like to give an example of a system of two linear
    equations in two unknowns which has no solution. One rather
    obvious example is
    \begin{alignat*}{6}
      x_1\ &+&\ x_2\ &=&\ &0\\
      x_1\ &+&\ x_2\ &=&\ &1.
    \end{alignat*}
\end{enumerate}

\end{document}
