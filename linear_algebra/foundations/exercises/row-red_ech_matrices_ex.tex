\documentclass[12pt]{article}
\usepackage{mathtools}
\usepackage{amsthm}
\usepackage{amsfonts}
\usepackage{amssymb}
\usepackage{fontspec}
\usepackage{xfrac}
\usepackage{array}
\usepackage{siunitx}
\usepackage{gensymb}
\usepackage{enumitem}
\usepackage{dirtytalk}
\title{Row-reduced echelon matrices (exercises)}
\author{Zoë Sparks}

\begin{document}

\theoremstyle{definition}

\sisetup{quotient-mode=fraction}
\newtheorem{thm}{Theorem}
\newtheorem*{nthm}{Theorem}
\newtheorem{sthm}{}[thm]
\newtheorem{lemma}{Lemma}[thm]
\newtheorem{cor}{Corollary}[thm]
\newtheorem*{prop}{Property}
\newtheorem*{defn}{Definition}
\newtheorem*{comm}{Comment}
\newtheorem*{exm}{Example}

\maketitle

\begin{enumerate}
  \item
    We would like to find all solutions to
    \begin{alignat*}{8}
      \frac{1}{3}x_1\  &+&\ 2x_2\ &-&\  6x_3\          &=&\ &0\\
      -4x_1\           & &        &+&\  5x_3\          &=&\ &0\\
      -3x_1\           &+&\ 6x_2\ &-&\ 13x_3\          &=&\ &0\\
      -\frac{7}{3}x_1\ &+&\ 2x_2\ &-&\ \frac{8}{3}x_3\ &=&\ &0.
    \end{alignat*}
    We can proceed as follows:
    \begin{align*}
      \begin{bmatrix}
        \frac{1}{3}  & 2 & -6          \\
        -4           & 0 & 5           \\
        -3           & 6 & -13         \\
        -\frac{7}{3} & 2 & -\frac{8}{3}
      \end{bmatrix}
      \xrightarrow{(1)}
      \begin{bmatrix}
        \frac{1}{3}  & 2 & -6          \\
        1            & 0 & -\frac{5}{4}\\
        -3           & 6 & -13         \\
        -\frac{7}{3} & 2 & -\frac{8}{3}
      \end{bmatrix}
      \xrightarrow{(2)}
    \end{align*}
    \begin{align*}
      \begin{bmatrix}
        0            & 2 & -\frac{67}{12}\\
        1            & 0 & -\frac{5}{4}  \\
        -3           & 6 & -13           \\
        -\frac{7}{3} & 2 & -\frac{8}{3}
      \end{bmatrix}
      \xrightarrow{(2)}
      \begin{bmatrix}
        0            & 2 & -\frac{67}{12}\\
        1            & 0 & -\frac{5}{4}  \\
        0            & 6 & -\frac{67}{4} \\
        -\frac{7}{3} & 2 & -\frac{8}{3}
      \end{bmatrix}
      \xrightarrow{(2)}
    \end{align*}
    \begin{align*}
      \begin{bmatrix}
        0 & 2 & -\frac{67}{12}\\
        1 & 0 & -\frac{5}{4}  \\
        0 & 6 & -\frac{67}{4} \\
        0 & 2 & -\frac{67}{12}
      \end{bmatrix}
      \xrightarrow{(2)}
      \begin{bmatrix}
        0 & 2 & -\frac{67}{12}\\
        1 & 0 & -\frac{5}{4}  \\
        0 & 6 & -\frac{67}{4} \\
        0 & 0 & 0
      \end{bmatrix}
      \xrightarrow{(2)}
    \end{align*}
    \begin{align*}
      \begin{bmatrix}
        0 & 2 & -\frac{67}{12}\\
        1 & 0 & -\frac{5}{4}  \\
        0 & 0 & 0             \\
        0 & 0 & 0
      \end{bmatrix}
      \xrightarrow{(1)}
      \begin{bmatrix}
        0 & 1 & -\frac{67}{24}\\
        1 & 0 & -\frac{5}{4}  \\
        0 & 0 & 0             \\
        0 & 0 & 0
      \end{bmatrix}
      \xrightarrow{(3)}
    \end{align*}
    \begin{align*}
      \begin{bmatrix}
        1 & 0 & -\frac{5}{4}  \\
        0 & 1 & -\frac{67}{24}\\
        0 & 0 & 0             \\
        0 & 0 & 0
      \end{bmatrix}.
    \end{align*}
    Therefore, if we say $x_3 = c$, the solutions are given by
    \begin{align*}
      x_1 =&\ \frac{5}{4}c\\
      x_2 =&\ \frac{67}{24}c.
    \end{align*}

  \item
    We would like to find a row-reduced echelon matrix which is
    row-equivalent to
    \begin{align*}
      A =
      \begin{bmatrix}
        1 & -i\\
        2 & 2\\
        i & (1 + i)
      \end{bmatrix}.
    \end{align*}
    We can proceed as follows:
    \begin{align*}
      \begin{bmatrix}
        1 & -i\\
        2 & 2\\
        i & (1 + i)
      \end{bmatrix}
      \xrightarrow{(2)}
      \begin{bmatrix}
        1 & -i\\
        0 & (2 + 2i)\\
        i & (1 + i)
      \end{bmatrix}
      \xrightarrow{(2)}
    \end{align*}
    \begin{align*}
      \begin{bmatrix}
        1 & -i\\
        0 & (2 + 2i)\\
        0 & i
      \end{bmatrix}
      \xrightarrow{(1)}
      \begin{bmatrix}
        1 & -i\\
        0 & 1\\
        0 & i
      \end{bmatrix}
      \xrightarrow{(2)}
    \end{align*}
    \begin{align*}
      \begin{bmatrix}
        1 & -i\\
        0 & 1\\
        0 & 0
      \end{bmatrix}
      \xrightarrow{(2)}
      \begin{bmatrix}
        1 & 0\\
        0 & 1\\
        0 & 0
      \end{bmatrix}.
    \end{align*}
    We would also like to know the solutions of $AX = 0$; this
    shows they are $(x_1,\ x_2) = (0,0)$.

  \item
    We would like to explicitly describe all $2 \times 2$
    row-reduced echelon matrices. They are:
    \begin{align*}
      \begin{bmatrix}
        1 & 0\\
        0 & 1
      \end{bmatrix},\ 
      \begin{bmatrix}
        1 & a\\
        0 & 0
      \end{bmatrix},\ 
      \begin{bmatrix}
        0 & 1\\
        0 & 0
      \end{bmatrix},\ \text{and}\ 
      \begin{bmatrix}
        0 & 0\\
        0 & 0
      \end{bmatrix}.
    \end{align*}

  \item
    We consider the system of equations
    \begin{alignat*}{7}
      x_1\ &-&\ x_2\ &+&\ 2x_3\ &=&\ &1\\
      2x_1\ &&\ \ &+&\ 2x_3\ &=&\ &1\\
      x_1\ &-&\ 3x_2\ &+&\ 4x_3\ &=&\ &2.
    \end{alignat*}
    We would like to know if it has a solution, and if so, what
    its solutions are.

    We proceed as follows:
    \begin{align*}
      \begin{bmatrix}
        1 & -1 & 2 & 1\\
        2 & 0  & 2 & 1\\
        1 & -3 & 4 & 2
      \end{bmatrix}
      \xrightarrow{(2)}
      \begin{bmatrix}
        1 & -1 & 2  & 1\\
        0 & 2  & -2 & -1\\
        1 & -3 & 4  & 2
      \end{bmatrix}
      \xrightarrow{(2)}
    \end{align*}
    \begin{align*}
      \begin{bmatrix}
        1 & -1 & 2  & 1\\
        0 & 2  & -2 & -1\\
        0 & -2 & 2  & 0
      \end{bmatrix}
      \xrightarrow{(2)}
      \begin{bmatrix}
        1 & -1 & 2  & 1\\
        0 & 2  & -2 & -1\\
        0 & 0  & 0  & -1
      \end{bmatrix}
      \xrightarrow{(1)}
    \end{align*}
    \begin{align*}
      \begin{bmatrix}
        1 & -1 & 2  & 1\\
        0 & 1  & -1 & -\frac{1}{2}\\
        0 & 0  & 0  & -1
      \end{bmatrix}
      \xrightarrow{(2)}
      \begin{bmatrix}
        1 & 0 & 1  & \frac{1}{2}\\
        0 & 1 & -1 & -\frac{1}{2}\\
        0 & 0 & 0  & -1
      \end{bmatrix}.
    \end{align*}
    We can see here that this system has no solutions, as there
    is no way for $0x_1 + 0x_2 + 0x_3$ to equal $-1$.

  \item
    We would like to give an example of a system of two linear
    equations in two unknowns which has no solution. One rather
    obvious example is
    \begin{alignat*}{6}
      x_1\ &+&\ x_2\ &=&\ &0\\
      x_1\ &+&\ x_2\ &=&\ &1.
    \end{alignat*}

  \item
    We would like to show that
    \begin{alignat*}{10}
      x_1\ &-&\ 2x_2\ &+&\ x_3\  &+&\ 2x_4\ &=&\ 1&\\
      x_1\ &+&\ x_2\  &-&\ x_3\  &+&\ x_4\  &=&\ 2&\\
      x_1\ &+&\ 7x_2\ &-&\ 5x_3\ &-&\ x_4\  &=&\ 3&
    \end{alignat*}
    has no solution.
    We proceed as follows:
    \begin{align*}
      \begin{bmatrix}
        1 & -2 & 1  & 2  & 1\\
        1 & 1  & -1 & 1  & 2\\
        1 & 7  & -5 & -1 & 3
      \end{bmatrix}
      \xrightarrow{(2)}
      \begin{bmatrix}
        1 & -2 & 1  & 2  & 1\\
        0 & 3  & -2 & -1 & 1\\
        1 & 7  & -5 & -1 & 3
      \end{bmatrix}
      \xrightarrow{(2)}
    \end{align*}
    \begin{align*}
      \begin{bmatrix}
        1 & -2 & 1  & 2  & 1\\
        0 & 3  & -2 & -1 & 1\\
        0 & 9  & -6 & -3 & 2
      \end{bmatrix}
      \xrightarrow{(2)}
      \begin{bmatrix}
        1 & -2 & 1  & 2  & 1\\
        0 & 3  & -2 & -1 & 1\\
        0 & 0  & 0  & 0  & -1
      \end{bmatrix}.
    \end{align*}
    In truth, we don't even need to continue past this point.
    This already shows that the system we started with is
    equivalent to one with an equation of $0 = -1$, a
    contradiction, and thus neither has a solution.

  \item
    We would like to find all solutions of
    \begin{alignat*}{12}
      2x_1\ &-&\ 3x_2\ &-&\ 7x_3\ &+&\ 5x_4\ &+&\ 2x_5\
        &=&\ -2&\\
      x_1\  &-&\ 2x_2\ &-&\ 4x_3\ &+&\ 3x_4\ &+&\ x_5\
        &=&\ -2&\\
      2x_1\ & &\     \ &-&\ 4x_3\ &+&\ 2x_4\ &+&\ x_5\
        &=&\ 3&\\
      x_1\  &-&\ 5x_2\ &-&\ 7x_3\ &+&\ 6x_4\ &+&\ 2x_5\
        &=&\ -7&.
    \end{alignat*}
    We proceed as follows:
    \begin{align*}
      \begin{bmatrix}
        2 & -3 & -7 & 5 & 2 & -2\\
        1 & -2 & -4 & 3 & 1 & -2\\
        2 & 0  & -4 & 2 & 1 & 3\\
        1 & -5 & -7 & 6 & 2 & -7
      \end{bmatrix}
      \xrightarrow{(2)}
      \begin{bmatrix}
        0 & -3 & -3 & 3 & 1 & -5\\
        1 & -2 & -4 & 3 & 1 & -2\\
        2 & 0  & -4 & 2 & 1 & 3\\
        1 & -5 & -7 & 6 & 2 & -7
      \end{bmatrix}
      \xrightarrow{(1)}
    \end{align*}
    \begin{align*}
      \begin{bmatrix}
        0 & -3 & -3 & 3 & 1           & -5\\
        1 & -2 & -4 & 3 & 1           & -2\\
        1 & 0  & -2 & 1 & \frac{1}{2} & \frac{3}{2}\\
        1 & -5 & -7 & 6 & 2           & -7
      \end{bmatrix}
      \xrightarrow{(2)}
      \begin{bmatrix}
        0 & -3 & -3 & 3 & 1           & -5\\
        1 & -2 & -4 & 3 & 1           & -2\\
        1 & 0  & -2 & 1 & \frac{1}{2} & \frac{3}{2}\\
        0 & -3 & -3 & 3 & 1           & -5
      \end{bmatrix}
      \xrightarrow{(2)}
    \end{align*}
    \begin{align*}
      \begin{bmatrix}
        0 & -3 & -3 & 3 & 1           & -5\\
        0 & -2 & -2 & 2 & \frac{1}{2} & -\frac{7}{2}\\
        1 & 0  & -2 & 1 & \frac{1}{2} & \frac{3}{2}\\
        0 & -3 & -3 & 3 & 1           & -5
      \end{bmatrix}
      \xrightarrow{(2)}
      \begin{bmatrix}
        0 & -3 & -3 & 3 & 1           & -5\\
        0 & -2 & -2 & 2 & \frac{1}{2} & -\frac{7}{2}\\
        1 & 0  & -2 & 1 & \frac{1}{2} & \frac{3}{2}\\
        0 & 0  & 0  & 0 & 0           & 0
      \end{bmatrix}
      \xrightarrow{(1)}
    \end{align*}
    \begin{align*}
      \begin{bmatrix}
        0 & 1  & 1  & -1 & -\frac{1}{3} & \frac{5}{3}\\
        0 & -2 & -2 & 2  & \frac{1}{2}  & -\frac{7}{2}\\
        1 & 0  & -2 & 1  & \frac{1}{2}  & \frac{3}{2}\\
        0 & 0  & 0  & 0  & 0            & 0
      \end{bmatrix}
      \xrightarrow{(2)}
      \begin{bmatrix}
        0 & 1 & 1  & -1 & -\frac{1}{3} & \frac{5}{3}\\
        0 & 0 & 0  & 0  & -\frac{1}{6}  & -\frac{1}{6}\\
        1 & 0 & -2 & 1  & \frac{1}{2}  & \frac{3}{2}\\
        0 & 0 & 0  & 0  & 0            & 0
      \end{bmatrix}
      \xrightarrow{(1)}
    \end{align*}
    \begin{align*}
      \begin{bmatrix}
        0 & 1 & 1  & -1 & -\frac{1}{3} & \frac{5}{3}\\
        0 & 0 & 0  & 0  & 1            & 1\\
        1 & 0 & -2 & 1  & \frac{1}{2}  & \frac{3}{2}\\
        0 & 0 & 0  & 0  & 0            & 0
      \end{bmatrix}
      \xrightarrow{(2)}
      \begin{bmatrix}
        0 & 1 & 1  & -1 & -\frac{1}{3} & \frac{5}{3}\\
        0 & 0 & 0  & 0  & 1            & 1\\
        1 & 0 & -2 & 1  & 0            & 1\\
        0 & 0 & 0  & 0  & 0            & 0
      \end{bmatrix}
      \xrightarrow{(2)}
    \end{align*}
    \begin{align*}
      \begin{bmatrix}
        0 & 1 & 1  & -1 & 0 & 2\\
        0 & 0 & 0  & 0  & 1 & 1\\
        1 & 0 & -2 & 1  & 0 & 1\\
        0 & 0 & 0  & 0  & 0 & 0
      \end{bmatrix}
      \xrightarrow{(3)}
      \begin{bmatrix}
        1 & 0 & -2 & 1  & 0 & 1\\
        0 & 1 & 1  & -1 & 0 & 2\\
        0 & 0 & 0  & 0  & 1 & 1\\
        0 & 0 & 0  & 0  & 0 & 0
      \end{bmatrix}.
    \end{align*}
    Therefore, we can see that the solutions are given by
    \begin{align*}
      x_1 =&\ 2c - d + 1\\
      x_2 =&\ -c + d + 2\\
      x_3 =&\ c\\
      x_4 =&\ d\\
      x_5 =&\ 1
    \end{align*}
    for scalars $c,\ d$.

  \item
    We have
    \begin{align*}
      A =
      \begin{bmatrix}
        3 & -1 & 2\\
        2 & 1  & 1\\
        1 & -3 & 0\\
      \end{bmatrix}.
    \end{align*}
    We would like to know for which triples $(y_1,\ y_2,\ y_3)$
    the system $AX = Y$ has a solution.

    We proceed as follows:
    \begin{align*}
      \begin{bmatrix}
        3 & -1 & 2 & y_1\\
        2 & 1  & 1 & y_2\\
        1 & -3 & 0 & y_3\\
      \end{bmatrix}
      \xrightarrow{(2)}
      \begin{bmatrix}
        0 & 8  & 2 & (y_1 - 3y_3)\\
        2 & 1  & 1 & y_2\\
        1 & -3 & 0 & y_3\\
      \end{bmatrix}
      \xrightarrow{(2)}
    \end{align*}
    \begin{align*}
      \begin{bmatrix}
        0 & 8  & 2 & (y_1 - 3y_3)\\
        0 & 7  & 1 & (y_2 - 2y_3)\\
        1 & -3 & 0 & y_3\\
      \end{bmatrix}
      \xrightarrow{(1)}
      \begin{bmatrix}
        0 & 1  & \frac{1}{4} & \frac{1}{8}(y_1 - 3y_3)\\
        0 & 7  & 1           & (y_2 - 2y_3)\\
        1 & -3 & 0           & y_3\\
      \end{bmatrix}
      \xrightarrow{(2)}
    \end{align*}
    \begin{align*}
      \begin{bmatrix}
        0 & 1  & \frac{1}{4}  & \frac{1}{8}(y_1 - 3y_3)\\
        0 & 0  & -\frac{3}{4} & -\frac{1}{8}(y_1 - 8y_2 + 19y_3)\\
        1 & -3 & 0            & y_3\\
      \end{bmatrix}
      \xrightarrow{(2)}
      \begin{bmatrix}
        0 & 1 & \frac{1}{4}  & \frac{1}{8}(y_1 - 3y_3)\\
        0 & 0 & -\frac{3}{4} & -\frac{1}{8}(y_1 - 8y_2 + 19y_3)\\
        1 & 0 & \frac{3}{4}  & \frac{1}{8}(y_1 + 5y_3)\\
      \end{bmatrix}
      \xrightarrow{(1)}
    \end{align*}
    \begin{align*}
      \begin{bmatrix}
        0 & 1 & \frac{1}{4}  & \frac{1}{8}(y_1 - 3y_3)\\
        0 & 0 & 1            & \frac{1}{6}(y_1 - 8y_2 + 19y_3)\\
        1 & 0 & \frac{3}{4}  & \frac{1}{8}(y_1 + 5y_3)\\
      \end{bmatrix}
      \xrightarrow{(2)}
      \begin{bmatrix}
        0 & 1 & 0            & \frac{1}{12}(y_1 + 4y_2 - 14y_3)\\
        0 & 0 & 1            & \frac{1}{6}(y_1 - 8y_2 + 19y_3)\\
        1 & 0 & \frac{3}{4}  & \frac{1}{8}(y_1 + 5y_3)\\
      \end{bmatrix}
      \xrightarrow{(2)}
    \end{align*}
    \begin{align*}
      \begin{bmatrix}
        0 & 1 & 0 & \frac{1}{12}(y_1 + 4y_2 - 14y_3)\\
        0 & 0 & 1 & \frac{1}{6}(y_1 - 8y_2 + 19y_3)\\
        1 & 0 & 0 & \frac{1}{12}(12y_2 - 21y_3)
      \end{bmatrix}
      \xrightarrow{(3)}
      \begin{bmatrix}
        1 & 0 & 0 & \frac{1}{12}(12y_2 - 21y_3)\\
        0 & 1 & 0 & \frac{1}{12}(y_1 + 4y_2 - 14y_3)\\
        0 & 0 & 1 & \frac{1}{6}(y_1 - 8y_2 + 19y_3)
      \end{bmatrix}.
    \end{align*}
    In truth, $AX = Y$ has a solution for any set of triples
    $(y_1,\ y_2,\ y_3)$. The solutions are given by:
    \begin{align*}
      x_1 &=\ \frac{1}{12}(12y_2 - 21y_3)\\
      x_2 &=\ \frac{1}{12}(y_1 + 4y_2 - 14y_3)\\
      x_3 &=\ \frac{1}{6}(y_1 - 8y_2 + 19y_3).
    \end{align*}

  \item
    We would like to prove that, if $R$ and $R'$ are $2 \times 3$
    row-reduced echelon matrices and the systems $RX = 0$ and
    $R'X = 0$ have exactly the same solutions, $R = R'$.

    The possible $2 \times 3$ row-reduced echelon matrices are:
    \begin{align*}
      \begin{bmatrix}
        1 & 0 & p\\
        0 & 1 & q\\
      \end{bmatrix},\ 
      \begin{bmatrix}
        0 & 1 & 0\\
        0 & 0 & 1\\
      \end{bmatrix},\ 
      \begin{bmatrix}
        1 & p & q\\
        0 & 0 & 0\\
      \end{bmatrix},\ 
      \begin{bmatrix}
        0 & 1 & q\\
        0 & 0 & 0\\
      \end{bmatrix},
    \end{align*}
    \begin{align*}
      \begin{bmatrix}
        0 & 0 & 1\\
        0 & 0 & 0\\
      \end{bmatrix},\ \text{and}\ 
      \begin{bmatrix}
        0 & 0 & 0\\
        0 & 0 & 0\\
      \end{bmatrix}.
    \end{align*}
    If we treat any of these as $R$ and $p,\ q,\ \alpha,\ \beta,\
    \gamma$ as any scalars in the field, the solutions in turn
    would be:
    \begin{align*}
      x_1 =&\ -p\alpha\\
      x_2 =&\ -q\alpha\\
      x_3 =&\ \alpha;\\\\
      x_1 =&\ \alpha\\
      x_2 =&\ 0\\
      x_3 =&\ 0;\\\\
      x_1 =&\ -p\alpha -q\beta\\
      x_2 =&\ \alpha\\
      x_3 =&\ \beta;\\\\
      x_1 =&\ \alpha\\
      x_2 =&\ -q\beta\\
      x_3 =&\ \beta;\\\\
      x_1 =&\ \alpha\\
      x_2 =&\ \beta\\
      x_3 =&\ 0;\\\\
      x_1 =&\ \alpha\\
      x_2 =&\ \beta\\
      x_3 =&\ \gamma.
    \end{align*}
    We can state the solution sets of these systems in a clearer
    way as
    \begin{align*}
      (\alpha,&\ 0,\ 0),\\
      (\alpha,&\ \beta,\ 0),\\
      (\alpha,&\ \beta,\ -p\alpha),\\
      (\alpha,&\ \beta,\ -q\beta),\\
      (\alpha,&\ \beta,\ -p\alpha -q\beta),\\
      (\alpha,&\ \beta,\ \gamma).
    \end{align*}
    In each of these sets, $\alpha,\ \beta,\ \gamma$ span the
    whole field, whereas $p$ and $q$ are fixed for a given
    solution set. As such, we can make the situation even clearer
    by taking them out of the picture for a moment:
    \begin{align*}
      (\alpha,&\ 0,\ 0),\\
      (\alpha,&\ \beta,\ 0),\\
      (\alpha,&\ \beta,\ \alpha),\\
      (\alpha,&\ \beta,\ \beta),\\
      (\alpha,&\ \beta,\ \alpha + \beta),\\
      (\alpha,&\ \beta,\ \gamma).
    \end{align*}
    Seen like this, it's clear that none of these solution set
    \say{templates} could be equal to any of the others. In each
    case, the \say{range of motion} available to each parameter
    is different—in some cases fixed to $0$, in others determined
    by one or more other parameters, and in others the parameter
    is able to vary freely. As such, if $RX = 0$ and $R'X = 0$
    have the same solutions, $R = R'$.
\end{enumerate}

\end{document}
