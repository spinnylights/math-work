\documentclass[12pt]{article}
\usepackage{mathtools}
\usepackage{amsthm}
\usepackage{amsfonts}
\usepackage{amssymb}
\usepackage{fontspec}
\usepackage{xfrac}
\usepackage{array}
\usepackage{siunitx}
\usepackage{gensymb}
\usepackage{enumitem}
\usepackage{dirtytalk}
\title{Matrix multiplication (exercises)}
\author{Zoë Sparks}

\begin{document}

\theoremstyle{definition}

\sisetup{quotient-mode=fraction}
\newtheorem{thm}{Theorem}
\newtheorem*{nthm}{Theorem}
\newtheorem{sthm}{}[thm]
\newtheorem{lemma}{Lemma}[thm]
\newtheorem{cor}{Corollary}[thm]
\newtheorem*{prop}{Property}
\newtheorem*{defn}{Definition}
\newtheorem*{comm}{Comment}
\newtheorem*{exm}{Example}

\maketitle

\begin{enumerate}
  \item
    We have
    \begin{align*}
      A =
      \begin{bmatrix}
        2 & -1 & 1\\
        1 &  2 & 1\\
      \end{bmatrix},\
      B =
      \begin{bmatrix}
         3\\
         1\\
        -1\\
      \end{bmatrix},\
      C =
      \begin{bmatrix}
        1 & -1
      \end{bmatrix}.
    \end{align*}
    We would like to know the value of $ABC$ and $CAB$.
    \begin{align*}
      AB =
      \begin{bmatrix}
        2 & -1 & 1\\
        1 &  2 & 1\\
      \end{bmatrix}
      \begin{bmatrix}
         3\\
         1\\
        -1\\
      \end{bmatrix}
      =
      \begin{bmatrix}
        2(3) - 1(1) - 1(1)\\
        1(3) + 2(1) - 1(1)\\
      \end{bmatrix}
      =
      \begin{bmatrix}
        4\\
        4\\
      \end{bmatrix};
    \end{align*}
    \begin{align*}
      ABC =
      \begin{bmatrix}
        4\\
        4\\
      \end{bmatrix}
      \begin{bmatrix}
        1 & -1
      \end{bmatrix}
      =
      \begin{bmatrix}
        4(1) & 4(-1)\\
        4(1) & 4(-1)
      \end{bmatrix}
      =
      \begin{bmatrix}
        4 & -4\\
        4 & -4\\
      \end{bmatrix}.
    \end{align*}\\
    \begin{align*}
      CA =&
      \begin{bmatrix}
        1 & -1
      \end{bmatrix}
      \begin{bmatrix}
        2 & -1 & 1\\
        1 &  2 & 1\\
      \end{bmatrix}\\
      =&
      \begin{bmatrix}
        1(2) - 1(1) & 1(-1) - 1(2) & 1(1) - 1(1)
      \end{bmatrix}\\
      =&
      \begin{bmatrix}
        1 & -3 & 0
      \end{bmatrix};
    \end{align*}
    \begin{align*}
      CAB =
      \begin{bmatrix}
        1 & -3 & 0
      \end{bmatrix}
      \begin{bmatrix}
         3\\
         1\\
        -1\\
      \end{bmatrix}
      =
      \begin{bmatrix}
        1(3) - 3(1) + 0(-1)
      \end{bmatrix}
      =
      \begin{bmatrix}
        0
      \end{bmatrix}.
    \end{align*}

  \item
    We have
    \begin{align*}
      A =
      \begin{bmatrix}
        1 & -1 & 1\\
        2 &  0 & 1\\
        3 &  0 & 1\\
      \end{bmatrix},\
      B =
      \begin{bmatrix}
        2 & -2\\
        1 &  3\\
        4 &  4\\
      \end{bmatrix}.
    \end{align*}
    We would like to verify that $A(AB) = A^{2}B$.
    \begin{align*}
      AB =&
      \begin{bmatrix}
        1 & -1 & 1\\
        2 &  0 & 1\\
        3 &  0 & 1\\
      \end{bmatrix}
      \begin{bmatrix}
        2 & -2\\
        1 &  3\\
        4 &  4\\
      \end{bmatrix}\\
      =&
      \begin{bmatrix}
        1(2) - 1(1) + 1(4) & 1(-2) - 1(3) + 1(4)\\
        2(2) + 0(1) + 1(4) & 2(-2) + 0(3) + 1(4)\\
        3(2) + 0(1) + 1(4) & 3(-2) + 0(3) + 1(4)\\
      \end{bmatrix}\\
      =&
      \begin{bmatrix}
        5 & -1\\
        8 & 0\\
        10 & -2\\
      \end{bmatrix};\\
      A(AB) =&
      \begin{bmatrix}
        1 & -1 & 1\\
        2 &  0 & 1\\
        3 &  0 & 1\\
      \end{bmatrix}
      \begin{bmatrix}
        5 & -1\\
        8 & 0\\
        10 & -2\\
      \end{bmatrix}\\
      =&
      \begin{bmatrix}
        1(5) - 1(8) + 1(10) & 1(-1) - 1(0) + 1(-2)\\
        2(5) + 0(8) + 1(10) & 2(-1) + 0(0) + 1(-2)\\
        3(5) + 0(8) + 1(10) & 3(-1) + 0(0) + 1(-2)\\
      \end{bmatrix}\\
      =&
      \begin{bmatrix}
        7  & -3\\
        20 & -4\\
        25 & -5\\
      \end{bmatrix}.
    \end{align*}
    \begin{align*}
      A^{2} =&
      \begin{bmatrix}
        1 & -1 & 1\\
        2 &  0 & 1\\
        3 &  0 & 1\\
      \end{bmatrix}
      \begin{bmatrix}
        1 & -1 & 1\\
        2 &  0 & 1\\
        3 &  0 & 1\\
      \end{bmatrix}\\
      =&
      \begin{bmatrix}
        1(1) - 1(2) + 1(3)
          & 1(-1) - 1(0) + 1(0)
          & 1(1) - 1(1) + 1(1)\\
        2(1) + 0(2) + 1(3)
          & 2(-1) + 0(0) + 1(0)
          & 2(1)  + 0(1) + 1(1)\\
        3(1) + 0(2) + 1(3)
          & 3(-1) + 0(0) + 1(0)
          & 3(1)  + 0(1) + 1(1)\\
      \end{bmatrix}\\
      =&
      \begin{bmatrix}
        2 & -1 & 1\\
        5 & -2 & 3\\
        6 & -3 & 4\\
      \end{bmatrix};\\
      A^{2}B =&
      \begin{bmatrix}
        2 & -1 & 1\\
        5 & -2 & 3\\
        6 & -3 & 4\\
      \end{bmatrix}
      \begin{bmatrix}
        2 & -2\\
        1 &  3\\
        4 &  4\\
      \end{bmatrix}\\
      =&
      \begin{bmatrix}
        2(2) - 1(1) + 1(4) & 2(-2) - 1(3) + 1(4)\\
        5(2) - 2(1) + 3(4) & 5(-2) - 2(3) + 3(4)\\
        6(2) - 3(1) + 4(4) & 6(-2) - 3(3) + 4(4)\\
      \end{bmatrix}\\
      =&
      \begin{bmatrix}
        7  & -3\\
        20 & -4\\
        25 & -5\\
      \end{bmatrix}.
    \end{align*}
    Therefore $A(AB) = A^{2}B$.

  \item
    We would like to find two different $2 \times 2$ matrices $A$
    such that $A^{2} = 0$ but $A \neq 0$.
    \begin{align*}
      \begin{bmatrix}
        1  & 1\\
        -1 & -1\\
      \end{bmatrix}
      \begin{bmatrix}
        1  & 1\\
        -1 & -1\\
      \end{bmatrix}
      =
      \begin{bmatrix}
        1(1) + 1(-1) & 1(1) + 1(-1)\\
        -1(1) - 1(-1) & -1(1) - 1(-1)\\
      \end{bmatrix}
      =
      \begin{bmatrix}
        0 & 0\\
        0 & 0\\
      \end{bmatrix}.
    \end{align*}
    \begin{align*}
      \begin{bmatrix}
        -1  & -1\\
        1 & 1\\
      \end{bmatrix}
      \begin{bmatrix}
        -1  & -1\\
        1 & 1\\
      \end{bmatrix}
      =
      \begin{bmatrix}
        -1(-1) - 1(1) & -1(-1) - 1(1)\\
        1(-1) + 1(1) & 1(-1) + 1(1)\\
      \end{bmatrix}
      =
      \begin{bmatrix}
        0 & 0\\
        0 & 0\\
      \end{bmatrix}.
    \end{align*}
    Therefore
    \begin{align*}
      A =
      \begin{bmatrix}
        1  & 1\\
        -1 & -1\\
      \end{bmatrix}
    \end{align*}
    or
    \begin{align*}
      A =
      \begin{bmatrix}
        -1  & -1\\
        1 & 1\\
      \end{bmatrix}
    \end{align*}
    both work.

  \item
    We would like to find elementary matrices $E_1, E_2, \ldots,
    E_k$ for the matrix $A$ in exercise 2 such that
    \begin{align*}
      E_k \cdots E_2E_1A = I.
    \end{align*}
    \begin{align*}
      E_1 =
      \begin{bmatrix}
        1 &  0 & 0\\
        1 &  1 & 0\\
        0 &  0 & 1\\
      \end{bmatrix}
    \end{align*}
    \begin{align*}
      E_1A =
      \begin{bmatrix}
        1 &  0 & 0\\
        1 &  1 & 0\\
        0 &  0 & 1\\
      \end{bmatrix}
      \begin{bmatrix}
        1 & -1 & 1\\
        2 &  0 & 1\\
        3 &  0 & 1\\
      \end{bmatrix}
      =
      \begin{bmatrix}
        1 & -1 & 1\\
        3 & -1 & 2\\
        3 &  0 & 1\\
      \end{bmatrix}
    \end{align*}
    \begin{align*}
      E_2 =
      \begin{bmatrix}
        1 & -1 & 0\\
        0 &  1 & 0\\
        0 &  0 & 1\\
      \end{bmatrix}
    \end{align*}
    \begin{align*}
      E_2E_1A =
      \begin{bmatrix}
        1 & -1 & 0\\
        0 &  1 & 0\\
        0 &  0 & 1\\
      \end{bmatrix}
      \begin{bmatrix}
        1 & -1 & 1\\
        3 & -1 & 2\\
        3 &  0 & 1\\
      \end{bmatrix}
      =
      \begin{bmatrix}
        -2 &  0 & -1\\
         3 & -1 & 2\\
         3 &  0 & 1\\
      \end{bmatrix}
    \end{align*}
    \begin{align*}
      E_3 =
      \begin{bmatrix}
        1 &  0 & 1\\
        0 &  1 & 0\\
        0 &  0 & 1\\
      \end{bmatrix}
    \end{align*}
    \begin{align*}
      E_3E_2E_1A =
      \begin{bmatrix}
        1 &  0 & 1\\
        0 &  1 & 0\\
        0 &  0 & 1\\
      \end{bmatrix}
      \begin{bmatrix}
        -2 &  0 & -1\\
         3 & -1 & 2\\
         3 &  0 & 1\\
      \end{bmatrix}
      =
      \begin{bmatrix}
        1 &  0 & 0\\
        3 & -1 & 2\\
        3 &  0 & 1\\
      \end{bmatrix}
    \end{align*}
    \begin{align*}
      E_4 =
      \begin{bmatrix}
         1 &  0 & 0\\
        -3 &  1 & 0\\
         0 &  0 & 1\\
      \end{bmatrix}
    \end{align*}
    \begin{align*}
      E_4E_3E_2E_1A =
      \begin{bmatrix}
         1 &  0 & 0\\
        -3 &  1 & 0\\
         0 &  0 & 1\\
      \end{bmatrix}
      \begin{bmatrix}
        1 &  0 & 0\\
        3 & -1 & 2\\
        3 &  0 & 1\\
      \end{bmatrix}
      =
      \begin{bmatrix}
        1 &  0 & 0\\
        0 & -1 & 2\\
        3 &  0 & 1\\
      \end{bmatrix}
    \end{align*}
    \begin{align*}
      E_5 =
      \begin{bmatrix}
         1 &  0 & 0\\
         0 &  1 & 0\\
        -3 &  0 & 1\\
      \end{bmatrix}
    \end{align*}
    \begin{align*}
      E_5E_4E_3E_2E_1A =
      \begin{bmatrix}
         1 &  0 & 0\\
         0 &  1 & 0\\
        -3 &  0 & 1\\
      \end{bmatrix}
      \begin{bmatrix}
        1 &  0 & 0\\
        3 & -1 & 2\\
        3 &  0 & 1\\
      \end{bmatrix}
      =
      \begin{bmatrix}
        1 &  0 & 0\\
        0 & -1 & 2\\
        0 &  0 & 1\\
      \end{bmatrix}
    \end{align*}
    \begin{align*}
      E_6 =
      \begin{bmatrix}
        1 & 0 & 0\\
        0 & 1 & -2\\
        0 & 0 & 1\\
      \end{bmatrix}
    \end{align*}
    \begin{align*}
      E_6E_5E_4E_3E_2E_1A =
      \begin{bmatrix}
        1 & 0 & 0\\
        0 & 1 & -2\\
        0 & 0 & 1\\
      \end{bmatrix}
      \begin{bmatrix}
        1 &  0 & 0\\
        0 & -1 & 2\\
        0 &  0 & 1\\
      \end{bmatrix}
      =
      \begin{bmatrix}
        1 &  0 & 0\\
        0 & -1 & 0\\
        0 &  0 & 1\\
      \end{bmatrix}
    \end{align*}
    \begin{align*}
      E_7 =
      \begin{bmatrix}
        1 &  0 & 0\\
        0 & -1 & 0\\
        0 &  0 & 1\\
      \end{bmatrix}
    \end{align*}
    \begin{align*}
      E_7E_6E_5E_4E_3E_2E_1A =
      \begin{bmatrix}
        1 &  0 & 0\\
        0 & -1 & 0\\
        0 &  0 & 1\\
      \end{bmatrix}
      \begin{bmatrix}
        1 &  0 & 0\\
        0 & -1 & 0\\
        0 &  0 & 1\\
      \end{bmatrix}
      =
      \begin{bmatrix}
        1 & 0 & 0\\
        0 & 1 & 0\\
        0 & 0 & 1\\
      \end{bmatrix}.
    \end{align*}
    \begin{align*}
      E_7E_6E_5E_4E_3E_2E_1 =\\
      \begin{bmatrix}
        1 &  0 & 0\\
        0 & -1 & 0\\
        0 &  0 & 1\\
      \end{bmatrix}
      \begin{bmatrix}
        1 & 0 & 0\\
        0 & 1 & -2\\
        0 & 0 & 1\\
      \end{bmatrix}
      \begin{bmatrix}
         1 &  0 & 0\\
         0 &  1 & 0\\
        -3 &  0 & 1\\
      \end{bmatrix}
      \begin{bmatrix}
         1 &  0 & 0\\
        -3 &  1 & 0\\
         0 &  0 & 1\\
      \end{bmatrix}
      \begin{bmatrix}
        1 &  0 & 1\\
        0 &  1 & 0\\
        0 &  0 & 1\\
      \end{bmatrix}
      \begin{bmatrix}
        1 & -1 & 0\\
        0 &  1 & 0\\
        0 &  0 & 1\\
      \end{bmatrix}
      \begin{bmatrix}
        1 &  0 & 0\\
        1 &  1 & 0\\
        0 &  0 & 1\\
      \end{bmatrix} =\\
      \begin{bmatrix}
        1 &  0 & 0\\
        0 & -1 & 2\\
        0 &  0 & 1\\
      \end{bmatrix}
      \begin{bmatrix}
         1 &  0 & 0\\
         0 &  1 & 0\\
        -3 &  0 & 1\\
      \end{bmatrix}
      \begin{bmatrix}
         1 &  0 & 0\\
        -3 &  1 & 0\\
         0 &  0 & 1\\
      \end{bmatrix}
      \begin{bmatrix}
        1 &  0 & 1\\
        0 &  1 & 0\\
        0 &  0 & 1\\
      \end{bmatrix}
      \begin{bmatrix}
        1 & -1 & 0\\
        0 &  1 & 0\\
        0 &  0 & 1\\
      \end{bmatrix}
      \begin{bmatrix}
        1 &  0 & 0\\
        1 &  1 & 0\\
        0 &  0 & 1\\
      \end{bmatrix} =\\
      \begin{bmatrix}
         1 &  0 & 0\\
        -6 & -1 & 2\\
        -3 &  0 & 1\\
      \end{bmatrix}
      \begin{bmatrix}
         1 &  0 & 0\\
        -3 &  1 & 0\\
         0 &  0 & 1\\
      \end{bmatrix}
      \begin{bmatrix}
        1 &  0 & 1\\
        0 &  1 & 0\\
        0 &  0 & 1\\
      \end{bmatrix}
      \begin{bmatrix}
        1 & -1 & 0\\
        0 &  1 & 0\\
        0 &  0 & 1\\
      \end{bmatrix}
      \begin{bmatrix}
        1 &  0 & 0\\
        1 &  1 & 0\\
        0 &  0 & 1\\
      \end{bmatrix} =\\
      \begin{bmatrix}
         1 &  0 & 0\\
        -3 &  1 & 2\\
        -3 &  0 & 1\\
      \end{bmatrix}
      \begin{bmatrix}
        1 &  0 & 1\\
        0 &  1 & 0\\
        0 &  0 & 1\\
      \end{bmatrix}
      \begin{bmatrix}
        1 & -1 & 0\\
        0 &  1 & 0\\
        0 &  0 & 1\\
      \end{bmatrix}
      \begin{bmatrix}
        1 &  0 & 0\\
        1 &  1 & 0\\
        0 &  0 & 1\\
      \end{bmatrix} =\\
      \begin{bmatrix}
         1 &  0 &  1\\
        -3 & -1 & -1\\
        -3 &  0 & -2\\
      \end{bmatrix}
      \begin{bmatrix}
        1 & -1 & 0\\
        0 &  1 & 0\\
        0 &  0 & 1\\
      \end{bmatrix}
      \begin{bmatrix}
        1 &  0 & 0\\
        1 &  1 & 0\\
        0 &  0 & 1\\
      \end{bmatrix} =\\
      \begin{bmatrix}
         1 & -1 &  1\\
        -3 &  2 & -1\\
        -3 &  3 & -2\\
      \end{bmatrix}
      \begin{bmatrix}
        1 &  0 & 0\\
        1 &  1 & 0\\
        0 &  0 & 1\\
      \end{bmatrix} =\\
      \begin{bmatrix}
         0 & -1 &  1\\
        -1 &  2 & -1\\
         0 &  3 & -2\\
      \end{bmatrix}.
    \end{align*}
    \begin{align*}
      \begin{bmatrix}
         0 & -1 &  1\\
        -1 &  2 & -1\\
         0 &  3 & -2\\
      \end{bmatrix}
      \begin{bmatrix}
        1 & -1 & 1\\
        2 &  0 & 1\\
        3 &  0 & 1\\
      \end{bmatrix}
      =
      \begin{bmatrix}
        1 & 0 & 0\\
        0 & 1 & 0\\
        0 & 0 & 1\\
      \end{bmatrix}.
    \end{align*}

  \item
    We have
    \begin{align*}
      A =
      \begin{bmatrix}
        1 & -1\\
        2 &  2\\
        1 &  0\\
      \end{bmatrix},\
      B =
      \begin{bmatrix}
        3  & 1\\
        -4 & 4\\
      \end{bmatrix}.
    \end{align*}
    We would like to know if there is a matrix $C$ such that $CA
    = B$.

    If such a matrix exists, it would be a $2 \times 3$ matrix,
    since $A$ is $3 \times 2$ and $B$ is $2 \times 2$. Of note,
    there cannot be a matrix $C'$ such that $AC' = B$, because
    such a matrix would inevitably have $3$ rows.

    What if we say
    \begin{align*}
      C =
      \begin{bmatrix}
        a_1 & b_1 & c_1\\
        a_2 & b_2 & c_2\\
      \end{bmatrix}
    \end{align*}
    and multiply?
    \begin{align*}
      \begin{bmatrix}
        a_1 & b_1 & c_1\\
        a_2 & b_2 & c_2\\
      \end{bmatrix}
      \begin{bmatrix}
        1 & -1\\
        2 &  2\\
        1 &  0\\
      \end{bmatrix}
      =
      \begin{bmatrix}
        a_1 + 2b_1 + c_1 & -a_1 + 2b_1\\
        a_2 + 2b_2 + c_2 & -a_2 + 2b_2\\
      \end{bmatrix}.
    \end{align*}
    So what we're looking for are scalars such that
    \begin{align*}
      a_1 + 2b_1 + c_1 =& 3\\
      -a_1 + 2b_1 =& 1
    \end{align*}
    and
    \begin{align*}
      a_2 + 2b_2 + c_2 =& -4\\
      -a_2 + 2b_2 =& 4.
    \end{align*}
    We proceed:
    \begin{align*}
      \begin{bmatrix}
        1  & 2 & 1 & 3\\
        -1 & 2 & 0 & 1
      \end{bmatrix}
      \xrightarrow{(2)}
      \begin{bmatrix}
        1 & 2 & 1 & 3\\
        0 & 4 & 1 & 4
      \end{bmatrix}
      \xrightarrow{(1)}
      \begin{bmatrix}
        1 & 2 & 1           & 3\\
        0 & 1 & \frac{1}{4} & 1
      \end{bmatrix}
      \xrightarrow{(2)}
    \end{align*}
    \begin{align*}
      \begin{bmatrix}
        1 & 0 & \frac{1}{2} & 1\\
        0 & 1 & \frac{1}{4} & 1
      \end{bmatrix}.
    \end{align*}
    So in the first case, we have
    \begin{align*}
      a_1 =& -\frac{1}{2}c_1 + 1\\
      b_1 =& -\frac{1}{4}c_1 + 1.
    \end{align*}
    In the second case,
    \begin{align*}
      \begin{bmatrix}
        -4\\
        4
      \end{bmatrix}
      \xrightarrow{(2)}
      \begin{bmatrix}
        -4\\
        0
      \end{bmatrix}
    \end{align*}
    (the other operations would be redundant). So we then have
    \begin{align*}
      a_2 =& -\frac{1}{2}c_2 - 4\\
      b_2 =& -\frac{1}{4}c_2.
    \end{align*}
    Of course, this immediately makes me wonder what would happen
    if we took $(c_1,c_2) = (0,0)$.
    \begin{align*}
      \begin{bmatrix}
        1  & 1 & 0\\
        -4 & 0 & 0
      \end{bmatrix}
      \begin{bmatrix}
        1 & -1\\
        2 &  2\\
        1 &  0\\
      \end{bmatrix}
      =
      \begin{bmatrix}
        3  & 1\\
        -4 & 4
      \end{bmatrix}.
    \end{align*}
    It works! Of course, I bet would also work if we took, say,
    $(c_1,c_2) = (4,4)$.
    \begin{align*}
      \begin{bmatrix}
        -1 &  0 & 4\\
        -6 & -1 & 4
      \end{bmatrix}
      \begin{bmatrix}
        1 & -1\\
        2 &  2\\
        1 &  0\\
      \end{bmatrix}
      =
      \begin{bmatrix}
        3  & 1\\
        -4 & 4
      \end{bmatrix}.
    \end{align*}
    Yep! After all, our calculations above indicate there is an
    infinite number of matrices $C$ such that $CA = B$. This is
    unsurprising by theorem 6.

    The implications of this are rather interesting. Perhaps the
    fact that $A$ is $3 \times 2$ and $B$ is $2 \times 2$
    guarantees that there will be an infinite number of matrices
    $C$ such that $CA = B$, as long as $A$ doesn't have an all-0
    row or column? Even if it does, it might still work—probably
    you can figure by acting as if $A$ didn't have the all-0 row
    or column at all and mulling it over from there.

    After all, in any field, for scalars $x$ and $y$, there is
    some scalar $a$ such that $ax = y$ as long as $x \neq 0$. So,
    I think this implies that if $A$ is an $m \times n$ matrix
    where $m$ is the number of non-all-0 rows, $n$ is the number
    of non-all-0 columns, and $m > n$, there is an infinite
    number of $o \times m$ matrices $C$ such that $CA = B$ for
    any given $o \times n$ matrix $B$. If $m \leq n$, I think
    there will be one such matrix $C$. Of course, this is just
    thinking off the cuff—I would need to prove these statements
    to be sure.

  \item
    Given that $A$ is an $m \times n$ matrix and $B$ is an $n
    \times k$ matrix, we would like to show that the columns of
    $C = AB$ are linear combinations of the columns of $A$.

    Of note, if $\alpha_1,\ldots,\alpha_n$ are the columns of
    $A$ and $\gamma_1,\ldots,\gamma_n$ are the columns of $C$,
    then
    \begin{align*}
      \gamma_i = \sum_{r = 1}^{n} B_{rj}\alpha_r.
    \end{align*}

    A linear combination of the columns of $A$ would be an
    expression $x_1\alpha_1 + \ldots + x_n\alpha_n$ for scalars
    $x_1,\ldots,x_n$, or
    \begin{align*}
      \sum_{r = 1}^{n} x_r\alpha_r.
    \end{align*}
    If we say that $B_{rj} = x_r$ for all $B_{ij}$ and $x_i$, $1
    \leq i \leq n$, clearly the columns of $C = AB$ are linear
    combinations of the columns of $A$.

  \item
    Given that $A$ and $B$ are $2 \times 2$ matrices such that
    $AB = I$, we would like to prove that $BA = I$ as well.

    We know that
    \begin{align*}
      \begin{bmatrix}
        A_{11} & A_{12}\\
        A_{21} & A_{22}\\
      \end{bmatrix}
      \begin{bmatrix}
        B_{11}\\
        B_{21}\\
      \end{bmatrix}
      =
      \begin{bmatrix}
        1\\
        0\\
      \end{bmatrix}.
    \end{align*}
    So,
    \begin{align*}
      \begin{bmatrix}
        A_{11} & A_{12} & 1\\
        A_{21} & A_{22} & 0\\
      \end{bmatrix}
      \xrightarrow{(1)}
      \begin{bmatrix}
        1 & \frac{A_{12}}{A_{11}} & \frac{1}{A_{11}}\\
        A_{21} & A_{22} & 0\\
      \end{bmatrix}
      \xrightarrow{(2)}
    \end{align*}
    \begin{align*}
      \begin{bmatrix}
        1 & \frac{A_{12}}{A_{11}} & \frac{1}{A_{11}}\\
        0 & A_{22} - \frac{A_{21}A_{12}}{A_{11}} & -\frac{A_{21}}{A_{11}}\\
      \end{bmatrix}
      \xrightarrow{(1)}
      \begin{bmatrix}
        1 & \frac{A_{12}}{A_{11}} & \frac{1}{A_{11}}\\
        0 & 1 & -\frac{A_{21}}{A_{22}A_{11} - A_{21}A_{12}}\\
      \end{bmatrix}
      \xrightarrow{(2)}
    \end{align*}
    \begin{align*}
      \begin{bmatrix}
        1 &
          0 &
          \frac{A_{22}}{A_{22}A_{11} - A_{21}A_{12}}\\
        0 &
          1 &
          -\frac{A_{21}}{A_{22}A_{11} - A_{21}A_{12}}\\
      \end{bmatrix}.
    \end{align*}
    \begin{align*}
      \begin{bmatrix}
        A_{11} & A_{12}\\
        A_{21} & A_{22}\\
      \end{bmatrix}
      \begin{bmatrix}
        \frac{A_{22}}{A_{22}A_{11} - A_{21}A_{12}}\\
        -\frac{A_{21}}{A_{22}A_{11} - A_{21}A_{12}}\\
      \end{bmatrix}
      =
      \begin{bmatrix}
        \frac{A_{22}A_{11} - A_{21}A_{12}}
             {A_{22}A_{11} - A_{21}A_{12}}\\
        \frac{A_{22}A_{21} - A_{22}A_{21}}
             {A_{22}A_{11} - A_{21}A_{12}}\\
      \end{bmatrix}
      =
      \begin{bmatrix}
        1\\
        0\\
      \end{bmatrix}.
    \end{align*}
    Likewise,
    \begin{align*}
      \begin{bmatrix}
        A_{11} & A_{12}\\
        A_{21} & A_{22}\\
      \end{bmatrix}
      \begin{bmatrix}
        B_{21}\\
        B_{22}\\
      \end{bmatrix}
      =
      \begin{bmatrix}
        0\\
        1\\
      \end{bmatrix},
    \end{align*}
    so
    \begin{align*}
      \begin{bmatrix}
        A_{11} & A_{12} & 0\\
        A_{21} & A_{22} & 1\\
      \end{bmatrix}
      \xrightarrow{(1)}
      \begin{bmatrix}
        1 & \frac{A_{12}}{A_{11}} & 0\\
        A_{21} & A_{22} & 1\\
      \end{bmatrix}
      \xrightarrow{(2)}
    \end{align*}
    \begin{align*}
      \begin{bmatrix}
        1 & \frac{A_{12}}{A_{11}} & 0\\
        0 & \frac{A_{22}A_{11} - A_{21}A_{12}}{A_{11}} & 1\\
      \end{bmatrix}
      \xrightarrow{(1)}
      \begin{bmatrix}
        1 & \frac{A_{12}}{A_{11}} & 0\\
        0 & 1 & \frac{A_{11}}{A_{22}A_{11} - A_{21}A_{12}}\\
      \end{bmatrix}
      \xrightarrow{(2)}
    \end{align*}
    \begin{align*}
      \begin{bmatrix}
        1 &
          0 &
          -\frac{A_{12}}{A_{22}A_{11} - A_{21}A_{12}}\\
        0 &
          1 &
          \frac{A_{11}}{A_{22}A_{11} - A_{21}A_{12}}\\
      \end{bmatrix}.
    \end{align*}
    \begin{align*}
      \begin{bmatrix}
        A_{11} & A_{12}\\
        A_{21} & A_{22}\\
      \end{bmatrix}
      \begin{bmatrix}
        -\frac{A_{12}}{A_{22}A_{11} - A_{21}A_{12}}\\
        \frac{A_{11}}{A_{22}A_{11} - A_{21}A_{12}}\\
      \end{bmatrix}
      =
      \begin{bmatrix}
        \frac{A_{11}A_{12} - A_{11}A_{12}}
             {A_{22}A_{11} - A_{21}A_{12}}\\
        \frac{A_{22}A_{11} - A_{21}A_{12}}
             {A_{22}A_{11} - A_{21}A_{12}}\\
      \end{bmatrix}
      =
      \begin{bmatrix}
        0\\
        1\\
      \end{bmatrix}.
    \end{align*}
    Therefore,
    \begin{align*}
      B =
      \begin{bmatrix}
        \frac{A_{22}}{A_{22}A_{11} - A_{21}A_{12}} &
          -\frac{A_{12}}{A_{22}A_{11} - A_{21}A_{12}}\\
        -\frac{A_{21}}{A_{22}A_{11} - A_{21}A_{12}} &
          \frac{A_{11}}{A_{22}A_{11} - A_{21}A_{12}}\\
      \end{bmatrix}.
    \end{align*}
    \begin{align*}
      \begin{bmatrix}
        \frac{A_{22}}{A_{22}A_{11} - A_{21}A_{12}} &
          -\frac{A_{12}}{A_{22}A_{11} - A_{21}A_{12}}\\
        -\frac{A_{21}}{A_{22}A_{11} - A_{21}A_{12}} &
          \frac{A_{11}}{A_{22}A_{11} - A_{21}A_{12}}\\
      \end{bmatrix}
      \begin{bmatrix}
        A_{11} & A_{12}\\
        A_{21} & A_{22}\\
      \end{bmatrix}
      =&\\
      \begin{bmatrix}
        \frac{A_{22}A_{11} - A_{21}A_{12}}
             {A_{22}A_{11} - A_{21}A_{12}} &
          \frac{A_{22}A_{12} - A_{22}A_{12}}
               {A_{22}A_{11} - A_{21}A_{12}}\\
        \frac{A_{21}A_{11} - A_{21}A_{11}}
             {A_{22}A_{11} - A_{21}A_{12}} &
          \frac{A_{22}A_{11} - A_{21}A_{12}}
               {A_{22}A_{11} - A_{21}A_{12}}\\
      \end{bmatrix}
      =&\\
      \begin{bmatrix}
        1 & 0\\
        0 & 1\\
      \end{bmatrix},
    \end{align*}
    so $BA = I$.

    \begin{comm}
      We have found here that if $A$ is a $2 \times 2$ matrix and
      $A_{22}A_{11} - A_{21}A_{12} \neq 0$,
      \begin{align*}
        A^{-1} =
        \begin{bmatrix}
          \frac{A_{22}}{A_{22}A_{11} - A_{21}A_{12}} &
            -\frac{A_{12}}{A_{22}A_{11} - A_{21}A_{12}}\\
          -\frac{A_{21}}{A_{22}A_{11} - A_{21}A_{12}} &
            \frac{A_{11}}{A_{22}A_{11} - A_{21}A_{12}}\\
        \end{bmatrix}.
      \end{align*}
    \end{comm}
\end{enumerate}

\end{document}
