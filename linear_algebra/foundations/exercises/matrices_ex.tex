\documentclass[12pt]{article}
\usepackage{mathtools}
\usepackage{amsthm}
\usepackage{amsfonts}
\usepackage{amssymb}
\usepackage{fontspec}
\usepackage{xfrac}
\usepackage{array}
\usepackage{siunitx}
\usepackage{gensymb}
\usepackage{enumitem}
\title{Matrices (exercises)}
\author{Zoë Sparks}

\begin{document}

\theoremstyle{definition}

\sisetup{quotient-mode=fraction}
\newtheorem{thm}{Theorem}
\newtheorem*{nthm}{Theorem}
\newtheorem{sthm}{}[thm]
\newtheorem{lemma}{Lemma}[thm]
\newtheorem{cor}{Corollary}[thm]
\newtheorem*{prop}{Property}
\newtheorem*{defn}{Definition}
\newtheorem*{comm}{Comment}
\newtheorem*{exm}{Example}

\maketitle

\begin{enumerate}
    \item
      We have
      \begin{align*}
        (1 - i)x_1 - ix_2 = &\ 0\\
        2x_1 - (1 - i)x_2 = &\ 0.
      \end{align*}
      We would like to find the solutions. We know by theorem 3
      that every $m \times n$ matrix over a field $F$ is
      row-equivalent to a row-reduced matrix, and by theorem 2
      that if $A$ and $B$ are $m \times n$ row-equivalent
      matrices, $AX = 0$ and $BX = 0$ have exactly the same
      solutions.  Therefore, this system of equations has the
      same set of solutions as
      \begin{align*}
        x_1 = &\ 0\\
        x_2 = &\ 0,
      \end{align*}
      meaning the only solution is $x_1 = x_2 = 0$.
\end{document}
