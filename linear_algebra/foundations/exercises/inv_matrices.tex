\documentclass[12pt]{article}
\usepackage{mathtools}
\usepackage{amsthm}
\usepackage{amsfonts}
\usepackage{amssymb}
\usepackage{fontspec}
\usepackage{xfrac}
\usepackage{array}
\usepackage{siunitx}
\usepackage{gensymb}
\usepackage{enumitem}
\usepackage{dirtytalk}
\usepackage{bm}
\title{Invertible matrices (exercises)}
\author{Zoë Sparks}

\begin{document}

\theoremstyle{definition}

\sisetup{quotient-mode=fraction}
\newtheorem{thm}{Theorem}
\newtheorem*{nthm}{Theorem}
\newtheorem{sthm}{}[thm]
\newtheorem{lemma}{Lemma}[thm]
\newtheorem*{nlemma}{Lemma}
\newtheorem{cor}{Corollary}[thm]
\newtheorem*{prop}{Property}
\newtheorem*{defn}{Definition}
\newtheorem*{comm}{Comment}
\newtheorem*{exm}{Example}

\maketitle

\begin{enumerate}
  \item We have
    \begin{align*}
      A =
      \begin{bmatrix}
        1  & 2  & 1 & 0\\
        -1 & 0  & 3 & 5\\
        1  & -2 & 1 & 1\\
      \end{bmatrix}.
    \end{align*}
    We would like to find a row-reduced echelon matrix $R$ which
    is row-equivalent to $A$ and an invertible $3 \times 3$
    matrix $P$ such that $R = PA$.
    \begin{align*}
      \begin{bmatrix}
        1  & 2  & 1 & 0\\
        -1 & 0  & 3 & 5\\
        1  & -2 & 1 & 1\\
      \end{bmatrix},\
      \begin{bmatrix}
        1 & 0 & 0\\
        0 & 1 & 0\\
        0 & 0 & 1\\
      \end{bmatrix}
    \end{align*}
    \begin{align*}
      \begin{bmatrix}
        1  & 2  & 1 & 0\\
        0  & 2  & 4 & 5\\
        1  & -2 & 1 & 1\\
      \end{bmatrix},\
      \begin{bmatrix}
        1 & 0 & 0\\
        1 & 1 & 0\\
        0 & 0 & 1\\
      \end{bmatrix}
    \end{align*}
    \begin{align*}
      \begin{bmatrix}
        1  & 2  & 1 & 0\\
        0  & 2  & 4 & 5\\
        0  & -4 & 0 & 1\\
      \end{bmatrix},\
      \begin{bmatrix}
        1  & 0 & 0\\
        1  & 1 & 0\\
        -1 & 0 & 1\\
      \end{bmatrix}
    \end{align*}
    \begin{align*}
      \begin{bmatrix}
        1  & 0  & -3 & -5\\
        0  & 2  & 4  & 5\\
        0  & -4 & 0  & 1\\
      \end{bmatrix},\
      \begin{bmatrix}
        0  & -1 & 0\\
        1  & 1  & 0\\
        -1 & 0  & 1\\
      \end{bmatrix}
    \end{align*}
    \begin{align*}
      \begin{bmatrix}
        1  & 0  & -3 & -5\\
        0  & 2  & 4  & 5\\
        0  & 0  & 8  & 11\\
      \end{bmatrix},\
      \begin{bmatrix}
        0  & -1 & 0\\
        1  & 1  & 0\\
        1  & 2  & 1\\
      \end{bmatrix}
    \end{align*}
    \begin{align*}
      \begin{bmatrix}
        1  & 0  & -3 & -5\\
        0  & 1  & 2  & \frac{5}{2}\\
        0  & 0  & 8  & 11\\
      \end{bmatrix},\
      \begin{bmatrix}
        0  & -1 & 0\\
        \frac{1}{2} & \frac{1}{2} & 0\\
        1 & 2  & 1\\
      \end{bmatrix}
    \end{align*}
    \begin{align*}
      \begin{bmatrix}
        1  & 0  & -3 & -5\\
        0  & 1  & 2  & \frac{5}{2}\\
        0  & 0  & 1  & \frac{11}{8}\\
      \end{bmatrix},\
      \begin{bmatrix}
        0  & -1 & 0\\
        \frac{1}{2} & \frac{1}{2} & 0\\
        \frac{1}{8} & \frac{1}{4} & \frac{1}{8}\\
      \end{bmatrix}
    \end{align*}
    \begin{align*}
      \begin{bmatrix}
        1  & 0  & -3 & -5\\
        0  & 1  & 0  & -\frac{1}{4}\\
        0  & 0  & 1  & \frac{11}{8}\\
      \end{bmatrix},\
      \begin{bmatrix}
        0  & -1 & 0\\
        \frac{1}{4} & 0 & -\frac{1}{4}\\
        \frac{1}{8} & \frac{1}{4} & \frac{1}{8}\\
      \end{bmatrix}
    \end{align*}
    \begin{align*}
      \begin{bmatrix}
        1  & 0  & 0  & -\frac{7}{8}\\
        0  & 1  & 0  & -\frac{1}{4}\\
        0  & 0  & 1  & \frac{11}{8}\\
      \end{bmatrix},\
      \begin{bmatrix}
        \frac{3}{8}  & -\frac{1}{4} & \frac{3}{8}\\
        \frac{1}{4} & 0 & -\frac{1}{4}\\
        \frac{1}{8} & \frac{1}{4} & \frac{1}{8}\\
      \end{bmatrix}.
    \end{align*}
    \begin{align*}
      \begin{bmatrix}
        1  & 0  & 0  & -\frac{7}{8}\\
        0  & 1  & 0  & -\frac{1}{4}\\
        0  & 0  & 1  & \frac{11}{8}\\
      \end{bmatrix}
      =
      \begin{bmatrix}
        \frac{3}{8}  & -\frac{1}{4} & \frac{3}{8}\\
        \frac{1}{4} & 0 & -\frac{1}{4}\\
        \frac{1}{8} & \frac{1}{4} & \frac{1}{8}\\
      \end{bmatrix}
      \begin{bmatrix}
        1  & 2  & 1 & 0\\
        -1 & 0  & 3 & 5\\
        1  & -2 & 1 & 1\\
      \end{bmatrix}.
    \end{align*}
    \begin{align*}
      \begin{bmatrix}
        \frac{3}{8}  & -\frac{1}{4} & \frac{3}{8}\\
        \frac{1}{4} & 0 & -\frac{1}{4}\\
        \frac{1}{8} & \frac{1}{4} & \frac{1}{8}\\
      \end{bmatrix}
    \end{align*}
    is row-equivalent to $I$ as we have seen and is thus
    invertible, so
    \begin{align*}
      R =
      \begin{bmatrix}
        1  & 0  & 0  & -\frac{7}{8}\\
        0  & 1  & 0  & -\frac{1}{4}\\
        0  & 0  & 1  & \frac{11}{8}\\
      \end{bmatrix}\ \text{and}\
      P =
      \begin{bmatrix}
        \frac{3}{8}  & -\frac{1}{4} & \frac{3}{8}\\
        \frac{1}{4} & 0 & -\frac{1}{4}\\
        \frac{1}{8} & \frac{1}{4} & \frac{1}{8}\\
      \end{bmatrix}.
    \end{align*}

  \item
    We have
    \begin{align*}
      A =
      \begin{bmatrix}
        2 & 0 & i\\
        1 & -3 & -i\\
        i & 1 & 1
      \end{bmatrix},
    \end{align*}
    and we'd like to do the same thing with it that we did with
    $A$ in (1).
    \begin{align*}
      \begin{bmatrix}
        2 & 0 & i\\
        1 & -3 & -i\\
        i & 1 & 1
      \end{bmatrix},\
      \begin{bmatrix}
        1 & 0 & 0\\
        0 & 1 & 0\\
        0 & 0 & 1
      \end{bmatrix}
    \end{align*}
    \begin{align*}
      \begin{bmatrix}
        1 & 0 & \frac{1}{2}i\\
        1 & -3 & -i\\
        i & 1 & 1
      \end{bmatrix},\
      \begin{bmatrix}
        \frac{1}{2} & 0 & 0\\
        0 & 1 & 0\\
        0 & 0 & 1
      \end{bmatrix}
    \end{align*}
    \begin{align*}
      \begin{bmatrix}
        1 & 0 & \frac{1}{2}i\\
        0 & -3 & -\frac{3}{2}i\\
        i & 1 & 1
      \end{bmatrix},\
      \begin{bmatrix}
        \frac{1}{2} & 0 & 0\\
        -\frac{1}{2} & 1 & 0\\
        0 & 0 & 1
      \end{bmatrix}
    \end{align*}
    \begin{align*}
      \begin{bmatrix}
        1 & 0 & \frac{1}{2}i\\
        0 & -3 & -\frac{3}{2}i\\
        0 & 1 & \frac{3}{2}
      \end{bmatrix},\
      \begin{bmatrix}
        \frac{1}{2} & 0 & 0\\
        -\frac{1}{2} & 1 & 0\\
        -\frac{1}{2}i & 0 & 1
      \end{bmatrix}
    \end{align*}
    \begin{align*}
      \begin{bmatrix}
        1 & 0 & \frac{1}{2}i\\
        0 & 1 & \frac{1}{2}i\\
        0 & 1 & \frac{3}{2}
      \end{bmatrix},\
      \begin{bmatrix}
        \frac{1}{2} & 0 & 0\\
        \frac{1}{6} & -\frac{1}{3} & 0\\
        -\frac{1}{2}i & 0 & 1
      \end{bmatrix}
    \end{align*}
    \begin{align*}
      \begin{bmatrix}
        1 & 0 & \frac{1}{2}i\\
        0 & 1 & \frac{1}{2}i\\
        0 & 0 & \frac{3}{2}-\frac{1}{2}i
      \end{bmatrix},\
      \begin{bmatrix}
        \frac{1}{2} & 0 & 0\\
        \frac{1}{6} & -\frac{1}{3} & 0\\
        -\frac{1}{6}-\frac{1}{2}i & \frac{1}{3} & 1
      \end{bmatrix}
    \end{align*}
    \begin{align*}
      \begin{bmatrix}
        1 & 0 & \frac{1}{2}i\\
        0 & 1 & \frac{1}{2}i\\
        0 & 0 & 1
      \end{bmatrix},\
      \begin{bmatrix}
        \frac{1}{2} & 0 & 0\\
        \frac{1}{6} & -\frac{1}{3} & 0\\
        -\frac{1}{3}i & \frac{1}{5}+\frac{1}{15}i & \frac{3}{5}+\frac{1}{5}i
      \end{bmatrix}
    \end{align*}
    \begin{align*}
      \begin{bmatrix}
        1 & 0 & \frac{1}{2}i\\
        0 & 1 & 0\\
        0 & 0 & 1
      \end{bmatrix},\
      \begin{bmatrix}
        \frac{1}{2} & 0 & 0\\
        0 & -\frac{3}{10}-\frac{1}{10}i & \frac{1}{10}-\frac{3}{10}i\\
        -\frac{1}{3}i & \frac{1}{5}+\frac{1}{15}i & \frac{3}{5}+\frac{1}{5}i
      \end{bmatrix}
    \end{align*}
    \begin{align*}
      \begin{bmatrix}
        1 & 0 & 0\\
        0 & 1 & 0\\
        0 & 0 & 1
      \end{bmatrix},\
      \begin{bmatrix}
        \frac{1}{3} & \frac{1}{30}-\frac{1}{10}i & \frac{1}{10}-\frac{3}{10}i\\
        0 & -\frac{3}{10}-\frac{1}{10}i & \frac{1}{10}-\frac{3}{10}i\\
        -\frac{1}{3}i & \frac{1}{5}+\frac{1}{15}i & \frac{3}{5}+\frac{1}{5}i
      \end{bmatrix}.
    \end{align*}
    \begin{align*}
      \begin{bmatrix}
        1 & 0 & 0\\
        0 & 1 & 0\\
        0 & 0 & 1
      \end{bmatrix}=
      \begin{bmatrix}
        \frac{1}{3} & \frac{1}{30}-\frac{1}{10}i & \frac{1}{10}-\frac{3}{10}i\\
        0 & -\frac{3}{10}-\frac{1}{10}i & \frac{1}{10}-\frac{3}{10}i\\
        -\frac{1}{3}i & \frac{1}{5}+\frac{1}{15}i & \frac{3}{5}+\frac{1}{5}i
      \end{bmatrix}
      \begin{bmatrix}
        2 & 0 & i\\
        1 & -3 & -i\\
        i & 1 & 1
      \end{bmatrix},
    \end{align*}
    so
    \begin{align*}
      R =
      \begin{bmatrix}
        1 & 0 & 0\\
        0 & 1 & 0\\
        0 & 0 & 1
      \end{bmatrix}\ \text{and}\
      P =
      \begin{bmatrix}
        \frac{1}{3} & \frac{1}{30}-\frac{1}{10}i & \frac{1}{10}-\frac{3}{10}i\\
        0 & -\frac{3}{10}-\frac{1}{10}i & \frac{1}{10}-\frac{3}{10}i\\
        -\frac{1}{3}i & \frac{1}{5}+\frac{1}{15}i & \frac{3}{5}+\frac{1}{5}i
      \end{bmatrix}.
    \end{align*}

  \item
    We have
    \begin{align*}
      A =
      \begin{bmatrix}
        2 & 5 & -1\\
        4 & -1 & 2\\
        6 & 4 & 1
      \end{bmatrix}\ \text{and}\
      B =
      \begin{bmatrix}
        1 & -1 & 2\\
        3 & 2 & 4\\
        0 & 1 & -2
      \end{bmatrix}.
    \end{align*}
    We would like to determine if either is invertible and what
    its inverse is if so.
    \begin{align*}
      \begin{bmatrix}
        2 & 5 & -1\\
        4 & -1 & 2\\
        6 & 4 & 1
      \end{bmatrix},\
      \begin{bmatrix}
        1 & 0 & 0\\
        0 & 1 & 0\\
        0 & 0 & 1
      \end{bmatrix}
    \end{align*}
    \begin{align*}
      \begin{bmatrix}
        1 & \frac{5}{2} & -\frac{1}{2}\\
        4 & -1 & 2\\
        6 & 4 & 1
      \end{bmatrix},\
      \begin{bmatrix}
        \frac{1}{2} & 0 & 0\\
        0 & 1 & 0\\
        0 & 0 & 1
      \end{bmatrix}
    \end{align*}
    \begin{align*}
      \begin{bmatrix}
        1 & \frac{5}{2} & -\frac{1}{2}\\
        0 & -11 & 4\\
        6 & 4 & 1
      \end{bmatrix},\
      \begin{bmatrix}
        \frac{1}{2} & 0 & 0\\
        -2 & 1 & 0\\
        0 & 0 & 1
      \end{bmatrix}
    \end{align*}
    \begin{align*}
      \begin{bmatrix}
        1 & \frac{5}{2} & -\frac{1}{2}\\
        0 & -11 & 4\\
        0 & -11 & 4
      \end{bmatrix},\
      \begin{bmatrix}
        \frac{1}{2} & 0 & 0\\
        -2 & 1 & 0\\
        -3 & 0 & 1
      \end{bmatrix}
    \end{align*}
    \begin{align*}
      \begin{bmatrix}
        1 & \frac{5}{2} & -\frac{1}{2}\\
        0 & -11 & 4\\
        0 & 0 & 0
      \end{bmatrix},\
      \begin{bmatrix}
        \frac{1}{2} & 0 & 0\\
        -2 & 1 & 0\\
        -1 & -1 & 1
      \end{bmatrix}
    \end{align*}
    \begin{align*}
      \begin{bmatrix}
        1 & \frac{5}{2} & -\frac{1}{2}\\
        0 & 1 & -\frac{4}{11}\\
        0 & 0 & 0
      \end{bmatrix},\
      \begin{bmatrix}
        \frac{1}{2} & 0 & 0\\
        \frac{2}{11} & -\frac{1}{11} & 0\\
        -1 & -1 & 1
      \end{bmatrix}
    \end{align*}
    \begin{align*}
      \begin{bmatrix}
        1 & 0 & \frac{9}{22}\\
        0 & 1 & -\frac{4}{11}\\
        0 & 0 & 0
      \end{bmatrix},\
      \begin{bmatrix}
        \frac{1}{22} & \frac{5}{22} & 0\\
        \frac{2}{11} & -\frac{1}{11} & 0\\
        -1 & -1 & 1
      \end{bmatrix},
    \end{align*}
    so $A$ is not invertible.
    \begin{align*}
      \begin{bmatrix}
        1 & -1 & 2\\
        3 & 2 & 4\\
        0 & 1 & -2
      \end{bmatrix},\
      \begin{bmatrix}
        1 & 0 & 0\\
        0 & 1 & 0\\
        0 & 0 & 1
      \end{bmatrix}
    \end{align*}
    \begin{align*}
      \begin{bmatrix}
        1 & -1 & 2\\
        0 & 5 & -2\\
        0 & 1 & -2
      \end{bmatrix},\
      \begin{bmatrix}
        1 & 0 & 0\\
        -3 & 1 & 0\\
        0 & 0 & 1
      \end{bmatrix}
    \end{align*}
    \begin{align*}
      \begin{bmatrix}
        1 & -1 & 2\\
        0 & 1 & -\frac{2}{5}\\
        0 & 1 & -2
      \end{bmatrix},\
      \begin{bmatrix}
        1 & 0 & 0\\
        -\frac{3}{5} & \frac{1}{5} & 0\\
        0 & 0 & 1
      \end{bmatrix}
    \end{align*}
    \begin{align*}
      \begin{bmatrix}
        1 & -1 & 2\\
        0 & 1 & -\frac{2}{5}\\
        0 & 0 & -\frac{8}{5}
      \end{bmatrix},\
      \begin{bmatrix}
        1 & 0 & 0\\
        -\frac{3}{5} & \frac{1}{5} & 0\\
        \frac{3}{5} & -\frac{1}{5} & 1
      \end{bmatrix}
    \end{align*}
    \begin{align*}
      \begin{bmatrix}
        1 & 0 & \frac{8}{5}\\
        0 & 1 & -\frac{2}{5}\\
        0 & 0 & -\frac{8}{5}
      \end{bmatrix},\
      \begin{bmatrix}
        \frac{2}{5} & \frac{1}{5} & 0\\
        -\frac{3}{5} & \frac{1}{5} & 0\\
        \frac{3}{5} & -\frac{1}{5} & 1
      \end{bmatrix}
    \end{align*}
    \begin{align*}
      \begin{bmatrix}
        1 & 0 & 0\\
        0 & 1 & -\frac{2}{5}\\
        0 & 0 & -\frac{8}{5}
      \end{bmatrix},\
      \begin{bmatrix}
        1 & 0 & 1\\
        -\frac{3}{5} & \frac{1}{5} & 0\\
        \frac{3}{5} & -\frac{1}{5} & 1
      \end{bmatrix}
    \end{align*}
    \begin{align*}
      \begin{bmatrix}
        1 & 0 & 0\\
        0 & 1 & -\frac{2}{5}\\
        0 & 0 & 1
      \end{bmatrix},\
      \begin{bmatrix}
        1 & 0 & 1\\
        -\frac{3}{5} & \frac{1}{5} & 0\\
        -\frac{3}{8} & \frac{1}{8} & -\frac{5}{8}
      \end{bmatrix}
    \end{align*}
    \begin{align*}
      \begin{bmatrix}
        1 & 0 & 0\\
        0 & 1 & 0\\
        0 & 0 & 1
      \end{bmatrix},\
      \begin{bmatrix}
        1 & 0 & 1\\
        -\frac{3}{4} & \frac{1}{4} & -\frac{1}{4}\\
        -\frac{3}{8} & \frac{1}{8} & -\frac{5}{8}
      \end{bmatrix},
    \end{align*}
    so $B$ is invertible and its inverse is
    \begin{align*}
      \begin{bmatrix}
        1 & 0 & 1\\
        -\frac{3}{4} & \frac{1}{4} & -\frac{1}{4}\\
        -\frac{3}{8} & \frac{1}{8} & -\frac{5}{8}
      \end{bmatrix}.
    \end{align*}

  \item
    We have
    \begin{align*}
      A =
      \begin{bmatrix}
        5 & 0 & 0\\
        1 & 5 & 0\\
        0 & 1 & 5
      \end{bmatrix}.
    \end{align*}
    We would like to know for which $X$ there is a scalar $c$
    such that $AX = cX$. We know that if $R$ is a row-reduced
    echelon matrix that is row-equivalent to $A$, then $R = PA$
    where $P$ is a $3 \times 3$ invertible matrix. The solutions
    of $AX = cX$ are then the same as the solutions of $RX = PAX
    = PcX$.
    \begin{align*}
      \begin{bmatrix}
        5 & 0 & 0\\
        1 & 5 & 0\\
        0 & 1 & 5
      \end{bmatrix},\
      \begin{bmatrix}
        1 & 0 & 0\\
        0 & 1 & 0\\
        0 & 0 & 1
      \end{bmatrix}
    \end{align*}
    \begin{align*}
      \begin{bmatrix}
        1 & 0 & 0\\
        1 & 5 & 0\\
        0 & 1 & 5
      \end{bmatrix},\
      \begin{bmatrix}
        \frac{1}{5} & 0 & 0\\
        0 & 1 & 0\\
        0 & 0 & 1
      \end{bmatrix}
    \end{align*}
    \begin{align*}
      \begin{bmatrix}
        1 & 0 & 0\\
        0 & 5 & 0\\
        0 & 1 & 5
      \end{bmatrix},\
      \begin{bmatrix}
        \frac{1}{5} & 0 & 0\\
        -\frac{1}{5} & 1 & 0\\
        0 & 0 & 1
      \end{bmatrix}
    \end{align*}
    \begin{align*}
      \begin{bmatrix}
        1 & 0 & 0\\
        0 & 1 & 0\\
        0 & 1 & 5
      \end{bmatrix},\
      \begin{bmatrix}
        \frac{1}{5} & 0 & 0\\
        -\frac{1}{25} & \frac{1}{5} & 0\\
        0 & 0 & 1
      \end{bmatrix}
    \end{align*}
    \begin{align*}
      \begin{bmatrix}
        1 & 0 & 0\\
        0 & 1 & 0\\
        0 & 0 & 5
      \end{bmatrix},\
      \begin{bmatrix}
        \frac{1}{5} & 0 & 0\\
        -\frac{1}{25} & \frac{1}{5} & 0\\
        \frac{1}{25} & -\frac{1}{5} & 1
      \end{bmatrix}
    \end{align*}
    \begin{align*}
      \begin{bmatrix}
        1 & 0 & 0\\
        0 & 1 & 0\\
        0 & 0 & 1
      \end{bmatrix},\
      \begin{bmatrix}
        \frac{1}{5} & 0 & 0\\
        -\frac{1}{25} & \frac{1}{5} & 0\\
        \frac{1}{125} & -\frac{1}{25} & \frac{1}{5}
      \end{bmatrix}.
    \end{align*}
    Therefore $R = I$ and
    \begin{align*}
      P =
      \begin{bmatrix}
        \frac{1}{5} & 0 & 0\\
        -\frac{1}{25} & \frac{1}{5} & 0\\
        \frac{1}{125} & -\frac{1}{25} & \frac{1}{5}
      \end{bmatrix}.
    \end{align*}
    If
    \begin{align*}
      cX =
      \begin{bmatrix}
        cx_1\\
        cx_2\\
        cx_3\\
      \end{bmatrix},
    \end{align*}
    then
    \begin{align*}
      PcX =
      \begin{bmatrix}
        \frac{1}{5} & 0 & 0\\
        -\frac{1}{25} & \frac{1}{5} & 0\\
        \frac{1}{125} & -\frac{1}{25} & \frac{1}{5}
      \end{bmatrix}
      \begin{bmatrix}
        cx_1\\
        cx_2\\
        cx_3\\
      \end{bmatrix}
      =
      \begin{bmatrix}
        \frac{1}{5}cx_1\\
        -\frac{1}{25}cx_1+\frac{1}{5}cx_2\\
        \frac{1}{125}cx_1-\frac{1}{25}cx_2+\frac{1}{5}cx_3\\
      \end{bmatrix}.
    \end{align*}
    $RX = IX = X = PcX$. Since this implies that
    \begin{align*}
      x_1 =&\ \frac{1}{5}cx_1,
    \end{align*}
    $c$ has to be $5$ or $x_1$ has to be $0$. If we assume that
    $c = 5$,
    \begin{align*}
      x_2 =&\ -\frac{1}{5}x_1 + x_2,
    \end{align*}
    which is only possible if $x_1 = 0$, so $x_1 = 0$. Then
    \begin{align*}
      x_2 =&\ \frac{1}{5}cx_2,
    \end{align*}
    which implies that $c$ has to be $5$ or $x_2$ has to be $0$.
    If we assume that $c = 5$,
    \begin{align*}
      x_3 =&\ -\frac{1}{5}x_2 + x_3,
    \end{align*}
    which again is only possible if $x_2 = 0$, so $x_2 = 0$ as
    well. Then
    \begin{align*}
      x_3 =&\ \frac{1}{5}cx_3,
    \end{align*}
    which is valid if $c = 5$ or if $x_3 = 0$.

    Therefore, $AX = cX$ when $X = 0$ or when
    \begin{align*}
      c = 5,\ X =
      \begin{bmatrix}
        0\\
        0\\
        x_3\\
      \end{bmatrix},
    \end{align*}
    where $x_3$ can take on any value in the field.
\end{enumerate}

\end{document}
