\documentclass[12pt]{article}
\usepackage{mathtools}
\usepackage{amsthm}
\usepackage{amsfonts}
\usepackage{amssymb}
\usepackage{fontspec}
\usepackage{xfrac}
\usepackage{array}
\usepackage{siunitx}
\usepackage{gensymb}
\usepackage{enumitem}
\usepackage{dirtytalk}
\usepackage{bm}
\title{Invertible matrices (exercises)}
\author{Zoë Sparks}

\begin{document}

\theoremstyle{definition}

\sisetup{quotient-mode=fraction}
\newtheorem{thm}{Theorem}
\newtheorem*{nthm}{Theorem}
\newtheorem{sthm}{}[thm]
\newtheorem{lemma}{Lemma}[thm]
\newtheorem*{nlemma}{Lemma}
\newtheorem{cor}{Corollary}[thm]
\newtheorem*{prop}{Property}
\newtheorem*{defn}{Definition}
\newtheorem*{comm}{Comment}
\newtheorem*{exm}{Example}

\maketitle

\begin{enumerate}
  \item We have
    \begin{align*}
      A =
      \begin{bmatrix}
        1  & 2  & 1 & 0\\
        -1 & 0  & 3 & 5\\
        1  & -2 & 1 & 1\\
      \end{bmatrix}.
    \end{align*}
    We would like to find a row-reduced echelon matrix $R$ which
    is row-equivalent to $A$ and an invertible $3 \times 3$
    matrix $P$ such that $R = PA$.
    \begin{align*}
      \begin{bmatrix}
        1  & 2  & 1 & 0\\
        -1 & 0  & 3 & 5\\
        1  & -2 & 1 & 1\\
      \end{bmatrix},\
      \begin{bmatrix}
        1 & 0 & 0\\
        0 & 1 & 0\\
        0 & 0 & 1\\
      \end{bmatrix}
    \end{align*}
    \begin{align*}
      \begin{bmatrix}
        1  & 2  & 1 & 0\\
        0  & 2  & 4 & 5\\
        1  & -2 & 1 & 1\\
      \end{bmatrix},\
      \begin{bmatrix}
        1 & 0 & 0\\
        1 & 1 & 0\\
        0 & 0 & 1\\
      \end{bmatrix}
    \end{align*}
    \begin{align*}
      \begin{bmatrix}
        1  & 2  & 1 & 0\\
        0  & 2  & 4 & 5\\
        0  & -4 & 0 & 1\\
      \end{bmatrix},\
      \begin{bmatrix}
        1  & 0 & 0\\
        1  & 1 & 0\\
        -1 & 0 & 1\\
      \end{bmatrix}
    \end{align*}
    \begin{align*}
      \begin{bmatrix}
        1  & 0  & -3 & -5\\
        0  & 2  & 4  & 5\\
        0  & -4 & 0  & 1\\
      \end{bmatrix},\
      \begin{bmatrix}
        0  & -1 & 0\\
        1  & 1  & 0\\
        -1 & 0  & 1\\
      \end{bmatrix}
    \end{align*}
    \begin{align*}
      \begin{bmatrix}
        1  & 0  & -3 & -5\\
        0  & 2  & 4  & 5\\
        0  & 0  & 8  & 11\\
      \end{bmatrix},\
      \begin{bmatrix}
        0  & -1 & 0\\
        1  & 1  & 0\\
        1  & 2  & 1\\
      \end{bmatrix}
    \end{align*}
    \begin{align*}
      \begin{bmatrix}
        1  & 0  & -3 & -5\\
        0  & 1  & 2  & \frac{5}{2}\\
        0  & 0  & 8  & 11\\
      \end{bmatrix},\
      \begin{bmatrix}
        0  & -1 & 0\\
        \frac{1}{2} & \frac{1}{2} & 0\\
        1 & 2  & 1\\
      \end{bmatrix}
    \end{align*}
    \begin{align*}
      \begin{bmatrix}
        1  & 0  & -3 & -5\\
        0  & 1  & 2  & \frac{5}{2}\\
        0  & 0  & 1  & \frac{11}{8}\\
      \end{bmatrix},\
      \begin{bmatrix}
        0  & -1 & 0\\
        \frac{1}{2} & \frac{1}{2} & 0\\
        \frac{1}{8} & \frac{1}{4} & \frac{1}{8}\\
      \end{bmatrix}
    \end{align*}
    \begin{align*}
      \begin{bmatrix}
        1  & 0  & -3 & -5\\
        0  & 1  & 0  & -\frac{1}{4}\\
        0  & 0  & 1  & \frac{11}{8}\\
      \end{bmatrix},\
      \begin{bmatrix}
        0  & -1 & 0\\
        \frac{1}{4} & 0 & -\frac{1}{4}\\
        \frac{1}{8} & \frac{1}{4} & \frac{1}{8}\\
      \end{bmatrix}
    \end{align*}
    \begin{align*}
      \begin{bmatrix}
        1  & 0  & 0  & -\frac{7}{8}\\
        0  & 1  & 0  & -\frac{1}{4}\\
        0  & 0  & 1  & \frac{11}{8}\\
      \end{bmatrix},\
      \begin{bmatrix}
        \frac{3}{8}  & -\frac{1}{4} & \frac{3}{8}\\
        \frac{1}{4} & 0 & -\frac{1}{4}\\
        \frac{1}{8} & \frac{1}{4} & \frac{1}{8}\\
      \end{bmatrix}.
    \end{align*}
    \begin{align*}
      \begin{bmatrix}
        1  & 0  & 0  & -\frac{7}{8}\\
        0  & 1  & 0  & -\frac{1}{4}\\
        0  & 0  & 1  & \frac{11}{8}\\
      \end{bmatrix}
      =
      \begin{bmatrix}
        \frac{3}{8}  & -\frac{1}{4} & \frac{3}{8}\\
        \frac{1}{4} & 0 & -\frac{1}{4}\\
        \frac{1}{8} & \frac{1}{4} & \frac{1}{8}\\
      \end{bmatrix}
      \begin{bmatrix}
        1  & 2  & 1 & 0\\
        -1 & 0  & 3 & 5\\
        1  & -2 & 1 & 1\\
      \end{bmatrix}.
    \end{align*}
    \begin{align*}
      \begin{bmatrix}
        \frac{3}{8}  & -\frac{1}{4} & \frac{3}{8}\\
        \frac{1}{4} & 0 & -\frac{1}{4}\\
        \frac{1}{8} & \frac{1}{4} & \frac{1}{8}\\
      \end{bmatrix}
    \end{align*}
    is row-equivalent to $I$ as we have seen and is thus
    invertible, so
    \begin{align*}
      R =
      \begin{bmatrix}
        1  & 0  & 0  & -\frac{7}{8}\\
        0  & 1  & 0  & -\frac{1}{4}\\
        0  & 0  & 1  & \frac{11}{8}\\
      \end{bmatrix}\ \text{and}\
      P =
      \begin{bmatrix}
        \frac{3}{8}  & -\frac{1}{4} & \frac{3}{8}\\
        \frac{1}{4} & 0 & -\frac{1}{4}\\
        \frac{1}{8} & \frac{1}{4} & \frac{1}{8}\\
      \end{bmatrix}.
    \end{align*}

  \item
    We have
    \begin{align*}
      A =
      \begin{bmatrix}
        2 & 0 & i\\
        1 & -3 & -i\\
        i & 1 & 1
      \end{bmatrix},
    \end{align*}
    and we'd like to do the same thing with it that we did with
    $A$ in (1).
    \begin{align*}
      \begin{bmatrix}
        2 & 0 & i\\
        1 & -3 & -i\\
        i & 1 & 1
      \end{bmatrix},\
      \begin{bmatrix}
        1 & 0 & 0\\
        0 & 1 & 0\\
        0 & 0 & 1
      \end{bmatrix}
    \end{align*}
    \begin{align*}
      \begin{bmatrix}
        1 & 0 & \frac{1}{2}i\\
        1 & -3 & -i\\
        i & 1 & 1
      \end{bmatrix},\
      \begin{bmatrix}
        \frac{1}{2} & 0 & 0\\
        0 & 1 & 0\\
        0 & 0 & 1
      \end{bmatrix}
    \end{align*}
    \begin{align*}
      \begin{bmatrix}
        1 & 0 & \frac{1}{2}i\\
        0 & -3 & -\frac{3}{2}i\\
        i & 1 & 1
      \end{bmatrix},\
      \begin{bmatrix}
        \frac{1}{2} & 0 & 0\\
        -\frac{1}{2} & 1 & 0\\
        0 & 0 & 1
      \end{bmatrix}
    \end{align*}
    \begin{align*}
      \begin{bmatrix}
        1 & 0 & \frac{1}{2}i\\
        0 & -3 & -\frac{3}{2}i\\
        0 & 1 & \frac{3}{2}
      \end{bmatrix},\
      \begin{bmatrix}
        \frac{1}{2} & 0 & 0\\
        -\frac{1}{2} & 1 & 0\\
        -\frac{1}{2}i & 0 & 1
      \end{bmatrix}
    \end{align*}
    \begin{align*}
      \begin{bmatrix}
        1 & 0 & \frac{1}{2}i\\
        0 & 1 & \frac{1}{2}i\\
        0 & 1 & \frac{3}{2}
      \end{bmatrix},\
      \begin{bmatrix}
        \frac{1}{2} & 0 & 0\\
        \frac{1}{6} & -\frac{1}{3} & 0\\
        -\frac{1}{2}i & 0 & 1
      \end{bmatrix}
    \end{align*}
    \begin{align*}
      \begin{bmatrix}
        1 & 0 & \frac{1}{2}i\\
        0 & 1 & \frac{1}{2}i\\
        0 & 0 & \frac{3}{2}-\frac{1}{2}i
      \end{bmatrix},\
      \begin{bmatrix}
        \frac{1}{2} & 0 & 0\\
        \frac{1}{6} & -\frac{1}{3} & 0\\
        -\frac{1}{6}-\frac{1}{2}i & \frac{1}{3} & 1
      \end{bmatrix}
    \end{align*}
    \begin{align*}
      \begin{bmatrix}
        1 & 0 & \frac{1}{2}i\\
        0 & 1 & \frac{1}{2}i\\
        0 & 0 & 1
      \end{bmatrix},\
      \begin{bmatrix}
        \frac{1}{2} & 0 & 0\\
        \frac{1}{6} & -\frac{1}{3} & 0\\
        -\frac{1}{3}i & \frac{1}{5}+\frac{1}{15}i & \frac{3}{5}+\frac{1}{5}i
      \end{bmatrix}
    \end{align*}
    \begin{align*}
      \begin{bmatrix}
        1 & 0 & \frac{1}{2}i\\
        0 & 1 & 0\\
        0 & 0 & 1
      \end{bmatrix},\
      \begin{bmatrix}
        \frac{1}{2} & 0 & 0\\
        0 & -\frac{3}{10}-\frac{1}{10}i & \frac{1}{10}-\frac{3}{10}i\\
        -\frac{1}{3}i & \frac{1}{5}+\frac{1}{15}i & \frac{3}{5}+\frac{1}{5}i
      \end{bmatrix}
    \end{align*}
    \begin{align*}
      \begin{bmatrix}
        1 & 0 & 0\\
        0 & 1 & 0\\
        0 & 0 & 1
      \end{bmatrix},\
      \begin{bmatrix}
        \frac{1}{3} & \frac{1}{30}-\frac{1}{10}i & \frac{1}{10}-\frac{3}{10}i\\
        0 & -\frac{3}{10}-\frac{1}{10}i & \frac{1}{10}-\frac{3}{10}i\\
        -\frac{1}{3}i & \frac{1}{5}+\frac{1}{15}i & \frac{3}{5}+\frac{1}{5}i
      \end{bmatrix}.
    \end{align*}
    \begin{align*}
      \begin{bmatrix}
        1 & 0 & 0\\
        0 & 1 & 0\\
        0 & 0 & 1
      \end{bmatrix}=
      \begin{bmatrix}
        \frac{1}{3} & \frac{1}{30}-\frac{1}{10}i & \frac{1}{10}-\frac{3}{10}i\\
        0 & -\frac{3}{10}-\frac{1}{10}i & \frac{1}{10}-\frac{3}{10}i\\
        -\frac{1}{3}i & \frac{1}{5}+\frac{1}{15}i & \frac{3}{5}+\frac{1}{5}i
      \end{bmatrix}
      \begin{bmatrix}
        2 & 0 & i\\
        1 & -3 & -i\\
        i & 1 & 1
      \end{bmatrix},
    \end{align*}
    so
    \begin{align*}
      R =
      \begin{bmatrix}
        1 & 0 & 0\\
        0 & 1 & 0\\
        0 & 0 & 1
      \end{bmatrix}\ \text{and}\
      P =
      \begin{bmatrix}
        \frac{1}{3} & \frac{1}{30}-\frac{1}{10}i & \frac{1}{10}-\frac{3}{10}i\\
        0 & -\frac{3}{10}-\frac{1}{10}i & \frac{1}{10}-\frac{3}{10}i\\
        -\frac{1}{3}i & \frac{1}{5}+\frac{1}{15}i & \frac{3}{5}+\frac{1}{5}i
      \end{bmatrix}.
    \end{align*}
\end{enumerate}

\end{document}
