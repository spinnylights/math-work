\documentclass[12pt]{article}
\usepackage{mathtools}
\usepackage{amsthm}
\usepackage{amsfonts}
\usepackage{amssymb}
\usepackage{fontspec}
\usepackage{xfrac}
\usepackage{array}
\usepackage{siunitx}
\usepackage{gensymb}
\usepackage{enumitem}
\title{Linear equations}
\author{Zoë Sparks}

\begin{document}

\theoremstyle{definition}

\sisetup{quotient-mode=fraction}
\newtheorem{thm}{Theorem}
\newtheorem*{nthm}{Theorem}
\newtheorem{sthm}{}[thm]
\newtheorem{lemma}{Lemma}[thm]
\newtheorem*{cor}{Corollary}
\newtheorem*{prop}{Property}
\newtheorem*{defn}{Definition}
\newtheorem*{comm}{Comment}
\newtheorem*{exm}{Example}

\maketitle

\begin{defn}
  A \textbf{system of \textit{m} linear equations in \textit{n}
  unknowns} is a set of equations such that
  \begin{equation} \label{eq:syslin}
  \begin{array}{ccccccccc}
    A_{11}x_1 & + & A_{12}x_2 & + & \ldots & + & A_{1n}x_n & = & y_1\\
    A_{21}x_1 & + & A_{22}x_2 & + & \ldots & + & A_{2n}x_n & = & y_2\\
    \vdots    & + & \vdots    & + & \ldots & + & \vdots    & = & \vdots\\
    A_{m1}x_1 & + & A_{m2}x_2 & + & \ldots & + & A_{mn}x_n & = & y_m,
  \end{array}
  \end{equation}
  where $x_1,\ldots,x_n$, $y_1,\ldots,y_n$, and $A_{ij}, 1 \leq i
  \leq m, 1 \leq j \leq n$ are elements of a field $F$. Any
  $n$-tuple $x_1,\ldots,x_n$ of elements of $F$ which satisfies
  each of the equations of such a system is called one of its
  \textbf{solutions}. If $y_1 = y_2 = \ldots = y_n = 0$, the
  system is said to be \textbf{homogenous}, or each of its
  equations is said to be homogenous.
\end{defn}

\begin{defn}
  If we select $m$ scalars $c_1,\ldots,c_m$ from $F$ and multiply
  the $j$th equation in \eqref{eq:syslin} by $c_j$:
  \[
  \begin{array}{ccccccccc}
    c_{1}A_{11}x_1 & + & c_{1}A_{12}x_2 & + & \ldots & + & c_{1}A_{1n}x_n & = & c_{1}y_1\\
    c_{2}A_{21}x_1 & + & c_{2}A_{22}x_2 & + & \ldots & + & c_{2}A_{2n}x_n & = & c_{2}y_2\\
    \vdots    & + & \vdots    & + & \ldots & + & \vdots    & = & \vdots\\
    c_{m}A_{m1}x_1 & + & c_{m}A_{m2}x_2 & + & \ldots & + & c_{m}A_{mn}x_n & = & c_{m}y_m,
  \end{array}
  \]
  we can add the equations together to obtain
  \begin{equation} \label{eq:lincomb}
    (c_{1}A_{11} + \ldots + c_{m}A_{m1})x_1 + \ldots +
    (c_{1}A_{1n} + \ldots + c_{m}A_{mn})x_n = c_{1}y_1 + \ldots +
    c_{m}y_m.
  \end{equation}
  \eqref{eq:lincomb} is called a \textbf{linear combination} of
  the equations in \eqref{eq:syslin}. Given the way we arrived at
  \eqref{eq:lincomb}, any solution of \eqref{eq:syslin} is
  clearly a solution of \eqref{eq:lincomb} as well.
\end{defn}

\begin{defn}
  If we have another system of linear equations
  \begin{equation} \label{eq:syslin2}
  \begin{array}{ccccccc}
    B_{11}x_1 & + & \ldots & + & B_{1n}x_n & = & z_1\\
    \vdots    & + & \ldots & + & \vdots    & = & \vdots\\
    B_{k1}x_1 & + & \ldots & + & B_{kn}x_n & = & z_k
  \end{array}
  \end{equation}
  such that all $k$ equations in \eqref{eq:syslin2} are linear
  combinations of the equations in \eqref{eq:syslin}, it follows
  that any solution of \eqref{eq:syslin} is also a solution of
  \eqref{eq:syslin2}.

  It may be the case that some of the solutions of
  \eqref{eq:syslin2} are not solutions of \eqref{eq:syslin}.
  However, if all of the equations in \eqref{eq:syslin} are also
  linear combinations of the equations in \eqref{eq:syslin2},
  then clearly all of the solutions of \eqref{eq:syslin2}
  \textit{are} solutions of \eqref{eq:syslin}. If all of the
  equations in \eqref{eq:syslin2} are linear combinations of the
  equations in \eqref{eq:syslin} and vice versa,
  \eqref{eq:syslin2} and \eqref{eq:syslin} are said to be
  \textbf{equivalent}.
\end{defn}

\begin{thm}
  Equivalent systems of linear equations have exactly the same
  solutions.
  \begin{proof}
    Follows trivially from the definition of equivalence.
  \end{proof}
\end{thm}

\begin{comm}
  (Mine.) If we have a system
  \[
  \begin{array}{ccccccc}
    B_{11}x_1 & + & \ldots & + & B_{1n}x_n & = & z_1\\
    \vdots    & + & \ldots & + & \vdots    & = & \vdots\\
    B_{k1}x_1 & + & \ldots & + & B_{kn}x_n & = & z_k,
  \end{array}
  \]
  and all its equations are linear combinations of the equations
  in
  \[
  \begin{array}{ccccccc}
    A_{11}x_1 & + & \ldots & + & A_{1n}x_n & = & y_1\\
    \vdots    & + & \ldots & + & \vdots    & = & \vdots\\
    A_{k1}x_1 & + & \ldots & + & A_{kn}x_n & = & y_k,
  \end{array}
  \]
  that implies that one of its equations
  \begin{align*}
    B_{i1}x_1 + \ldots + B_{in}x_n = z_i,
  \end{align*}
  where $i$ is a number between $1$ and $k$, could be expressed
  as a linear combination of
  \[
  \begin{array}{ccccccc}
    c_{i1}A_{11}x_1 & + & \ldots & + & c_{i1}A_{1n}x_n & = & c_{i1}y_1\\
    \vdots    & + & \ldots & + & \vdots    & = & \vdots\\
    c_{ik}A_{k1}x_1 & + & \ldots & + & c_{ik}A_{kn}x_n & = & c_{ik}y_k
  \end{array}
  \]
  for some list of scalars $c_{i1}, ..., c_{ik}$. If $j$ is a
  number between $1$ and $n$, this also implies that $B_{ij} =
  c_{i1}A_{1j} + \ldots + c_{ik}A_{kj}$ and that $z_{i} =
  c_{i1}y_{1} + \ldots + c_{ik}y_{k}$.

  It's worth noting that the only new element which is introduced
  here is the list of scalars $c_{i1}, \ldots, c_{ik}$; everything
  else is given by the two systems of equations. For the whole of
  both systems, we thus have a list of lists of scalars; we can
  represent this as $c_{11}, \ldots, c_{kk}$:
  \begin{align*}
    \begin{bmatrix}
      c_{11} & \cdots & c_{1k}\\
      \vdots & \ddots & \vdots\\
      c_{k1} & \cdots & c_{kk}
    \end{bmatrix}
  \end{align*}
  Showing that the two systems are equivalent is the same as
  showing that such a list of lists of scalars exists going in
  both directions.
\end{comm}

\end{document}
