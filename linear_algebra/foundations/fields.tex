\documentclass[12pt]{article}
\usepackage{mathtools}
\usepackage{amsthm}
\usepackage{amsfonts}
\usepackage{amssymb}
\usepackage{fontspec}
\usepackage{xfrac}
\usepackage{array}
\usepackage{siunitx}
\usepackage{gensymb}
\usepackage{enumitem}
\title{Fields}
\author{Zoë Sparks}

\begin{document}

\theoremstyle{definition}

\sisetup{quotient-mode=fraction}
\newtheorem{thm}{Theorem}
\newtheorem*{nthm}{Theorem}
\newtheorem{sthm}{}[thm]
\newtheorem{lemma}{Lemma}[thm]
\newtheorem*{cor}{Corollary}
\newtheorem*{prop}{Property}
\newtheorem*{defn}{Definition}
\newtheorem*{comm}{Comment}
\newtheorem*{exm}{Example}

\maketitle

\begin{defn}
  A \textbf{field} $F$ is a set of objects $x, y, z, …$ together
  with two operations on the elements of $F$: \textbf{addition},
  which associates each pair of elements $x, y$ in $F$ with an
  element $x + y$ in $F$, and \textbf{multiplication}, which
  associates each pair of elements $x, y$ with an element $xy$ in
  $F$. Furthermore, these two operations should satisfy the
  following conditions:

  \begin{enumerate}
      \item Addition is commutative: $x + y = y + x$ for all $x,
        y$ in $F$.
      \item Addition is associative: $x + (y + z) = (x + y) + z$
        for all $x, y, z$ in $F$.
      \item There is a unique element $0$ (zero) in $F$ such that
        $x + 0 = x$ for every $x$ in $F$.
      \item For every $x$ in $F$ there is a unique element $-x$
        in $F$ such that $x + (-x) = 0$.
      \item Multiplication is commutative: $xy = yx$ for all $x,
        y$ in $F$.
      \item Multiplication is associative: $x(yz) = (xy)z$ for
        all $x, y, z$ in $F$.
      \item There is a unique element $1$ (one) in $F$ such that
        $1 \neq 0$ and $1x = x$ for every $x$ in $F$.
      \item For each element $x \neq 0$ in $F$ there is a unique
        element $x^{-1}$ (or $1/x$) in $F$ such that $xx^{-1} =
        1$.
      \item Multiplication distributes over addition: $x(y + z) =
        xy + xz$ for all $x, y, z$ in $F$.
  \end{enumerate}
\end{defn}

\begin{exm}
  The set of natural numbers $\mathbb{N}$ is not a field. $0$ is
  not a natural number, nor is $-n$ for any natural number $n$,
  nor $1/n$ (except in the case of $1$).
\end{exm}

\begin{exm}
  The set of integers $\mathbb{Z}$ is not a field. It comes
  close, but for an integer $n$, $1/n$ is not an integer unless
  $n = 1$ or $n = -1$.
\end{exm}

\begin{exm}
  The set of rational numbers $\mathbb{Q}$ is a field.
\end{exm}

\begin{exm}
  The set of real numbers $\mathbb{R}$ is a field.
\end{exm}

\begin{exm}
  The set of complex numbers $\mathbb{C}$ is a field.
\end{exm}

\begin{comm}
  In any field, one usually writes
  \begin{align*}
    x - y,\ \frac{x}{y},\ x + y + z,\ xyz,\ x^2,\ x^3,\ 2x,\ 3x,\ldots
  \end{align*}
  in place of
  \begin{align*}
    x + (-y),\ x \cdot \frac{1}{y},\ (x + y) + z,\ (xy)z,\ xx,\ xxx,\ x + x,\ x + x + x,\ldots
  \end{align*}
\end{comm}

\begin{thm}
  The axioms for addition imply that
  \begin{enumerate}
    \item
      if $x + y = x + z$ then $y = z$,
    \item
      if $x + y = x$ then $y = 0$,
    \item
      if $x + y = 0 $ then $y = -x$, and
    \item
      $-(-x) = x$.
  \end{enumerate}

  \begin{proof}
    Let $F$ be a field such that $x,y \in F$. Let $a = x + y$ and $b = x + z$; then
    $a,b \in F$ and $a = b$, so $a-x,\ b-x \in F$ and $a - x = b - x$. Then $x + y -
    x = x + z - x$, so $x - x + y = x - x + z$, thus $0 + y = 0 + z$ and thus $y = z$.

    Letting $z = 0$ in (1) gives $x + y = x + 0$, which implies $y = 0$ by (1).
    Likewise, letting $z = -x$ in (1) gives $x + y = x - x = 0$, which implies $y =
    -x$ by (1). Last, letting $x = -x$ in (3) gives $y = -(-x)$, and if $-x - (-x) =
    0$, then $x - x - (-x) = x = -(-x)$.
  \end{proof}
\end{thm}

\begin{thm}
  The axioms for mulitplication imply that
  \begin{enumerate}
    \item
      if $x \neq 0$ and $xy = xz$, then $y = z$,
    \item
      if $x \neq 0$ and $xy = x$ then $y = 1$,
    \item
      if $x \neq 0$ and $xy = 1$ then $y = 1/x$, and
    \item
      if $x \neq 0$ then $1/(1/x) = x$.
  \end{enumerate}

  \begin{proof}
    Let $F$ be a field such that $x,y \in F$, and let $x \neq 0$. Let $a = xy$ and $b
    = xz$; then $a,b \in F$ and $a = b$, so $a/x,\ b/x \in F$ and $a/x = b/x$. Then
    $(xy)/x = (xz)/x$, so $y \cdot x/x = z \cdot x/x$, thus $y \cdot 1 = z \cdot 1$
    and thus $y = z$.

    Letting $z = 0$ in (1) gives $xy = x0$, which implies $y = 0$ by (1).  Likewise,
    letting $z = 1/x$ in (1) gives $xy = x(1/x) = 1$, which implies $y = 1/x$ by (1).
    Last, letting $x = 1/x$ in (3) gives $y = 1/(1/x)$, and if $1/x \cdot 1/(1/x) =
    1$, then $x \cdot 1/x \cdot 1/(1/x) = x \cdot 1 = 1 \cdot 1/(1/x) = 1/(1/x) = x$.
  \end{proof}
\end{thm}

\begin{defn}
  A \textbf{subfield} of the field $F$ is a set $G \subseteq F$
  such that $G$ is a field under the same definitions of addition
  and mulitplication used with $F$.
\end{defn}

\begin{comm}
  If $G$ is a subfield of $F$, $0$ and $1$ are in $G$, and if $x,
  y$ are in $G$, then $(x + y)$, $xy$, $-x$, and $x^{-1}$ (if
  $x \neq 0$) are also in $G$.
\end{comm}

\begin{exm}
  The field $\mathbb{R}$ is a subfield of $\mathbb{C}$.

  We can represent a real number $x$ as a complex number $(x +
  0i)$. This means that the real numbers $1$ and $0$ are in
  $\mathbb{C}$. It also means that if $x, y$ are real numbers,
  $(x + y)$, $xy$, $-x$, and (if $x \neq 0$) $x^{-1}$ are in
  $\mathbb{C}$.
\end{exm}

\begin{thm}
  Any subfield of $\mathbb{C}$ must contain every rational
  number.
  \begin{proof}
    By contradiction. Assume that $F$ is a subfield of
    $\mathbb{C}$ which omits the rational number $p$, $p \neq 0$.
    We know that for any rational number $q$, there is a rational
    number $r$ such that $qr = p$. This is because $qq^{-1} = 1$,
    and thus $qq^{-1}p = p$; therefore $r = q^{-1}p$. For $F$ to
    be a field, $qr$ would need to produce a result in $F$; since
    $F$ omits $p$, $q$ and $r$ must be omitted as well. But this
    ultimately leaves no rational numbers in $F$ at all, and thus
    $F$ cannot be a field in any sense, let alone a subfield of
    $\mathbb{C}$.
  \end{proof}
\end{thm}

\begin{thm}
  The set of all complex numbers of the form $x + y\sqrt{2}$,
  where $x$ and $y$ are rational, is a subfield of $\mathbb{C}$.

  \begin{proof}
    By lemmas 2.*.
  \end{proof}
\end{thm}

\begin{lemma}
  If $p,q,r,s$ are rational numbers, then $(p + q\sqrt{2}) + (r
  + s\sqrt{2})$ is a complex number of the form $x +
  y\sqrt{2}$ where $x, y$ are rational.
  \begin{proof}
    \begin{align*}
      (p + q\sqrt{2}) + (r + s\sqrt{2}) &=\\
      p + q\sqrt{2} + r + s\sqrt{2} &=\\
      p + r + q\sqrt{2} + s\sqrt{2} &=\\
      (p + r) + (q + s)\sqrt{2}&.
    \end{align*}
    Then $(p + q\sqrt{2}) + (r + s\sqrt{2})$ is a complex number
    of the form $x + y\sqrt{2}$, $x = (p + q)$, $y = (r + s)$.
  \end{proof}
\end{lemma}

\begin{lemma}
  If $p,q,r,s$ are rational numbers, then $(p + q\sqrt{2})
  \cdot (r + s\sqrt{2})$ is a complex number of the form $x +
  y\sqrt{2}$ where $x, y$ are rational.
  \begin{proof}
    \begin{align*}
      (p + q\sqrt{2}) \cdot (r + s\sqrt{2}) &=\\
      pr + qr\sqrt{2} + ps\sqrt{2} + qs(\sqrt{2})^{2} &=\\
      (pr + 2qs) + (qr + ps)\sqrt{2}&.
    \end{align*}
    Then $(p + q\sqrt{2}) \cdot (r + s\sqrt{2})$ is a complex
    number of the form $x + y\sqrt{2}$, $x = (pr + 2qs)$, $y =
    (qr + ps)$.
  \end{proof}
\end{lemma}

\begin{lemma}
  If $p,q$ are rational numbers, then $-(p + q\sqrt{2})$ is a
  complex number of the form $x + y\sqrt{2}$ where $x, y$ are
  rational.
  \begin{proof}
    $-(p + q\sqrt{2}) = -p + (-q)\sqrt{2}$. Then $-(p +
    q\sqrt{2})$ is a complex number of the form $x + y\sqrt{2}$,
    $x = -p$, $y = -q$.
  \end{proof}
\end{lemma}

\begin{lemma}
  If $p,q$ are rational numbers and $p + q\sqrt{2} \neq 0$, then
  $(p + q\sqrt{2})^{-1}$ is a complex number of the form $x +
  y\sqrt{2}$ where $x, y$ are rational.
  \begin{proof}
    \begin{align*}
      (p + q\sqrt{2})^{-1} &=\\
      \frac{1}{p + q\sqrt{2}} &=\\
      \frac{p - q\sqrt{2}}{(p + q\sqrt{2})(p - q\sqrt{2})} &=\\
      \frac{p - q\sqrt{2}}{p^{2} - 2q^{2}} &=\\
      \frac{p}{p^{2} - 2q^{2}} - \frac{q}{p^{2} - 2q^{2}}\sqrt{2}&.
    \end{align*}
    Then $(p + q\sqrt{2})^{-1}$ is a complex number of the form
    $x + y\sqrt{2}$, $x = \frac{p}{p^{2} - 2q^{2}}$, $y =
    -\frac{q}{p^{2} - 2q^{2}}$.
  \end{proof}
\end{lemma}

\begin{defn}
  A \textbf{scalar} is an element of a field.
\end{defn}

\begin{comm}
  If $x, y$ are scalars from a subfield $F$ of $\mathbb{C}$, then
  the performance of addition, multiplication, subtraction, or
  division with $x$ and $y$ does not leave $F$.
\end{comm}

\begin{comm}
  In some fields, it is possible to add $1$ to itself a finite
  number of times and obtain $0$, i.e.
  \begin{center}
    $1 + 1 + … + 1 = 0.$
  \end{center}
  However, this is not the case for $\mathbb{C}$, nor any
  subfield of it.
\end{comm}

\begin{defn}
  If a field $F$ has the property that it is possible to add $1$
  to itself a finite number of times and obtain $0$ in it, the
  least $n$ such that the sum of $n$ $1$s yields $0$ is called
  the \textbf{characteristic} of $F$. If $F$ does not have this
  property, it is said to be of \textbf{characteristic zero}.
\end{defn}

\begin{comm}
  Often, when assuming that $F$ is a subfield of $\mathbb{C}$,
  what is desired is to ensure that $F$ is a field of
  characteristic zero.
\end{comm}

\end{document}
