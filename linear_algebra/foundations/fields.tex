\documentclass[12pt]{article}
\usepackage{mathtools}
\usepackage{amsthm}
\usepackage{amsfonts}
\usepackage{amssymb}
\usepackage{fontspec}
\usepackage{xfrac}
\usepackage{array}
\usepackage{siunitx}
\usepackage{gensymb}
\usepackage{enumitem}
\usepackage{dirtytalk}
\usepackage{bm}
\title{Fields}
\author{Zoë Sparks}

\begin{document}

\theoremstyle{definition}

\sisetup{quotient-mode=fraction}
\newtheorem{thm}{Theorem}
\newtheorem*{nthm}{Theorem}
\newtheorem{sthm}{}[thm]
\newtheorem{lemma}{Lemma}[thm]
\newtheorem*{cor}{Corollary}
\newtheorem*{prop}{Property}
\newtheorem*{defn}{Definition}
\newtheorem*{comm}{Comment}
\newtheorem*{exm}{Example}

\maketitle

\begin{defn}
  A \textbf{field} $F$ is a set of objects $x, y, z, …$ together
  with two operations on the elements of $F$: \textbf{addition},
  which associates each pair of elements $x, y$ in $F$ with an
  element $x + y$ in $F$, and \textbf{multiplication}, which
  associates each pair of elements $x, y$ with an element $xy$ in
  $F$. Furthermore, these two operations should satisfy the
  following conditions:

  \begin{enumerate}
      \item Addition is commutative: $x + y = y + x$ for all $x,
        y$ in $F$.
      \item Addition is associative: $x + (y + z) = (x + y) + z$
        for all $x, y, z$ in $F$.
      \item There is a unique element $0$ (zero) in $F$ such that
        $x + 0 = x$ for every $x$ in $F$.
      \item For every $x$ in $F$ there is a unique element $-x$
        in $F$ such that $x + (-x) = 0$.
      \item Multiplication is commutative: $xy = yx$ for all $x,
        y$ in $F$.
      \item Multiplication is associative: $x(yz) = (xy)z$ for
        all $x, y, z$ in $F$.
      \item There is a unique element $1$ (one) in $F$ such that
        $1 \neq 0$ and $1x = x$ for every $x$ in $F$.
      \item For each element $x \neq 0$ in $F$ there is a unique
        element $x^{-1}$ (or $1/x$) in $F$ such that $xx^{-1} =
        1$.
      \item Multiplication distributes over addition: $x(y + z) =
        xy + xz$ for all $x, y, z$ in $F$.
  \end{enumerate}
\end{defn}

\begin{exm}
  The set of natural numbers $\mathbb{N}$ is not a field. $0$ is
  not a natural number, nor is $-n$ for any natural number $n$,
  nor $1/n$ (except in the case of $1$).
\end{exm}

\begin{exm}
  The set of integers $\mathbb{Z}$ is not a field. It comes
  close, but for an integer $n$, $1/n$ is not an integer unless
  $n = 1$ or $n = -1$.
\end{exm}

\begin{exm}
  The set of rational numbers $\mathbb{Q}$ is a field.
\end{exm}

\begin{exm}
  The set of real numbers $\mathbb{R}$ is a field.
\end{exm}

\begin{exm}
  The set of complex numbers $\mathbb{C}$ is a field.
\end{exm}

\begin{comm}
  In any field, one usually writes
  \begin{align*}
    x - y,\ \frac{x}{y},\ x + y + z,\ xyz,\ x^2,\ x^3,\ 2x,\ 3x,\ldots
  \end{align*}
  in place of
  \begin{align*}
    x + (-y),\ x \cdot \frac{1}{y},\ (x + y) + z,\ (xy)z,\ xx,\ xxx,\ x + x,\ x + x + x,\ldots
  \end{align*}
\end{comm}

\begin{thm} \label{thm:fieldaxforadd}
  The axioms for addition imply that
  \begin{enumerate}
    \item
      if $x + y = x + z$ then $y = z$,
    \item
      if $x + y = x$ then $y = 0$,
    \item
      if $x + y = 0 $ then $y = -x$, and
    \item
      $-(-x) = x$.
  \end{enumerate}

  \begin{proof}
    Let $F$ be a field such that $x,y \in F$. Let $a = x + y$ and $b = x + z$; then
    $a,b \in F$ and $a = b$, so $a-x,\ b-x \in F$ and $a - x = b - x$. Then $x + y -
    x = x + z - x$, so $x - x + y = x - x + z$, thus $0 + y = 0 + z$ and thus $y = z$.

    Letting $z = 0$ in (1) gives $x + y = x + 0$, which implies $y = 0$ by (1).
    Likewise, letting $z = -x$ in (1) gives $x + y = x - x = 0$, which implies $y =
    -x$ by (1). Last, letting $x = -x$ in (3) gives $y = -(-x)$, and if $-x - (-x) =
    0$, then $x - x - (-x) = x = -(-x)$.
  \end{proof}
\end{thm}

\begin{thm}
  The axioms for mulitplication imply that
  \begin{enumerate}
    \item
      if $x \neq 0$ and $xy = xz$, then $y = z$,
    \item
      if $x \neq 0$ and $xy = x$ then $y = 1$,
    \item
      if $x \neq 0$ and $xy = 1$ then $y = 1/x$, and
    \item
      if $x \neq 0$ then $1/(1/x) = x$.
  \end{enumerate}

  \begin{proof}
    Let $F$ be a field such that $x,y \in F$, and let $x \neq 0$. Let $a = xy$ and $b
    = xz$; then $a,b \in F$ and $a = b$, so $a/x,\ b/x \in F$ and $a/x = b/x$. Then
    $(xy)/x = (xz)/x$, so $y \cdot x/x = z \cdot x/x$, thus $y \cdot 1 = z \cdot 1$
    and thus $y = z$.

    Letting $z = 0$ in (1) gives $xy = x0$, which implies $y = 0$ by (1).  Likewise,
    letting $z = 1/x$ in (1) gives $xy = x(1/x) = 1$, which implies $y = 1/x$ by (1).
    Last, letting $x = 1/x$ in (3) gives $y = 1/(1/x)$, and if $1/x \cdot 1/(1/x) =
    1$, then $x \cdot 1/x \cdot 1/(1/x) = x \cdot 1 = 1 \cdot 1/(1/x) = 1/(1/x) = x$.
  \end{proof}
\end{thm}

\begin{thm}
  Let $F$ be a field. The field axioms imply that, for any $x,y \in F$,
  \begin{enumerate}
    \item
      $0x = 0$,
    \item
      if $x \neq 0$ and $y \neq 0$ then $xy \neq 0$,
    \item
      $(-x)y = -(xy) = x(-y)$, and
    \item
      $(-x)(-y) = xy$.
  \end{enumerate}

  \begin{proof}
    $0x = [x + (-x)]x = xx + (-xx) = 0$, which proves (1).

    If $x \neq 0$ and $y \neq 0$, there exist elements $1/x \in F$ and $1/y \in F$
    such that $x \cdot (1/x) = 1$ and $y \cdot (1/y) = 1$. Then $x(1/x)y(1/y) = 1(1)
    = 1 = xy(1/x)(1/y)$. By (1), $0 \cdot (1/x)(1/y) = 0$, and $1 \neq 0$, so $xy
    \neq 0$.

    $(-x)y + xy = y(-x) + yx = y(-x + x) = y0 = 0$. Therefore $(-x)y = -(xy)$.
    Likewise $x(-y) + xy = x(-y + y) = x0 = 0$, so $(-x)y = -(xy) = x(-y)$.

    If $(-x)y = -(xy)$, then $(-x)(-y) = -[x(-y)] = -[-(xy)] = xy$.
  \end{proof}
\end{thm}

\begin{defn}
  An \textbf{ordered field} is a field $F$ which is also an ordered set, with the
  order
  \begin{enumerate}
    \item
      $x + y < x + z$ if $x,y,z \in F$ and $y < z$, and
    \item
      $xy > 0$ if $x,y \in F$, $x > 0$, and $y > 0$.
  \end{enumerate}
  If $x > 0$, we say that $x$ is \textbf{positive}, and if $x < 0$, we say that $x$
  is \textbf{negative}.
\end{defn}

\begin{exm}
  $\mathbb{Q}$ is an ordered field.
\end{exm}

\begin{thm}
  The following is true in every ordered field:
  \begin{enumerate}
    \item
      if $x > 0$, then $-x < 0$ and vice versa,
    \item
      if $x > 0$ and $y < z$ then $xy < xz$,
    \item
      if $x < 0$ and $y < z$ then $xy > xz$,
    \item
      if $x \neq 0$ then $x^2 > 0$, and in particular $1 > 0$, and
    \item
      if $0 < x < y$, then $0 < 1/y < 1/x$.
  \end{enumerate}

  \begin{proof}
    If $x > 0$, then $-x + 0 < -x + x$, i.e. $-x < 0$. Likewise if $x < 0$, then $-x
    + x < -x + 0$, e.g. $0 < -x$. Also if $-x < 0$, then $x - x < x + 0$, i.e. $0 <
    x$, and if $0 < -x$, then $x + 0 < x - x$, i.e. $x < 0$.

    If $y < z$, then $z > y$, so $z - y > y - y = 0$. Then if $x > 0$, $x(z - y) >
    0$, so $xz - xy > 0$, e.g. $xz > xy$, so $xy < xz$. If $x < 0$, then $x(z - y) <
    0$, so $xz - xy < 0$, e.g. $xz < xy$, so $xy > xz$.

    Assume that $x > 0$. Then $xx = x^2 > 0$. Now assume that $x < 0$; then $xx = x^2
    > x0 = 0$.

    If $x > 0$, then if $v \leq 0$, $xv \leq 0$. However, $x(1/x) = 1 > 0$, so
    therefore $1/x > 0$. $1/y > 0$ by the same logic. Then if $0 < x < y$, $0 <
    x(1/x)(1/y) = 1/y < y(1/y)(1/x) = 1/x$.
  \end{proof}
\end{thm}

\begin{defn}
  A \textbf{subfield} of the field $F$ is a set $G \subseteq F$
  such that $G$ is a field under the same definitions of addition
  and mulitplication used with $F$.
\end{defn}

\begin{comm}
  If $G$ is a subfield of $F$, $0$ and $1$ are in $G$, and if $x,
  y$ are in $G$, then $(x + y)$, $xy$, $-x$, and $x^{-1}$ (if
  $x \neq 0$) are also in $G$.
\end{comm}

\begin{thm} \label{thm:rlubsubq}
  There exists an ordered field $\mathbb{R}$ which has the least-upper-bound
  property, and which contains $\mathbb{Q}$ as a subfield.

  \begin{proof}
    This proof follows the strategy of Richard Dedekind, a German mathematician who
    lived primarily during the 19th century. This method of constructing the real
    numbers is perhaps what he is best known for; he published his version in 1872.

    In the following discussion, we define \say{$\subset$} to mean \say{is a proper
    subset of,} and \say{$\subseteq$} to mean \say{is a subset of.} In other words,
    if $\alpha \subset \beta$, then $\alpha \neq \beta$, while if $\alpha \subseteq
    \beta$, it is possible that $\alpha = \beta$.

    We define $\mathbb{R}$ to be a set of objects called \textit{cuts}, and we define
    a cut to be a set $\alpha \subset \mathbb{Q}$ with the following properties.
    \begin{enumerate}
      \item
        $\alpha \neq \emptyset$ and $\alpha \neq \mathbb{Q}$ (the latter is implied
        by $\alpha \subset \mathbb{Q}$ but it's good to remember).
      \item
        If $p \in \alpha$, $q \in \mathbb{Q}$, and $q < p$, then $q \in \alpha$.
      \item
        If $p \in \alpha$, then $p < r$ for some $r \in \alpha$.
    \end{enumerate}

    Throughout this discussion, $p,q,r,\ldots$ will denote members of $\mathbb{Q}$,
    and $\alpha,\beta,\gamma,\ldots$ will denote cuts.

    (3) implies that $\alpha$ has no largest member. (2) implies the following:
    \begin{itemize}
      \item
        If $p \in \alpha$ and $q \notin \alpha$, then $p < q$; otherwise $\alpha$
        would be equal to $\mathbb{Q}$.
      \item
        If $r \notin \alpha$ and $r < s$, then $s \notin \alpha$.
    \end{itemize}

    We define \say{$\alpha < \beta$} to mean $\alpha \subset \beta$. This fulfills
    the criteria for an order. If $\alpha < \beta$ and $\beta < \gamma$, then $\alpha
    < \gamma$, because a proper subset of a proper subset of a set is a proper subset
    of that set. Also, at most one of
    \begin{align*}
      \alpha < \beta,\ \ \alpha = \beta,\ \ \beta < \alpha
    \end{align*}
    can hold for any pair $\alpha,\beta$. If $\alpha \nless \beta$ and $\alpha \neq
    \beta$, $\alpha$ is not a subset of $\beta$, so there is some $p \in \alpha$ such
    that $p \notin \beta$. If $q \in \beta$, then $q < p$, so $q \in \alpha$ by (2).
    Therefore, $\beta \subset \alpha$, and since $\beta \neq \alpha$, $\beta <
    \alpha$. This implies that if two of the above relations do not hold, the other
    one does. Thus, with this order, $\mathbb{R}$ is an ordered set.

    We will now show that $\mathbb{R}$ has the least-upper-bound property.

    Let $A \subset \mathbb{R}$ be such that $A \neq \emptyset$. Let $\beta \in
    \mathbb{R}$ be an upper bound of $A$. Let $\gamma$ be the union of all $\alpha
    \in A$, i.e. $p \in \gamma$ if and only if $p \in \alpha$ for some $\alpha \in
    A$. Then $\gamma \in \mathbb{R}$ and $\gamma = \sup A$, as we will show.

    Since $A$ is not empty, there exists some $\alpha_0 \in A$ that is itself not
    empty. Since $\alpha_0 \subseteq \gamma$, $\gamma$ is not empty. Also, $\gamma
    \subseteq \beta$, as $\alpha \subseteq \beta$ for every $\alpha \in A$ by our
    order for $\mathbb{R}$. Therefore $\gamma \neq \mathbb{Q}$, and thus $\gamma$
    satisfies property (1). Now let $p \in \gamma$; then $p \in \alpha_1$ for some
    $\alpha_1 \in A$. Let $q \in \mathbb{Q}$; if $q < p$, then $q \in \alpha_1$, and
    so $q \in \gamma$. This proves (2). Now let $r \in \alpha_1$ be such that $r >
    p$; then $r \in \gamma$ as $\alpha_1 \subseteq \gamma$, so $\gamma$ satisfies
    (3). Therefore $\gamma \in \mathbb{R}$.

    Now let $\delta < \gamma$; there is then some $s \in \gamma$ such that $s \notin
    \delta$. Since $s \in \gamma$, $s \in \alpha$ for some $\alpha \in A$. As such,
    $\delta < \alpha$, so $\delta$ is not an upper bound of $A$. Therefore $\alpha
    \leq \gamma$ for every $\alpha \in A$, i.e. $\gamma = \sup A$.

    This proves that $\mathbb{R}$ has the least-upper-bound property.

    If $\alpha \in \mathbb{R}$ and $\beta \in \mathbb{R}$, we define $\alpha + \beta
    = \{r \in \alpha,\ s \in \beta: r + s\}$. Before proceeding further, we must
    assure ourselves that this definition of addition in $\mathbb{R}$ satisfies the
    axioms of addition for a field.

    We define $0^* = \{p \in \mathbb{Q}: p < 0\}$. $0^*$ is clearly in $\mathbb{R}$;
    it is neither empty nor $\mathbb{Q}$, if $p \in 0^*$ and $q \in \mathbb{Q}$ such
    that $q < p$ then $q < 0$ and therefore $q \in 0^*$, and if $p = -\frac{n}{m}$
    where $n,m \in \mathbb{N}^+$, there is a number $r = -\frac{n}{2m}$ so that $r
    \in 0^*$ and $p < r < 0$.

    Now we need to show that $\alpha + \beta \in \mathbb{R}$.

    $\alpha + \beta \neq \emptyset$ because $\alpha,\beta \neq \emptyset$ by
    definition. Now let $r' \notin \alpha, s' \notin \beta$. Then $r' + s' > r + s$
    for all $r \in \alpha, s \in \beta$, as $r' > r$ and $s' > s$ for any $r \in
    \alpha, s \in \beta$. Then $r' + s' \notin \alpha + \beta$, so $\alpha + \beta
    \neq \mathbb{Q}$. Therefore $\alpha + \beta$ satisfies (1).

    Let $p \in \alpha + \beta$; then $p = r + s$ for some $r \in \alpha,\ s \in
    \beta$. If $q < p$, then $q < r + s$, so $q - s < r$ and thus $q - s \in \alpha$.
    Since $q = (q - s) + s$, $q \in \alpha + \beta$. Therefore $\alpha + \beta$
    satisfies (2). Now let $t \in \alpha$ such that $t > r$; then $p < t + s \in
    \alpha + \beta$, so $\alpha + \beta$ satisfies (3).

    Therefore $\alpha + \beta \in \mathbb{R}$.

    Because $\alpha + \beta$ is the set of all $r + s$ with $r \in \alpha,\ s \in
    \beta$, and $\beta + \alpha$ is the set of all $s + r$ defined likewise, $\alpha
    + \beta = \beta + \alpha$, as $r + s = s + r$ for all $r,s \in \mathbb{Q}$.
    Therefore addition in $\mathbb{R}$ is commutative.

    Likewise, because $(\alpha + \beta) + \gamma$ is the set of all $(r + s) + t$
    with $r \in \alpha,\ s \in \beta,\ t \in \gamma$, $\alpha + (\beta + \gamma)$
    is the set of all $r + (s + t)$ defined likewise, and $(r + s) + t = r + (s + t)$
    for all $r,s,t \in \mathbb{Q}$, addition in $\mathbb{R}$ is associative.

    Let $r \in \alpha$ and $s \in 0^*$. Then $r + s < r$, so $r + s \in \alpha$, and
    therefore $\alpha + 0^* \subseteq \alpha$. Now let $p \in \alpha$ such that $r >
    p$; then $p - r \in 0^*$, so $p = r + (p - r) \in \alpha + 0^*$. Therefore
    $\alpha \subseteq \alpha + 0^*$. Since $\alpha + 0^* \subseteq \alpha$ and
    $\alpha \subseteq \alpha + 0^*$, the only possibility is that $\alpha + 0^* =
    \alpha$.

    Let $\alpha \in \mathbb{R}$. Let $\beta$ be the set of all $p$ such that there is
    some $r > 0$ for which $-p - r \notin \alpha$; in other words, for which some
    rational number smaller that $-p$ is not in $\alpha$. Then $\beta \in \mathbb{R}$
    and $\alpha + \beta = 0^*$, as we will now show.

    Just in case you worry that this definition contradicts our definition of an
    element in $\mathbb{R}$, consider that if $\alpha < 0^*$, there is some $q < 0$
    such that $q \notin \alpha$, and then $-2q > 0$ and $-q - (-2q) = q \notin
    \alpha$ so $q \in \beta$. Likewise if $\alpha > 0^*$, there is some $q > 0$ such
    that $q \notin \alpha$, and then $-(-2q) - q = q \notin \alpha$, so $-2q \in
    \beta$. It works out either way.

    Anyway, let $s \notin \alpha$ and $p = -s - 1$. Then $-p -1 \notin \alpha$, as
    $-p = s + 1$ so $-p - 1 = s$, and so $p \in \beta$. Therefore $\beta \neq
    \emptyset$. If $q \in \alpha$, then $-q \notin \beta$, and thus $\beta \neq
    \mathbb{Q}$. Therefore $\beta$ satisfies (1).

    Let $p \in \beta$ and $r > 0$; then $-p -r \notin \alpha$. Let $q < p$; then $-q
    - r > -p - r$, so $-q - r \notin \alpha$. Therefore $q \in \beta$, so (2) holds
    for $\beta$. Now let $t = p + (r/2)$; then $t > p$, and $-t - (r/2) = -p - r
    \notin \alpha$, so $t \in \beta$. Therefore $\beta$ satisfies (3).

    Thus $\beta \in \mathbb{R}$.

    If $r \in \alpha$ and $s \in \beta$, then $-s \notin \alpha$, so $r < -s$, and
    thus $r + s < 0$. Therefore $\alpha + \beta \subseteq 0^*$.

    Let $v \in 0^*$ and $w = -v/2$. Then $w > 0$, and there is some integer $n$ such
    that $nw \in \alpha$ but $(n + 1)w \notin \alpha$. (This is because $\mathbb{Q}$
    is archimedean, i.e. there is no $q \in \mathbb{Q}$ such that $|q| \geq x$ for
    all $x \in \mathbb{Q}$.) Let $p = -(n + 2)w$; then
    \begin{align*}
      -p - w &= -[-(n + 2)w] - w\\
      &= nw + 2w - w\\
      &= nw + w\\
      &= (n + 1)w\\
      &\notin \alpha,
    \end{align*}
    so $p \in \beta$. If $w = -v/2$, then $v = -2w$;
    \begin{align*}
      nw + p &= nw - (n + 2)w\\
      &= nw - nw - 2w\\
      &= -2w\\
      &= v,
    \end{align*}
    and $nw + p$ i.e.  $v \in \alpha + \beta$, so $0^* \subseteq \alpha + \beta$.

    Therefore $0^* \subseteq \alpha + \beta$ and $\alpha + \beta \subseteq 0^*$, so
    $\alpha + \beta = 0^*$. As such, we will denote this $\beta$ by $-\alpha$.

    We have shown that $\mathbb{R}$ as defined satisfies the field axioms for
    addition. We can also show that if $\alpha,\beta,\gamma \in \mathbb{R}$ and
    $\beta < \gamma$, then $\alpha + \beta < \alpha + \gamma$; if $\alpha + \beta =
    \alpha + \gamma$, we would have $\beta = \gamma$ by theorem
    \ref{thm:fieldaxforadd}. This fulfills half of a proof that $\mathbb{R}$ is an
    ordered field. Also, if $\alpha > 0^*$, then $-\alpha < 0^*$; $-\alpha + 0^* <
    -\alpha + \alpha$ under our definitions.

    We now move on to multiplication. The fact that products of negative rationals
    are positive complicates multiplication in comparison to addition, so we start
    first with $\mathbb{R}^+$, i.e. the set of all $\alpha \in \mathbb{R}$ such that
    $\alpha > 0^*$.

    If $\alpha \in \mathbb{R}^+$ and $\beta \in \mathbb{R}^+$, we define
    $\alpha\beta$ to be the set of all $p$ such that $p \leq rs$ for all $r \in
    \alpha,\ s \in \beta$ such that $r,s > 0$.

    We define $1^* = \{q \in \mathbb{Q}: q < 1\}$. Clearly $1^*$ is neither empty nor
    $\mathbb{Q}$. If $p \in 1^*$ i.e. $p < 1$, and $p > q$ for some $q \in
    \mathbb{Q}$, then $q < p < 1$, so $q \in 1^*$.

    If $p < 1$, then $p = n/m$ where $n,m \in \mathbb{N}$, $m \neq 0$, and $|n| <
    |m|$ Let $r \in \mathbb{Q}$ be such that $r = a/b$, where $a = |n| + 1$ and $b =
    |m| + 1$. If $|n| < |m|$, then $|n| + 1 < |m| + 1$, so $r < 1$. $|n|(|m| + 1) =
    |n||m| + |n|$ and $|m|(|n| + 1) = |n||m| + |m|$; $|n||m| + |n| < |n||m| + |m|$,
    so $|n| < \frac{|n||m| + |m|}{|m| + 1}$ and therefore $\frac{|n|}{|m|} <
    \frac{|n| + 1}{|m| + 1}$. Since $n/m \leq |n|/|m|$, $p < r$, and since $|n| + 1 <
    |m| + 1$, $r \in 1^*$.

    Therefore $1^* \in \mathbb{R}$.

    $\alpha\beta \neq \emptyset$ because $\alpha,\beta \neq \emptyset$. Since
    $\alpha,\beta \neq \mathbb{Q}$, there is some $r' \notin \alpha$ and $s' \notin
    \beta$ such that $rs < r's'$. Then if $t = r's'$, $t \notin \alpha\beta$, so
    $\alpha\beta \neq \mathbb{Q}$. This satisfies (1).

    If $p \in \alpha\beta$, $q < p$ for some $q \in \mathbb{Q}$. Then $q < rs$ for
    some $r \in \alpha,\ s \in \beta$ and therefore $q \in \alpha\beta$. This
    satisfies (2). Let $t \in \alpha$ be such that $t > r$; then $ts \in \alpha\beta$
    and $q < t$, which satisfies (3).

    Therefore $\alpha\beta \in \mathbb{R}$.

    Since $\alpha\beta = \{p \in \mathbb{Q},\ r \in \alpha,\ s \in \beta,\ r,s > 0: p
    \leq rs\}$ and $\beta\alpha = \{p \in \mathbb{Q},\ r \in \alpha,\ s \in \beta,\ r,s > 0: p
    \leq sr\}$, $\alpha\beta = \beta\alpha$, i.e. because multiplication in
    $\mathbb{Q}$ is commutative. Likewise if $\gamma \in \mathbb{R}^+$,
    $(\alpha\beta)\gamma = \{p \in \mathbb{Q},\ r \in \alpha,\ s \in \beta,\ t \in
    \gamma,\ r,s,t > 0: p \leq (rs)t = rst\}$ and $\alpha(\beta\gamma) = \{p \in
    \mathbb{Q},\ r \in \alpha,\ s \in \beta,\ t \in \gamma,\ r,s,t > 0: p \leq r(st)
    = rst\}$, $(\alpha\beta)\gamma = \alpha(\beta\gamma)$, i.e. because
    multiplication in $\mathbb{Q}$ is associative.

    $1^* \neq 0^*$ because there are elements $p \in 1^*$ such that $0 < p < 1$.
    Since every $q \in 0^*$ is such that $q < 0 < 1$, $0^* < 1^*$, so $0^* \neq 1^*$.

    Let $r \in \alpha$ and $s \in 1^*$. Then $rs < r$, since $r > 0$ and $s < 1$.
    Therefore $rs \in \alpha$, so $1^*\alpha \subseteq \alpha$. Now let $p \in
    \alpha$ such that $r > p$; then $p/r \in 1^*$, so $p = r \cdot p/r \in
    1^*\alpha$. Therefore $\alpha \subseteq 1^*\alpha$, and thus $1^*\alpha =
    \alpha$.

    Fix $\alpha \in \mathbb{R}^+$, and let $\beta$ be the set of all $q$ such that,
    for some rational number $r > 0$, $q^{-1} - r \notin \alpha$. Let $s \notin
    \alpha$, and let $t = \frac{1}{s + 1}$. Then $t^{-1} - 1 \notin \alpha$, so $t
    \in \beta$ and thus $\beta$ is not empty. Now let $u \in \alpha$ be such that $u
    > 0$. Then $u^{-1} > 0$, and $u^{-1} - u^{-1} = 0 \in \alpha$, so therefore $u
    \notin \beta$, and as such $\beta \neq \mathbb{Q}$. As such, $\beta$ satisfies
    (1).

    Now let $p \in \beta$ be such that $p > 0$, and let $q < p$ for some $q > 0$.
    Since $p$ is in $\beta$, we know that $p^{-1} - r \notin \alpha$ for some $r >
    0$. Let $s$ then be such that $p^{-1} - s \notin \alpha$. Since $q < p$, $q^{-1}
    > p^{-1}$, so then $q^{-1} - s > p^{-1} - s$, and therefore $q^{-1} - s \notin
    \alpha$, so $q \in \beta$. As such, $\beta$ satisfies (2). Let $a = p^{-1} - s$;
    then $a \notin \alpha$ and $p = \frac{1}{a + s}$. Let $t = \frac{1}{a +
    \frac{s}{2}}$; then $t > p$, and $t^{-1} - \frac{s}{2} = a + \frac{s}{2} -
    \frac{s}{2} = a$. Since $a \notin \alpha$ by definition, $t \in \beta$. Thus
    $\beta$ satisfies (3), and therefore $\beta \in \mathbb{R}$.

    Let $u \in \alpha$ and $v \in \beta$ such that $u,v > 0$. Then $v^{-1} \notin
    \alpha$, so $u < v^{-1}$ and therefore $uv < 1$; thus $\alpha\beta \subseteq
    1^*$.

    Now let $w \in 1^*$ be such that $w > 0$,

    Actually I'm going to put this aside for now as it's taking \textit{forever} and
    the book suggests you can get to this material \say{whenever the time seems
    ripe.} I will come back to it as I'd like to finish what I've started—goodness
    gracious though!
  \end{proof}
\end{thm}

\begin{thm} \label{thm:rarchqdenseinr}
  \
  \begin{enumerate}
    \item
      If $x,y \in \mathbb{R}$ and $x > 0$, there is a positive integer $n$ such that
      $nx > y$. In other words, $\mathbb{R}$ is archimedean.
    \item
      If $x,y \in \mathbb{R}$ and $x < y$, there is a $p \in \mathbb{Q}$ such that $x
      < p < y$. In other words, $\mathbb{Q}$ is dense in $\mathbb{R}$.
  \end{enumerate}

  \begin{proof}
    \
    \begin{enumerate}
      \item
        Let $A = \{n \in \mathbb{Z}^+,\ x \in \mathbb{R}^+: nx\}$. Then $A \subset
        \mathbb{R}$ and $A$ is not empty (e.g. $1x = x \in A$). Assume (1) is false;
        then some $y \in \mathbb{R}$ is an upper bound of $A$. Because $\mathbb{R}$
        has the least-upper-bound property, there is then some $\alpha = \sup A \in
        \mathbb{R}$. Since $x > 0$, $\alpha - x < \alpha$, so $\alpha - x$ is not an
        upper bound of $A$. As such, $\alpha - x < mx$ for some $m \in \mathbb{Z}^+$.
        But then $\alpha - x + x = \alpha < mx + x = (m + 1)x = px$, where $p = m +
        1$ and thus $p > 0$. This implies that $\alpha$ is not an upper bound of $A$,
        which is contradictory, and thus (1) must be true.

      \item
        Since $x < y$, $y - x > 0$. As such, there is some positive integer $n$ such
        that $n(y - x) > 1$ by (1).  Also by (1), there are positive integers $m_1$
        and $m_2$ such that $m_1 > nx$ and $m_2 > -nx$; $nx$, $-nx$, and $1$ are real
        numbers and $m_1,m_2$ are positive, so $1m_1 > nx$ exists and so on. If $m_2
        > -nx$, then $-m_2 < nx$, so we then have $-m_2 < nx < m_1$.

        Consider an integer $m$ such that $-m_2 \leq m - 1 \leq nx < m \leq m_1$.
        Then $m \leq nx + 1$. Furthermore, if $n(y - x) > 1$, then $ny - nx > 1$, so
        $ny > nx + 1$. Therefore $nx < m \leq nx + 1 < ny$.

        Since $n > 0$ by definition, we then have $x < \frac{m}{n} \leq x +
        \frac{1}{n} < y$, i.e. $x < \frac{m}{n} < y$. Since $m$ and $n$ are integers,
        $\frac{m}{n} \in \mathbb{Q}$. We can thus say $p = \frac{m}{n}$; in full,
        this proves (2).
    \end{enumerate}
  \end{proof}
\end{thm}

\begin{thm} \label{thm:uniqrealroots}
  For every real $x > 0$ and every integer $n > 0$ there is one and only one positive
  real $y$ such that $y^n = x$. ($y$ is generally written as $\sqrt[n]{x}$ or
  $x^{1/n}$.)

  \begin{proof}
    Let $y_1,y_2 \in \mathbb{R}$ such that $0 < y_1 < y_2$; then $y_1^n < y_2^n$.
    This implies that if $y^n = x$ and $z^n = x$ for positive real numbers $y,z$,
    then $y = z$, i.e. there can only be one such $y$ at most. Therefore our goal is
    to show that $y^n = x$ exists for all integers $n > 0$.

    Let $E = \{t \in \mathbb{R}: 0 < t^n < x\}$. If $t = \frac{x}{1 + x}$, then $0
    < t < 1$, and then $0 < t^n \leq t < x$. Thus $t \in E$, so $E$ is not empty.

    If $u > 1 + x$, then $x < u \leq u^n$, and thus $u \notin E$. Therefore $1 + x$
    is an upper bound of $E$. By theorem \ref{thm:rlubsubq}, there is thus a positive
    real number $y = \sup E$.

    We assert that $y^n = x$. We will prove this by contradiction, showing that the
    inequalities $y^n < x$ and $y^n > x$ are both contradictory.

    Assuming that $y^n < x$, we would like to find some $h > 0$ such that $(y + h)^n
    < x$. That would imply that $y + h \in E$; since $y + h > y$, this would
    contradict our assumption that $y = \sup E$.

    Consider the identity $b^n - a^n = (b - a)(b^{n - 1} + b^{n - 2}a + \cdots +
    ba^{n - 2} + a^{n - 1})$. If $0 < a < b$, we then have $a^n < b^n$ for any
    integer $n > 0$. In that case,
    \begin{align*}
      b^{n - 1} + b^{n - 2}a + \cdots + ba^{n - 2} + a^{n - 1}
      &\leq b^{n - 1} + b^{n - 2}b + \cdots + bb^{n - 2} + b^{n - 1}\\
      &= b^{n - 1} + b^{n - 1} + \cdots + b^{n - 1} + b^{n - 1}\\
      &= nb^{n - 1}.
    \end{align*}
    (If it seems less-than-obvious that there are $n$ terms in $b^{n - 1} + b^{n - 2}a
    + \cdots + ba^{n - 2} + a^{n - 1}$, note that the coefficients range from
    $b^0$ to $b^{n - 1}$ and from $a^0$ to $a^{n - 1}$.)

    This implies that
    \begin{align*}
      b^n - a^n \leq (b - a)nb^{n-1}
    \end{align*}
    when $0 < a < b$.

    If we put $a = y$ and $b = y + h$ for some real number $h > 0$, we then have $(y
    + h)^n - y^n \leq ((y + h) - y)n(y + h)^{n-1} = hn(y + h)^{n-1}$. To be
    satisfactory, $h$ must also be such that $(y + h)^n < x$, i.e. $(y + h)^n - y^n <
    x - y^n$. If a suitable $h$ exists, it can be as small as we need it to be as
    long as it's positive, so we say hypothetically that $(y + h)^n - y^n \leq hn(y +
    h)^{n-1} < x - y^n$.

    In theory, we could try to isolate $h$ in the expression $hn(y + h)^{n-1} < x -
    y^n$ to get its value. However, this is made difficult by the fact that it
    appears twice on the left side of the inequality. Since again $h$ can be as small
    as we need it to be as long as it does its job, we say that $0 < h < 1$, so that
    $hn(y + h)^{n-1} < hn(y + 1)^{n-1}$. Then $hn(y + 1)^{n-1} < x - y^n$, so
    \begin{align*}
      h < \frac{x - y^n}{n(y + 1)^{n-1}}.
    \end{align*}

    As long as we pick $h > 0$ which satisfies this inequality, we then have that $(y
    + h)^n - y^n \leq hn(y + h)^{n-1} < hn(y + 1)^{n-1} < x - y^n$. Then $(y + h)^n <
    x$, so $y + h \in E$. Since $y + h > y$, this contradicts our assumption that $y
    = \sup E$, and therefore it must be the case that $y^n \nless x$.

    Now assume that $y^n > x$. We would then like to find some real number $k$ such
    that $0 < k < y$ and $(y - k)^n > x$. This would imply that $y - k$ was an upper
    bound of $E$; since $y - k < y$, this would contradict our assumption that $y =
    \sup E$.

    If $y^n > (y - k)^n > x$, then $y^n - (y - k)^n < y^n - x$. Since $0 < y - k < y$, we
    have furthermore that $y^n - (y - k)^n < kny^{n-1}$. Say $k$ is such that $y^n -
    x = kny^{n-1}$; then $k$ satisfies both inequalities and
    \begin{align*}
      k = \frac{y^n - x}{ny^{n-1}}.
    \end{align*}

    Take $t \geq y - k$. Then $y^n - t^n \leq y^n - (y - k)^n < kny^{n-1} = y^n - x$;
    this implies that $t^n > x$ and thus that $t \notin E$, which implies that $y -
    k$ is an upper bound of $E$. However, $y - k < y$, so this contradicts our
    assumption that $y = \sup E$. Thus, $y^n \ngtr x$.

    The only remaining possibility is that $y^n = x$, which completes our proof.
  \end{proof}
\end{thm}

\begin{cor}
  If $a$ and $b$ are positive real numbers and $n$ is a positive integer, then
  \begin{align*}
    (ab)^{1/n} = a^{1/n}b^{1/n}.
  \end{align*}

  \begin{proof}
    Let $\alpha = a^{1/n}$ and $\beta = b^{1/n}$. Because multiplication in a field
    is commutative, we have
    \begin{align*}
      ab = \alpha^n\beta^n = \alpha \cdots \alpha \cdot \beta \cdots \beta =
      \alpha\beta \cdots \alpha\beta = (\alpha\beta)^n.
    \end{align*}
    By theorem \ref{thm:uniqrealroots}, there is only one real $x$ such that $ab =
    x^n$. Therefore if $ab = (\alpha\beta)^n$,
    \begin{align*}
      (ab)^{1/n} = (\alpha\beta)^{n/n} = \alpha\beta = a^{1/n}b^{1/n}.
    \end{align*}
  \end{proof}
\end{cor}

\begin{comm}
  This is a brief discussion of decimals as they relate to real numbers.

  Let $x > 0$ be real, and let $n_0$ be the largest integer such that $n_0 \leq x$.
  (This is possible by theorem \ref{thm:rarchqdenseinr}.1.) Now choose integers
  $n_1,\cdots,n_{k-1}$, and then let $n_k$ be the largest integer such that
  \begin{align*}
    n_0 + \frac{n_1}{10} + \cdots + \frac{n_k}{10^k} \leq x.
  \end{align*}

  Let $E$ be the set of all such numbers
  \begin{align*}
    n_0 + \frac{n_1}{10} + \cdots + \frac{n_k}{10^k}\ \ (k = 0,1,2,\ldots)
  \end{align*}
  $\leq x$; then $x = \sup E$. The decimal expansion of $x$ is then
  \begin{align*}
    n_0\ .\ n_1n_2n_3\cdots.
  \end{align*}
  Because $E$ is bounded above, this represents the decimal expansion of $\sup E$
  even if it is infinite.
\end{comm}

\begin{defn}
  The \textbf{magnitude}, or \textbf{absolute value}, of a real number $x$ is given
  by
  \begin{align*}
    |x| =
    \begin{cases}
      x\ &\text{if}\ x \geq 0\\
      -x\ &\text{if}\ x < 0.
    \end{cases}
  \end{align*}
  Therefore $x \leq |x|$ as we would hope.
\end{defn}

\begin{thm}
  For all $x,y \in \mathbb{R}$, $|x + y| \leq |x| + |y|$. This is known as the
  \textbf{triangle inequality}.

  \begin{proof}
    We know that $x \leq |x|$ and $-x \leq |x|$. Then
    \begin{align*}
      x + y &\leq |x| + y \leq |x| + |y|
      \shortintertext{and}
      -x - y &\leq |x| - y \leq |x| + |y|.
    \end{align*}
    If $x + y < 0$, then $|x + y| = -(x + y) = -x - y$; otherwise $|x + y| = x + y$.
    In either case, we have shown that $|x + y| \leq |x| + |y|$.
  \end{proof}
\end{thm}

\begin{exm}
  The field $\mathbb{R}$ is a subfield of $\mathbb{C}$.

  We can represent a real number $x$ as a complex number $(x +
  0i)$. This means that the real numbers $1$ and $0$ are in
  $\mathbb{C}$. It also means that if $x, y$ are real numbers,
  $(x + y)$, $xy$, $-x$, and (if $x \neq 0$) $x^{-1}$ are in
  $\mathbb{C}$.
\end{exm}

\begin{thm}
  Any subfield of $\mathbb{C}$ must contain every rational
  number.
  \begin{proof}
    By contradiction. Assume that $F$ is a subfield of
    $\mathbb{C}$ which omits the rational number $p$, $p \neq 0$.
    We know that for any rational number $q$, there is a rational
    number $r$ such that $qr = p$. This is because $qq^{-1} = 1$,
    and thus $qq^{-1}p = p$; therefore $r = q^{-1}p$. For $F$ to
    be a field, $qr$ would need to produce a result in $F$; since
    $F$ omits $p$, $q$ and $r$ must be omitted as well. But this
    ultimately leaves no rational numbers in $F$ at all, and thus
    $F$ cannot be a field in any sense, let alone a subfield of
    $\mathbb{C}$.
  \end{proof}
\end{thm}

\begin{thm}
  The set of all complex numbers of the form $x + y\sqrt{2}$,
  where $x$ and $y$ are rational, is a subfield of $\mathbb{C}$.

  \begin{proof}
    By lemmas 2.*.
  \end{proof}
\end{thm}

\begin{lemma}
  If $p,q,r,s$ are rational numbers, then $(p + q\sqrt{2}) + (r
  + s\sqrt{2})$ is a complex number of the form $x +
  y\sqrt{2}$ where $x, y$ are rational.
  \begin{proof}
    \begin{align*}
      (p + q\sqrt{2}) + (r + s\sqrt{2}) &=\\
      p + q\sqrt{2} + r + s\sqrt{2} &=\\
      p + r + q\sqrt{2} + s\sqrt{2} &=\\
      (p + r) + (q + s)\sqrt{2}&.
    \end{align*}
    Then $(p + q\sqrt{2}) + (r + s\sqrt{2})$ is a complex number
    of the form $x + y\sqrt{2}$, $x = (p + q)$, $y = (r + s)$.
  \end{proof}
\end{lemma}

\begin{lemma}
  If $p,q,r,s$ are rational numbers, then $(p + q\sqrt{2})
  \cdot (r + s\sqrt{2})$ is a complex number of the form $x +
  y\sqrt{2}$ where $x, y$ are rational.
  \begin{proof}
    \begin{align*}
      (p + q\sqrt{2}) \cdot (r + s\sqrt{2}) &=\\
      pr + qr\sqrt{2} + ps\sqrt{2} + qs(\sqrt{2})^{2} &=\\
      (pr + 2qs) + (qr + ps)\sqrt{2}&.
    \end{align*}
    Then $(p + q\sqrt{2}) \cdot (r + s\sqrt{2})$ is a complex
    number of the form $x + y\sqrt{2}$, $x = (pr + 2qs)$, $y =
    (qr + ps)$.
  \end{proof}
\end{lemma}

\begin{lemma}
  If $p,q$ are rational numbers, then $-(p + q\sqrt{2})$ is a
  complex number of the form $x + y\sqrt{2}$ where $x, y$ are
  rational.
  \begin{proof}
    $-(p + q\sqrt{2}) = -p + (-q)\sqrt{2}$. Then $-(p +
    q\sqrt{2})$ is a complex number of the form $x + y\sqrt{2}$,
    $x = -p$, $y = -q$.
  \end{proof}
\end{lemma}

\begin{lemma}
  If $p,q$ are rational numbers and $p + q\sqrt{2} \neq 0$, then
  $(p + q\sqrt{2})^{-1}$ is a complex number of the form $x +
  y\sqrt{2}$ where $x, y$ are rational.
  \begin{proof}
    \begin{align*}
      (p + q\sqrt{2})^{-1} &=\\
      \frac{1}{p + q\sqrt{2}} &=\\
      \frac{p - q\sqrt{2}}{(p + q\sqrt{2})(p - q\sqrt{2})} &=\\
      \frac{p - q\sqrt{2}}{p^{2} - 2q^{2}} &=\\
      \frac{p}{p^{2} - 2q^{2}} - \frac{q}{p^{2} - 2q^{2}}\sqrt{2}&.
    \end{align*}
    Then $(p + q\sqrt{2})^{-1}$ is a complex number of the form
    $x + y\sqrt{2}$, $x = \frac{p}{p^{2} - 2q^{2}}$, $y =
    -\frac{q}{p^{2} - 2q^{2}}$.
  \end{proof}
\end{lemma}

\begin{defn}
  A \textbf{scalar} is an element of a field.
\end{defn}

\begin{comm}
  If $x, y$ are scalars from a subfield $F$ of $\mathbb{C}$, then
  the performance of addition, multiplication, subtraction, or
  division with $x$ and $y$ does not leave $F$.
\end{comm}

\begin{comm}
  In some fields, it is possible to add $1$ to itself a finite
  number of times and obtain $0$, i.e.
  \begin{center}
    $1 + 1 + … + 1 = 0.$
  \end{center}
  However, this is not the case for $\mathbb{C}$, nor any
  subfield of it.
\end{comm}

\begin{defn}
  If a field $F$ has the property that it is possible to add $1$
  to itself a finite number of times and obtain $0$ in it, the
  least $n$ such that the sum of $n$ $1$s yields $0$ is called
  the \textbf{characteristic} of $F$. If $F$ does not have this
  property, it is said to be of \textbf{characteristic zero}.
\end{defn}

\begin{comm}
  Often, when assuming that $F$ is a subfield of $\mathbb{C}$,
  what is desired is to ensure that $F$ is a field of
  characteristic zero.
\end{comm}

\end{document}
