\documentclass[12pt]{article}
\usepackage{mathtools}
\usepackage{amsthm}
\usepackage{amsfonts}
\usepackage{amssymb}
\usepackage{fontspec}
\usepackage{xfrac}
\usepackage{array}
\usepackage{siunitx}
\usepackage{gensymb}
\usepackage{enumitem}

\newcolumntype{L}{>{$}l<{$}}

\title{1.2 Exercises}
\author{Zoë Sparks}

\begin{document}

\theoremstyle{definition}

\sisetup{quotient-mode=fraction}
\newtheorem{thm}{Theorem}
\newtheorem*{nthm}{Theorem}
\newtheorem{sthm}{}[thm]
\newtheorem{lemma}{Lemma}[thm]
\newtheorem*{cor}{Corollary}
\newtheorem*{prop}{Property}
\newtheorem*{defn}{Definition}

\maketitle

\begin{enumerate}
    \item
      See 1.1, Theorem 2.

    \item
      We have the following system of linear equations:
      \begin{align*}
         x_1 - x_2 &= 0\\
        2x_1 + x_2 &= 0.
      \end{align*}
      This immediately yields the linear combination
      \begin{align*}
        x_1 + 2x_1 - x_2 + x_2 &=\\
        3x_1 &= 0,
      \end{align*}
      so
      \begin{align*}
        x_1 &= 0.
      \end{align*}
      If we multiply the top equation by $-2$, we get
      \begin{align*}
       -2x_1 + 2x_2 &= 0.
      \end{align*}
      This yields the linear combination
      \begin{align*}
        -2x_1 + 2x_1 + 2x_2 + x_2 &=\\
        2x_2 + x_2 &=\\
        3x_2 &= 0,
      \end{align*}
      which implies that
      \begin{align*}
         x_2 = 0
      \end{align*}
      as well.\\

      We now consider the other system:
      \begin{align*}
        3x_1 + x_2 &= 0;\\
        x_1 + x_2 &= 0.
      \end{align*}
      Subtracting the second equation from the first gives us
      \begin{align*}
        3x_1 - x_1 + x_2 - x_2 &=\\
        2x_1 &= 0,
      \end{align*}
      so $x_1 = 0$ in this case as well. Multiplying the second
      equation by $-3$ yields
      \begin{align*}
        -3x_1 - 3x_2 &= 0,
      \end{align*}
      which produces the linear combination
      \begin{align*}
        3x_1 - 3_x1 + x_2 - 3x_2 &=\\
        x_2 - 3x_2 &=\\
        -2x_2 &= 0,
      \end{align*}
      so $x_2 = 0$ here as well, and thus the two systems are
      equivalent.\\\\
      Since the only possible value for $x_1$ and $x_2$ in both
      systems is $0$, expressing each equation in each system as
      a linear combination of the equations in the other system
      is trivial. We can simply take scalars $c_1, c_2 = 1$,
      since every term evaluates to $0$.
      \begin{align*}
        (1(3x_1) + 1(x_2)) + (1(x_1) + 1(x_2)) &= 1(0) + 1(0)\\
        &= 0\\
        &= x_1 - x_2\\
        &= 2x_1 + x_2.\\
        \\
        (1(x_1) - 1(x_2)) + (1(2x_1) + 1(x_2)) &= 1(0) + 1(0)\\
        &= 0\\
        &= 3x_1 + x_2\\
        &= x_1 + x_2.
      \end{align*}

    \item
      We consider the system
      \[
      \begin{array}{rcrcrcl}
        -x_1           & + & x_2  & + & 4x_3           & = & 0\\
         x_1           & + & 3x_2 & + & 8x_3           & = & 0\\
        \frac{1}{2}x_1 & + & x_2  & + & \frac{5}{2}x_3 & = & 0.
      \end{array}
      \]
      We multiply the first and third equations by $2$ to
      eliminate $x_1$:
      \[
      \begin{array}{rcrcrcl}
        -2x_1 & + & 2x_2 & + & 8x_3 & = & 0\\
          x_1 & + & 3x_2 & + & 8x_3 & = & 0\\
          x_1 & + & 2x_2 & + & 5x_3 & = & 0.
      \end{array}
      \]
      \begin{align*}
        2x_2 + 3x_2 + 2x_2 + 8x_3 + 8x_3 + 5x_3 &= 7x_2 + 21x_3\\
        &= x_2 + 3x_3\\
        &= 0,
      \end{align*}
      so $x_2 = -3x_3$.\\\\
      Then we multiply the first equation by $3$ and the third
      equation by $-6$ to eliminate $x_2$:
      \[
      \begin{array}{rcrcrcl}
        -3x_1 & + & 3x_2 & + & 12x_3 & = & 0\\
          x_1 & + & 3x_2 & + &  8x_3 & = & 0\\
        -3x_1 & - & 6x_2 & - & 15x_3 & = & 0.
      \end{array}
      \]
      \begin{align*}
        -3x_1 + x_1 - 3x_1 + 12x_3 + 8x_3 - 15x_3 &= -5x_1 + 5x_3,\\
        &= x_1 - x_3,\\
        &= 0,
      \end{align*}
      so $x_1 = x_3$. This implies that $x_1 = -\frac{1}{3}x_2$,
      and thus that if $(x_1,x_2,x_3)$ is a solution then $x_1 =
      -\frac{1}{3}x_2 = x_3$.\\\\
      We now turn to the other system. We have
      \[
      \begin{array}{rcrcrcl}
        x_1 &   &     & - &  x_3 & = & 0\\
            &   & x_2 & + & 3x_3 & = & 0.\\
      \end{array}
      \]
      The equivalence of these systems is fairly obvious. If $x_1 - x_3
      = 0$, then $x_1 = x_3$; likewise, if $x_2 + 3x_3 = 0$ then
      $x_2 = -3x_3$.\\\\
      We now turn to expressing each equation in each system as a
      linear combination of the equations in the other
      system. We start with the first system.\\\\
      For $-x_1 + x_2 + 4x_3 = 0$, we use scalars $c_1, c_2, c_3
      = -1, 1, 2$:
      \begin{align*}
        (-1)x_1 + (1)x_2 + (2\cdot-1 + 2\cdot3)x_3 &=\\
        -x_1 + x_2 + (-2 + 6)x_3 &=\\
        -x_1 + x_2 + 4x_3 &= 0.
      \end{align*}
      For $x_1 + 3x_2 + 8x_3 = 0$, we use scalars $c_1, c_2, c_3
      = 1, 3, 4$:
      \begin{align*}
        (1)x_1 + (3)x_2 + (4\cdot-1 + 4\cdot3)x_3 &=\\
        x_1 + 3x_2 + (-4 + 12)x_3 &=\\
        x_1 + 3x_2 + 8x_3 &= 0.
      \end{align*}
      For $\frac{1}{2}x_1 + x_2 + \frac{5}{2}x_3 = 0$, we use scalars $c_1, c_2, c_3
      = \frac{1}{2}, 1, \frac{5}{4}$:
      \begin{align*}
        (\frac{1}{2})x_1 + (1)x_2 + (\frac{5}{4}\cdot-1 + \frac{5}{4}\cdot3)x_3 &=\\
        \frac{1}{2}x_1 + x_2 + (-\frac{5}{4} + \frac{15}{4})x_3 &=\\
        \frac{1}{2}x_1 + x_2 + \frac{5}{2}x_3 &= 0.
      \end{align*}
      Now for the other system.\\\\
      For $x_1 - x_3 = 0$, we use scalars $c_1, c_2, c_3
      = 2, 0, -\frac{2}{29}$:
      \begin{align*}
        (2\cdot-1 + 2\cdot1 + 2\cdot\frac{1}{2})x_1 +
        (0\cdot1 + 0\cdot3 + 0\cdot1)x_2 +
        (-\frac{2}{29}\cdot4 - \frac{2}{29}\cdot8 - \frac{2}{29}\cdot\frac{5}{2})x_3 &=\\
        (-2 + 2 + 1)x_1 + (0)x_2 + (-\frac{8}{29} - \frac{16}{29}
        - \frac{10}{58})x_3 &=\\
        x_1 + (-\frac{16}{58} - \frac{32}{29} - \frac{10}{58})x_3 &=\\
        x_1 + (-\frac{58}{58})x_3 &=\\
        x_1 - x_3 &= 0.
      \end{align*}
      For $x_2 + 3x_3 = 0$, we use scalars $c_1, c_2, c_3
      = 0, \frac{1}{5}, \frac{6}{29}$:
      \begin{align*}
        (0\cdot-1 + 0\cdot1 + 0\cdot\frac{1}{2})x_1 +
        (\frac{1}{5}\cdot1 + \frac{1}{5}\cdot3 + \frac{1}{5}\cdot1)x_2 +
        (\frac{6}{29}\cdot4 + \frac{6}{29}\cdot8 + \frac{6}{29}\cdot\frac{5}{2})x_3 &=\\
        (0)x_1 + (\frac{1}{5} + \frac{3}{5} + \frac{1}{5})x_2 +
        (\frac{24}{29} + \frac{48}{29} + \frac{30}{58})x_3 &=\\
        (\frac{5}{5})x_2 + (\frac{48}{58} + \frac{96}{58} + \frac{30}{58})x_3 &=\\
        (1)x_2 + (\frac{174}{58})x_3 &=\\
        x_2 + 3x_3 &= 0.
      \end{align*}

    \item
      I'm fairly positive I could solve this if I kept working at
      it, but it's really tedious, so I'm going to leave it for
      now.

      \noindent\rule{\textwidth}{1pt}

      First let's find the solutions to
      \begin{alignat*}{6}
        2x_1 & {}+{} & (-1 + i)x_2 & {} {} &       & {}+{} &  x_4 & {}={} & 0 &\\
             & {} {} &        3x_2 & {}-{} & 2ix_3 & {}+{} & 5x_4 & {}={} & 0 &.
      \end{alignat*}

      First we multiply the second equation by $(\frac{1}{3} -
      \frac{1}{3}i)$:
      \begin{align*}
        (\frac{1}{3} - \frac{1}{3}i)3x_2
          - (\frac{1}{3} - \frac{1}{3}i)2ix_3
          + (\frac{1}{3} - \frac{1}{3}i)5x_4 =&\\
        (1 - i)x_2
          - (\frac{2}{3} + \frac{2}{3}i)x_3
          + (\frac{5}{3} - \frac{5}{3}i)x_4.
      \end{align*}

      Then we add:
      \begin{align*}
        2x_1
          + (-1 + i)x_2 + (1 - i)x_2
          - (\frac{2}{3} + \frac{2}{3}i)x_3
          + x_4 + (\frac{5}{3} - \frac{5}{3}i)x_4 &= \\
        2x_1
          - (\frac{2}{3} + \frac{2}{3}i)x_3
          + (\frac{8}{3} - \frac{5}{3}i)x_4 &= 0.
      \end{align*}

      \noindent\rule{\textwidth}{1pt}

      We have systems
      \begin{alignat}{5}
        2x_1 & {}+{} & (-1 + i)x_2 & {} {} &       & {}+{} &  x_4 & {}={} & 0 \label{eq:11}\\
             & {} {} &        3x_2 & {}-{} & 2ix_3 & {}+{} & 5x_4 & {}={} & 0 \label{eq:12}
      \end{alignat}
      and
      \begin{alignat}{6}
        (1 + \frac{i}{2})x_1 & {}+{} &           8x_2 & {}-{} & ix_3 & {}-{} &  x_4 & {}={} & 0 & \label{eq:21} \\
              \frac{2}{3}x_1 & {}-{} & \frac{1}{2}x_2 & {}+{} & x_3 & {}+{} & 7x_4 & {}={}  & 0 &.\label{eq:22}
      \end{alignat}
      We would like to know if they are equivalent.\\\\
      This seems immediately unlikely. If we consider expressing
      \eqref{eq:11} as a linear combination of \eqref{eq:21} and
      \eqref{eq:22}, it's not clear that there is a way. For instance,
      consider that we use the coefficients $(1,i)$ with
      \eqref{eq:21} and \eqref{eq:22} respectively, i.e.
      \[
      \begin{array}{rcrcrcrcl}
        (1 + \frac{i}{2})x_1 & + & 8x_2 & - & ix_3 & - & x_4 & = & 0\\
        \frac{2}{3}ix_1 & - & \frac{1}{2}ix_2 & + & ix_3 & + & 7ix_4 & = & 0,
      \end{array}
      \]
      to produce a linear combination of them in which $x_3$ takes
      the coefficient $0$:
      \begin{align*}
        (1 + \frac{i}{2})x_1 + \frac{2}{3}ix_1 + 8x_2 - \frac{1}{2}ix_2 - ix_3 + ix_3 - x_4 + 7ix_4 &=\\
        (1 + \frac{7}{6}i)x_1 + (8 - \frac{1}{2}i)x_2 + (-1 + 7i)x_4 &= 0.
      \end{align*}
      This is obviously not the same as \eqref{eq:11}, nor is it
      obvious that we could adjust our coefficients to get there.

      In a sense, what we are looking for is a set of scalars
      $(c_1,c_2)$ such that
      \begin{alignat*}{6}
        (1 + \frac{i}{2})c_1 & {}+{} & \frac{2}{3}c_2 & {}={} &        2&\\
                        8c_1 & {}-{} & \frac{1}{2}c_2 & {}={} & (-1 + i)&\\
                       -ic_1 & {}+{} &            c_2 & {}={} &        0&\\
                        -c_1 & {}+{} &           7c_2 & {}={} &        1&,
      \end{alignat*}
      or
      \begin{alignat*}{7}
        (1 + \frac{i}{2})c_1 & {}+{} & \frac{2}{3}c_2 & {}-{} &       2 & {}={} & 0&\\
                        8c_1 & {}-{} & \frac{1}{2}c_2 & {}-{} &(-1 + i) & {}={} & 0&\\
                       -ic_1 & {}+{} &            c_2 & {} {} &         & {}={} & 0&\\
                        -c_1 & {}+{} &           7c_2 & {}-{} &       1 & {}={} & 0&.
      \end{alignat*}
      Now we have a new system of homogenous linear equations. We
      can apply the coefficient $-\frac{43}{6}$ to the third
      equation:
      \begin{align*}
        \frac{43}{6}ic_1 -\frac{43}{6}c_2 &= 0;
      \end{align*}
      this permits a linear combination of these equations in
      which $c_2$ is eliminated:
      \begin{align*}
        (1 + \frac{i}{2})c_1 + 8c_1 + \frac{43}{6}ic_1 - c_1
        + \frac{2}{3}c_2 - \frac{1}{2}c_2 - \frac{43}{6}c_2 + 7c_2
        - 2 -(-1 + i) - 1 &=\\
        (8 + \frac{46}{6}i)c_1
        + \frac{23}{3}c_2 - \frac{23}{3}c_2
        - (2 + i) &=\\
        (8 + \frac{46}{6}i)c_1
        - (2 + i) &=0.
      \end{align*}
      This permits us to solve for $c_1$:
      \begin{align*}
        c_1 &= (2 + i) \cdot (8 + i\frac{46}{6})^{-1}\\
        &= (2 + i) \cdot (\frac{8}{8^2 + (\frac{46}{6})^2} - i\frac{\frac{46}{6}}{8^2 + (\frac{46}{6})^2})\\
        &= (2 + i) \cdot (\frac{8}{64 + \frac{529}{9}} - i\frac{\frac{46}{6}}{64 + \frac{529}{9}})\\
        &= (2 + i) \cdot (\frac{8}{\frac{1105}{9}} - i\frac{\frac{46}{6}}{\frac{1105}{9}})\\
        &= (2 + i) \cdot (\frac{72}{1105} - i\frac{414}{6630})\\
        &= (2 + i) \cdot (\frac{72}{1105} - i\frac{69}{1105})\\
        &= (2\frac{72}{1105} + \frac{69}{1105} - i2\frac{69}{1105} + i\frac{72}{1105})\\
        &= (\frac{213}{1105} - i\frac{66}{1105}).
      \end{align*}
      We can substitute this value for $c_1$ in our system:
      \begin{alignat*}{7}
        (1 + i\frac{1}{2})(\frac{213}{1105} - i\frac{66}{1105}) & {}+{} & \frac{2}{3}c_2 & {}-{} &       2 & {}={} & 0&\\
                         8(\frac{213}{1105} - i\frac{66}{1105}) & {}-{} & \frac{1}{2}c_2 & {}-{} &(-1 + i) & {}={} & 0&\\
                        -i(\frac{213}{1105} - i\frac{66}{1105}) & {}+{} &            c_2 & {} {} &         & {}={} & 0&\\
                         -(\frac{213}{1105} - i\frac{66}{1105}) & {}+{} &           7c_2 & {}-{} &       1 & {}={} & 0&,
      \end{alignat*}
      i.e.
      \begin{alignat*}{7}
        (\frac{213}{1105} + \frac{66}{2210}- i\frac{66}{1105} + i\frac{213}{2210}) & {}+{} & \frac{2}{3}c_2 & {}-{} &       2 & {}={} & 0&\\
        (\frac{1704}{1105} - i\frac{528}{1105}) & {}-{} & \frac{1}{2}c_2 & {}-{} &(-1 + i) & {}={} & 0&\\
       (\frac{66}{1105} -i\frac{213}{1105}) & {}+{} &            c_2 & {} {} &         & {}={} & 0&\\
        (-\frac{213}{1105} + i\frac{66}{1105}) & {}+{} &           7c_2 & {}-{} &       1 & {}={} & 0&,
      \end{alignat*}
      i.e.
      \begin{alignat*}{7}
        (\frac{492}{2210} + i\frac{81}{2210}) & {}+{} & \frac{2}{3}c_2 & {}-{} &       2 & {}={} & 0&\\
        (\frac{3408}{2210} - i\frac{1056}{2210}) & {}-{} & \frac{1}{2}c_2 & {}-{} &(-1 + i) & {}={} & 0&\\
       (\frac{132}{2210} -i\frac{426}{2210}) & {}+{} &            c_2 & {} {} &         & {}={} & 0&\\
        (-\frac{426}{2210} + i\frac{132}{2210}) & {}+{} &           7c_2 & {}-{} &       1 & {}={} & 0&,
      \end{alignat*}
      i.e.
      \begin{alignat*}{6}
        (-\frac{3928}{2210} + i\frac{81}{2210}) & {}+{} & \frac{2}{3}c_2 & {}={} & 0&\\
        (\frac{5618}{2210} - i\frac{3266}{2210}) & {}-{} & \frac{1}{2}c_2 & {}={} & 0&\\
       (\frac{132}{2210} -i\frac{426}{2210}) & {}+{} &            c_2 & {}={} & 0&\\
        (-\frac{2636}{2210} + i\frac{132}{2210}) & {}+{} &           7c_2 & {}={} & 0&,
      \end{alignat*}
      then find a value for $c_2$ via the implied linear
      combination
      \begin{align*}
        \frac{49}{6}c_2 - (\frac{814}{2210} + i\frac{3479}{2210}) &= 0,
      \end{align*}
      i.e.
      \begin{align*}
        c_2 &= (\frac{814}{2210} + i\frac{3479}{2210}) \cdot \frac{6}{49}\\
        &= (\frac{2242}{54145} + i\frac{10437}{54145}).
      \end{align*}
      We now have two coefficients, $((\frac{213}{1105} -
      i\frac{66}{1105}), (\frac{2242}{54145} +
      i\frac{10437}{54145}))$, that \textit{might} yield
      \eqref{eq:11} via a linear combination of \eqref{eq:21} and
      \eqref{eq:22}. We get the system
      \begin{alignat*}{6}
        (\frac{213}{1105} - i\frac{66}{1105})(1 + \frac{i}{2})x_1 & {}+{} &           (\frac{213}{1105} - i\frac{66}{1105})8x_2 & {}-{} & (\frac{213}{1105} - i\frac{66}{1105})ix_3 & {}-{} &  (\frac{213}{1105} - i\frac{66}{1105})x_4 & {}={} & 0 &\\
              (\frac{2242}{54145} + i\frac{10437}{54145})\frac{2}{3}x_1 & {}-{} & (\frac{2242}{54145} + i\frac{10437}{54145})\frac{1}{2}x_2 & {}+{} & (\frac{2242}{54145} + i\frac{10437}{54145})x_3 & {}+{} & (\frac{2242}{54145} + i\frac{10437}{54145})7x_4 & {}={}  & 0 &,
      \end{alignat*}
      i.e.
      \begin{alignat*}{6}
        (\frac{246}{1105} + i\frac{81}{2210})x_1 & {}+{} &
        (\frac{1704}{1105} - i\frac{528}{1105})x_2 & {}-{} &
        (\frac{66}{1105} + i\frac{213}{1105})x_3 & {}-{} & 
        (\frac{213}{1105} - i\frac{66}{1105})x_4 & {}={} & 0 &\\
        (\frac{4484}{162435} + i\frac{142}{1105})x_1 & {}-{} &
        (\frac{1121}{54145} + i\frac{213}{2210})x_2 & {}+{} &
        (\frac{2242}{54145} + i\frac{10437}{54145})x_3 & {}+{} &
        (\frac{2242}{7735}+i\frac{1491}{1105})x_4 & {}={}  & 0 &,
      \end{alignat*}
      which yields the linear combination
      \begin{align*}
        (\frac{492}{1105} + i\frac{81}{1105})x_1
        + (\frac{22987}{54145} - i\frac{3}{130})x_2
        - \frac{992}{54145}x_3
        + (\frac{751}{7735} + i\frac{1557}{1105})x_4
        &= 0.
      \end{align*}
      Obviously this is not \eqref{eq:11}. This casts further
      doubt on the idea that the two systems we started with are
      equivalent, although this may not be the only solution of
      our new system.

      In any case, doubt is not proof. What we would really like
      is a way to say \textit{for sure} that they are not
      equivalent.

      One thing we could do is try finding solutions for both
      systems and see if we find any that they don't share. But
      this is ungraceful, and potentially laborious—one could
      imagine a situation where two systems shared many solutions
      and only differed on one each. What if you arrived at those
      solutions last?

      It would be nice to have a more efficient way to prove
      this. We need a way to say something about all the
      solutions of a system at once, or a more fleshed-out
      definition of equivalence; in this case whatever we find
      may turn out to be both.

      A "naïve" linear combination of the second system yields
      \begin{align*}
        (1 + \frac{i}{2})x_1 + \frac{2}{3}x_1 + 8x_2 - \frac{1}{2}x_2 - ix_3 + x_3 - x_4 + 7x_4 &=\\
        (\frac{5}{2} + \frac{i}{2})x_1 + \frac{15}{2}x_2 + (1 - i)x_3 + 6x_4 &= 0.
      \end{align*}
      If we multiply the first equation of the second system by
      $(\frac{16}{15} - \frac{8}{15}i)$, we get
      \begin{align*}
        (1 + \frac{i}{2})(\frac{16}{15} - \frac{8}{15}i)x_1 + 8(\frac{16}{15} - \frac{8}{15}i)x_2 - i(\frac{16}{15} - \frac{8}{15}i)x_3 - (\frac{16}{15} - \frac{8}{15}i)x_4 &=\\
        \frac{4}{3}x_1 + (\frac{128}{15} - \frac{64}{15}i)x_2 -
        (\frac{8}{15} + \frac{16}{15}i)x_3 - (\frac{16}{15} -
        \frac{8}{15}i)x_4 &= 0.
      \end{align*}

      $(1 + \frac{i}{2})x = \frac{4}{3}$\\
      $x = \frac{4}{(3 + \frac{3i}{2})}$\\
      $(3, \frac{3}{2})^{-1} = (\frac{3}{3^{2} + (\frac{3}{2})^2}, -\frac{\frac{3}{2}}{3^{2} + (\frac{3}{2})^2})$\\
      $(\frac{3}{\frac{45}{4}}, -\frac{3}{2}(\frac{1}{\frac{45}{4}}))$\\
      $(\frac{3(4)}{45}, -\frac{3}{2}(\frac{4}{45}))$\\
      $(\frac{12}{45}, -\frac{12}{90})$\\
      $(\frac{4}{15}, -\frac{2}{15})$\\

      $4(\frac{4}{15} - \frac{2i}{15})(1 + \frac{i}{2})$\\
      $4(\frac{4}{15}, -\frac{2}{15})(1, \frac{1}{2})$\\
      $4(\frac{4}{15} + \frac{2}{30}, \frac{4}{30} - \frac{2}{15})$\\
      $4(\frac{1}{3}, 0)$\\

      $(\frac{16}{15} - \frac{8}{15}i)$

      \noindent\rule{\textwidth}{1pt}

      Multiplying the first equation by $-5$ gives
      \[
      \begin{array}{rcrcrcrcl}
        -10x_1 & + & (5 - 5i)x_2 &   &       & - & 5x_4 & = & 0,
      \end{array}
      \]
      which yields the linear combination
      \begin{align*}
        2x_1 + (5 - 5i)x_2 + 3x_2 - 2ix_3 - 5x_4 + 5x_4 &=\\
        2x_1 + (8 - 5i)x_2 - 2ix_3 &= 0.
      \end{align*}
      Multiplying the first equation by $(\frac{3}{2} + \frac{3}{2}i)$ gives
      \begin{align*}
        (\frac{3}{2} + \frac{3}{2}i)2x_1 + (\frac{3}{2} + \frac{3}{2}i)(-1 + i)x_2 + (\frac{3}{2} + \frac{3}{2}i)x_4 &=\\
        (3 + 3i)x_1 - 3x_2 + (\frac{3}{2} + \frac{3}{2}i)x_4 &= 0,
      \end{align*}
      which yields the linear combination
      \begin{align*}
        (3 + 3i)x_1 - 3x_2 + 3x_2 - 2ix_3 + (\frac{3}{2} + \frac{3}{2}i)x_4 + 5x_4 &=\\
        (3 + 3i)x_1 - 2ix_3 + (\frac{13}{2} + \frac{3}{2}i)x_4 &= 0.
      \end{align*}
      This implies that
      \begin{align*}
        2x_1 + (8 - 5i)x_2 &= 2ix_3,
      \end{align*}
      and also that
      \begin{align*}
        (3 + 3i)x_1 + (\frac{13}{2} + \frac{3}{2}i)x_4 &= 2ix_3,
      \end{align*}
      which means that
      \begin{align*}
        2x_1 + (8 - 5i)x_2 &= (3 + 3i)x_1 + (\frac{13}{2} + \frac{3}{2}i)x_4.
      \end{align*}

  \item
    I'm pretty sure I've done this one before, more or less.

  \item
    \begin{thm}
      If two homogenous systems of linear equations in two
      unknowns have the same solutions, then they are equivalent.
      \begin{proof}
        We have
        \begin{equation} \label{eq:61}
        \begin{array}{cccccccccc}
          A_{11}x_1 & + & A_{12}x_2 & = & 0\\
          A_{21}x_1 & + & A_{22}x_2 & = & 0\\
          \vdots    & + & \vdots    & = & \vdots\\
          A_{m1}x_1 & + & A_{m2}x_2 & = & 0
        \end{array}
        \end{equation}
        and
        \begin{equation} \label{eq:62}
        \begin{array}{cccccccccc}
          B_{11}y_1 & + & B_{12}y_2 & = & 0\\
          B_{21}y_1 & + & B_{22}y_2 & = & 0\\
          \vdots    & + & \vdots    & = & \vdots\\
          B_{n1}y_1 & + & B_{n2}y_2 & = & 0.
        \end{array}
        \end{equation}
        A solution for \eqref{eq:61} is a pair of numbers
        $(x_1,x_2)$ such that
        \begin{align*}
        (A_{11} + \ldots + A_{m1})x_1
          + (A_{12} + \ldots + A_{m2})x_2
          &= 0,
        \end{align*}
        and solution for \eqref{eq:62} is a pair of numbers
        $(y_1,y_2)$ such that
        \begin{align*}
        (B_{11} + \ldots + B_{n1})y_1
          + (B_{12} + \ldots + B_{n2})y_2
          &= 0.
        \end{align*}
        If \eqref{eq:61} and \eqref{eq:62} have the same
        solutions, then the set of all $(x_1,x_2)$ is the same as
        the set of all $(y_1,y_2)$. Let's indicate all the pairs
        in this set by $(x_{1r},x_{2r})$. We can then say that
        \begin{align*}
          (A_{11} + \ldots + A_{m1})x_{1r}
          + (A_{12} + \ldots + A_{m2})x_{2r}
          &= 0
        \end{align*}
        and that
        \begin{align*}
        (B_{11} + \ldots + B_{n1})x_{1r}
          + (B_{12} + \ldots + B_{n2})x_{2r}
          &= 0.
        \end{align*}
        Let's denote $(A_{11} + \ldots + A_{m1})$ as $a_1$ and
        $(A_{12} + \ldots + A_{m2})$ as $a_2$, and likewise for
        the $B$s. We then have
        \begin{equation*}
        \begin{array}{cccccc}
          a_{1}x_1 & + & a_{2}x_2 & = & 0 &\\
          b_{1}x_1 & + & b_{2}x_2 & = & 0 &
        \end{array}
        \end{equation*}
        for all $(x_{1r},x_{2r})$.

        If we take an equation from \eqref{eq:61}
        \begin{align*}
          A_{i1}x_1 + A_{i2}x_2 = 0
        \end{align*}
        where $i$ is a number between $1$ and $m$, we know that
        \begin{align*}
          A_{i1}x_1 + A_{i2}x_2 = b_{1}x_1 + b_{2}x_2.
        \end{align*}
        This implies that
        \begin{align*}
          (b_{1} - A_{i1})x_1 + (b_{2} - A_{i2})x_2 = 0,
        \end{align*}
        which means there is some pair of coefficients $(c_1,c_2)
        = (1 - \frac{A_{i1}}{b_{1}}, 1 - \frac{A_{i2}}{b_{2}})$
        (presuming neither $b$ is $0$, in which case the entire
        factor can be disregarded) such that
        \begin{align*}
          c_{1}b_{1}x_1 + c_{2}b_{2}x_2 = 0.
        \end{align*}
        We can thus conclude that there is a system of linear
        equations
        \begin{equation*}
        \begin{array}{cccccccccc}
          c_{1}B_{11}x_1 & + & c_{1}B_{12}x_2 & = & 0\\
          c_{2}B_{21}x_1 & + & c_{2}B_{22}x_2 & = & 0\\
          \vdots         & + & \vdots         & = & \vdots\\
          c_{m}B_{n1}x_1 & + & c_{m}B_{n2}x_2 & = & 0
        \end{array}
        \end{equation*}
        such that
        \begin{equation*}
          (c_{1}B_{11} + \ldots + c_{n}B_{n1})x_1 + (c_{1}B_{12}
          + \ldots + c_{n}B_{n2})x_2 = A_{i1}x_1 + A_{i2}x_2,
        \end{equation*}
        which is the same as saying that $A_{i1}x_1 + A_{i2}x_2$
        is a linear combination of the equations in
        \eqref{eq:62}. Since we've shown this to be true for
        every equation in \eqref{eq:61}, and we can apply the
        same process in the other direction for \eqref{eq:62}, we
        can now see that they must be equivalent.
      \end{proof}
    \end{thm}

  \item
    See 1.1.

  \item
    \begin{thm}
      Each field of characteristic zero contains a copy of the
      rational number field.
      \begin{proof}
        \begin{lemma}
          Each field of characteristic zero contains a copy of
          the natural numbers.
          \begin{proof}
            First, we briefly describe the natural numbers. We
            consider $0$ as the empty set $\{\}$. We then define a
            function $S(A) = A \cup \{A\}$ where $A$ is a set, i.e.
            $S$ maps $A$ to the set that contains all the elements of
            $A$ as well as $A$ itself. We say that the natural
            numbers are the set of all sets formed by recursive
            application of $S$ to $0$, and take it for granted that
            this set is infinite: $1 = S(0) = \{\{\}\}$, $2 = S(1) =
            \{\{\{\}\}\}$, and so on.

            In this context, we can see that $A + 1$ in the natural
            numbers can be satisfactorily defined as $S(A)$, and that
            $A + N$ is then equivalent to applying $S$ to $A$
            recursively a number of times equal to the number of sets
            contained in $N$.  So, $A + 0 = A$, $A + 2 = S(S(A))$,
            etc.

            $NA$ in the natural numbers can then be defined as
            recursively applying $S$ to $0$ for every member of $A$
            for every member of $N$. So, $0A$ is $0$ with $S$ applied
            to it no times i.e. $0$, $1A$ is $0$ with $S$ applied to
            it number-of-elements-in-$A$ times i.e. $A$, etc.

            In this manner we see that both addition and
            multiplication in the natural numbers can be defined
            based on $S$.

            A field of characteristic zero is one such that $1 +
            1 + \ldots + 1 \neq 0$ for any number of $1$s. We
            represent the summation of $n$ $1$s here by
            $\sum_{1}^{n}1$, so we could also say that a field of
            characteristic zero is one in which $\sum_{1}^{n}1
            \neq 0$ for any $n$.

            In a field that is of characteristic $q \neq 0$, this
            summation will cycle at $q$; in other words,
            $\sum_{1}^{q}1 = \sum_{1}^{2q}1 = \ldots = \sum_{1}^{pq}1
            = 0$ for any natural number $p$. $0$ is the additive
            identity in any field, i.e. $x + 0 = x$ for any scalar
            $x$. This implies that any field not of characteristic
            zero cannot contain a copy of the natural numbers; the
            range of a function like $S$ in that field will be
            finite.

            By the same token, this implies that any field of
            characteristic zero \textit{will} contain a copy of
            the natural numbers. If $\sum_{1}^{n}1 \neq 0$, every
            scalar from $1$ up to $n$ must be unique for any
            natural number $n$; if there was some $x \neq n$ such
            that $\sum_{1}^{x}1 = \sum_{1}^{n}1$, it would imply
            that this operation was cyclical and thus that the
            field was not of characteristic zero. This means that
            the natural numbers can be constructed in any field
            of characteristic zero starting with that field's $0$
            and using a definition of $S$ as $S(n) = n + 1$ in
            that field.
          \end{proof}
        \end{lemma}

        We consider $A - 1$ in the natural numbers as equal to
        the set $N$ such that $S(N) = A$ in the above lemma, and
        $A - N$ as equal to the set through which $A$ can be
        obtained by number-of-elements-in-$N$ applications of
        $S$. Based on this definition, the number of elements in
        $A$ must be equal to or larger than the number of
        elements in $N$ for the operation to make sense, as there
        is no set $X$ such that $S(X) = 0$.

        This restriction can be worked around in the set of all
        pairs of natural numbers $(a,b)$. We can say in this
        context that given two such pairs $(a,b)$ and $(c,d)$,
        they are \textit{equal} if $a + d$ yields the same set as
        $b + c$.

        We can then define $(a,b) + (c,d)$ as $(a + c, b + d)$,
        and $(a,b) - (c,d)$ as $(a,b) + (d,c)$ or $(a + d, b +
        c)$. If we represent the natural number $0$ as $(0,0)$
        and the natural number $a$ as $(a,0)$, we can see that
        these operations work as we would expect: $(0,0) + (a,0)
        = (0 + a, 0 + 0) = (a,0)$, and $(a,0) - (a,0) = (a + 0, 0
        + a) = (a,a) = (0,0)$. $(a,a) = (0,0)$ because $a + 0 = a
        + 0$. Since this definition relies only on the addition
        of natural numbers, we don't have to worry about
        subtraction taking us past $0$: $(0,0) - (a,0) = (0,a)$,
        which is still a pair of natural numbers. Notably, we
        have additive inverses in this context for all the
        elements in the set: $(a,b) - (a,b) = (a + b, b + a) =
        (0,0)$.

        We can define $(a,b) \cdot (c,d)$ as $(ac + bd, ad +
        bc)$. This also works as we would expect: $(a,a) \cdot
        (1,0) = (a(1) + a(0), a(0) + a(1)) = (a,a)$ and so on.

        As you may have suspected, this is a satisfactory
        construction of the integers. One downside of this
        system is that multiplicative inverses can be hard to
        come by. For example, consider $2$—what integer $(a,b)$
        is there such that $(2,0) \cdot (a,b) = (1,0)$? We would
        want $(2a + 0b, 2b + 0a) = (2a,2b) = (1,0)$. There is no
        natural number $a$ such that $2a = 1$, so we're out of
        luck. In fact, the only numbers for which this will work
        are $(1,0)$ and $(0,1)$, as you can easily see.

        To get around \textit{this} problem, we can construct the
        rational numbers. We say the rational numbers are the set
        of all pairs of integers $(a,b)$ where $b \neq 0$. We
        call two rational numbers $(a,b)$ and $(c,d)$ equal if
        $ad = bc$. Addition is defined between rational numbers
        $(x,y)$ and $(z,y)$ such that $(x,y) + (z,a) = (xa + yz,
        ya)$.  Multiplication is defined such that $(x,y) \cdot
        (z,a) = (xz,ya)$.

        We can represent the integers in the rationals as all the
        rational numbers $(n,1)$, where $n$ is an integer.
        Additive inverses can thus be found for all the integers
        in the rationals via $(1,n)$; $(n,1) \cdot (1,n) =
        (1n,1n) = (n,n) = (1,1)$, because $n1 = n1$. In fact,
        they can be found for any rational number $\neq 0$:
        $(a,b) \cdot (b,a) = (ab,ba) = (1,1)$.

        At this point, we can see that all the elements and
        fundamental operations that we need to have the rational
        number field can be constructed using only the natural
        numbers and addition and multiplication over them. Since
        any field of characteristic zero contains a copy of the
        natural numbers that supports these operations by our
        lemma, this implies that any such field also contains a
        copy of the rational number field.
      \end{proof}
    \end{thm}
\end{enumerate}

\end{document}
